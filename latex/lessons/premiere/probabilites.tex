\documentclass[12pt, a4paper]{report}

% Packages :

\usepackage[french]{babel}
%\usepackage[utf8]{inputenc}
%\usepackage[T1]{fontenc}
\usepackage[pdfencoding=auto, pdfauthor={Bacomathiques}]{hyperref}
\usepackage{sectsty}
\usepackage[explicit]{titlesec}
\usepackage{xcolor}
%\usepackage{amsmath}
%\usepackage{amssymb}
\usepackage{amsthm}
\usepackage{fourier-otf}
\usepackage{titlesec}
\usepackage{fancyhdr}
\usepackage{catchfilebetweentags}
\usepackage[french, capitalise, noabbrev]{cleveref}
\usepackage[fit, breakall]{truncate}
\usepackage[margin=3cm]{geometry}
\usepackage{tocloft}
\usepackage{tikz}
\usepackage{tocloft}
\usepackage{microtype}
\usepackage{listings}
\usepackage{tabularx}
\usepackage{calc}
\usepackage[export]{adjustbox}
\usepackage[most]{tcolorbox}
\usepackage{standalone}
\usepackage{xlop}
\usepackage{etoolbox}
\usepackage{environ}

\usetikzlibrary{arrows.meta}
\usetikzlibrary{trees}

% Police :

\setmathfont{Erewhon Math}

% Paramètres :

\author{Bacomathiques}
\definecolor{graphe}{HTML}{93c9ff}
\setcounter{MaxMatrixCols}{12}
\setlength{\parindent}{0pt}
\setlength{\fboxsep}{0pt}
%\pdfsuppresswarningpagegroup=1

% Code :

\lstdefinestyle{style}{
	backgroundcolor=\color{white},
	commentstyle=\em\color[HTML]{999988},
	keywordstyle=\bfseries,
	identifierstyle=\normalfont,
	stringstyle=\color[rgb]{0.87, 0.07, 0.27},
	basicstyle=\ttfamily\color{black},
	breakatwhitespace=false,
	breaklines=true,
	captionpos=b,
	keepspaces=true,
	numbers=left,
	numbersep=5pt,
	showspaces=false,
	showstringspaces=false,
	showtabs=false,
	tabsize=2,
	numbers=none
}

\lstset{style=style}
\lstset{
	literate=
	{á}{{\'a}}1
	{à}{{\`a}}1
	{ã}{{\~a}}1
	{é}{{\'e}}1
	{ê}{{\^e}}1
	{í}{{\'i}}1
	{ó}{{\'o}}1
	{õ}{{\~o}}1
	{ú}{{\'u}}1
	{ü}{{\"u}}1
	{ç}{{\c{c}}}1
}

\lstset{
	framextopmargin=10pt,
	framexrightmargin=10pt,
	framexbottommargin=10pt,
	framexleftmargin=10pt,
	xleftmargin=10pt,
	xrightmargin=10pt,
}

% Couleurs :

\definecolor{title}{HTML}{912c21}
\definecolor{section}{HTML}{1c567d}
\definecolor{subsection}{HTML}{2980b9}

\definecolor{rule}{HTML}{c4c4c4}

\definecolor{formula}{HTML}{ebf3fb}
\definecolor{formula-left}{HTML}{3583d6}

\definecolor{tip}{HTML}{dcf3d8}
\definecolor{tip-left}{HTML}{26a65b}

\definecolor{demonstration}{HTML}{fff8de}
\definecolor{demonstration-left}{HTML}{f1c40f}

\definecolor{exercise}{HTML}{e0f2f1}
\definecolor{exercise-left}{HTML}{009688}

\definecolor{correction}{HTML}{e0f7fa}
\definecolor{correction-left}{HTML}{00bcd4}

\definecolor{toc}{HTML}{fceae9}
\definecolor{toc-left}{HTML}{e74c3c}
\definecolor{toc-dark}{HTML}{87281f}

% Titres :

\renewcommand{\thesection}{\Roman{section} - }
\renewcommand{\thesubsection}{\arabic{subsection}. }

\newcommand{\sectionstyle}{\normalfont\LARGE\bfseries\color{section}}
\titleformat{\section}{\sectionstyle}{\thesection #1}{0pt}{}
\titleformat{name=\section, numberless}{\sectionstyle}{#1}{0pt}{}

\newcommand{\subsectionstyle}{\normalfont\Large\bfseries\color{subsection}}
\titleformat{\subsection}{\subsectionstyle}{\thesubsection #1}{0pt}{}
\titleformat{name=\subsection, numberless}{\subsectionstyle}{#1}{0pt}{}

\titlelabel{\thetitle\ }

% Table des matières :

\addto\captionsfrench{\renewcommand\contentsname{}}
\renewcommand{\cftsecpagefont}{\color{toc-dark}}
\renewcommand{\cftsubsecpagefont}{\color{toc-dark}}
\renewcommand{\cftsecleader}{\cftdotfill{\cftdotsep}}
\renewcommand{\cftsecfont}{\bfseries}
\renewcommand{\cftsecpagefont}{\bfseries\color{toc-dark}}
\setlength{\cftbeforetoctitleskip}{0pt}
\setlength{\cftaftertoctitleskip}{0pt}
\setlength{\cftsecindent}{0pt}
\setlength{\cftsubsecindent}{20pt}
\setlength{\cftsubsecnumwidth}{20pt}

% Commandes :

\newcommand{\newpar}{\\[\medskipamount]}
\newcommand{\lesson}[3]{%
	\newcommand{\level}{#1}%
	\newcommand{\id}{#2}%
	\hypersetup{pdftitle={#3}}
	\begin{center}%
		\includegraphics[width=150px]{\imagespath/bacomathiques}%
		
		\vspace{30pt}%
		{\Huge\color{title} #3}%
		
		\vspace{10pt}%
		{Bacomathiques --- \href{https://bacomathiqu.es/cours/#1/#2}{\color{section} https://bacomathiqu.es}}%
		
		\vspace{20pt}%
	\end{center}%
	\begin{toc}
		\tableofcontents%
	\end{toc}
	\thispagestyle{empty}%
	\newpage%
	\setcounter{page}{1}%
}
\newcommand{\imagespath}{../../images}
\newcommand{\lessonimagespath}{\imagespath/lessons/\level/\id/}
\newcommand{\includelatexpicture}[2][\textwidth - 100pt]{%
	\begin{center}%
		\resizebox{#1}{!}{%
			\input{\lessonimagespath#2}%
		}%
	\end{center}%
	\medskip%
}
\newcommand{\includeimage}[1]{%
	\begin{center}%
		\includegraphics{\lessonimagespath#1}%
	\end{center}%
	\medskip%
}
\newcommand{\includerepresentation}[1]{%
	\begin{center}%
		\setlength{\fboxrule}{0.5pt}%
		\href{https://www.geogebra.org/m/#1}{\includegraphics[width=\textwidth-1pt,fbox]{\lessonimagespath#1}}%
	\end{center}%
}
\newcommand{\floor}[1]{\lfloor #1 \rfloor}

\makeatletter
\newcommand\inputcontent{\@ifstar{\inputcontent@star}{\inputcontent@nostar}}
\newcommand{\inputcontent@star}[1]{%
	\ExecuteMetaData[#1]{content}%
}
\newcommand{\inputcontent@nostar}[1]{%
	\newpage%
	\inputcontent@star{#1}%
}
\makeatother

\let\oldsection\section
\renewcommand\section{\clearpage\oldsection}
\newcommand{\contentwidth}[1][medium]{}

% En-têtes :

\pagestyle{fancy}

\renewcommand{\sectionmark}[1]{\markboth{\thesection \ #1}{}}

\fancyhead[R]{\truncate{0.23\textwidth}{\color{title}\thepage}}
\fancyhead[L]{\truncate{0.73\textwidth}{\color{title}\leftmark}}
\fancyfoot[C]{\scriptsize \href{https://bacomathiqu.es/cours/\level/\id}{\texttt{bacomathiqu.es}}}

\makeatletter
\patchcmd{\f@nch@head}{\rlap}{\color{rule}\rlap}{}{}
\patchcmd{\headrule}{\hrule}{\color{rule}\hrule}{}{}
\makeatother

% Environnements :

\newenvironment{nosummary}{}{}
\newcommand{\tcolorboxtitle}[2]{\setlength{\fboxsep}{2.5pt}\hspace{-10pt}\colorbox{#1-left}{\hspace{8pt}\MakeUppercase{#2} \hspace{2pt} \includegraphics[height=0.8em]{\imagespath/bubbles/#1}\hspace{5pt}}}
\newcommand{\tcolorboxsubtitle}[2]{\ifstrempty{#2}{}{\textcolor{#1-left}{\large#2}\\[\medskipamount]}}
\tcbset{
	frame hidden,
	boxrule=0pt,
	boxsep=0pt,
	enlarge bottom by=8.5pt,
	enhanced jigsaw,
	boxed title style={sharp corners,boxrule=0pt,coltitle={white},titlerule=0pt},
	fonttitle=\fontsize{6pt}{6pt}\bfseries\boldmath,
	top=10pt,
	right=10pt,
	bottom=10pt,
	left=10pt,
	arc=0pt,
	outer arc=0pt,
}
\newtcolorbox{toc}[1][]{
	colback=toc,
	borderline west={3pt}{0pt}{toc-left},
	title=\tcolorboxtitle{toc}{Table des matières},
	colbacktitle=toc,
	before upper={\tcolorboxsubtitle{toc}{#1}}
}
\newtcolorbox{formula}[1][]{
	colback=formula,
	borderline west={3pt}{0pt}{formula-left},
	title=\tcolorboxtitle{formula}{À retenir},
	colbacktitle=formula,
	before upper={\tcolorboxsubtitle{formula}{#1}}
}
\newtcolorbox{tip}[1][]{
	colback=tip,
	borderline west={3pt}{0pt}{tip-left},
	title=\tcolorboxtitle{tip}{À lire},
	colbacktitle=tip,
	before upper={\tcolorboxsubtitle{tip}{#1}}
}
\newtcolorbox{demonstration}[1][]{
	colback=demonstration,
	borderline west={3pt}{0pt}{demonstration-left},
	title=\tcolorboxtitle{demonstration}{Démonstration},
	colbacktitle=demonstration,
	before upper={\tcolorboxsubtitle{demonstration}{#1}}
}

\NewEnviron{whitetabularx}[1]{%
	\renewcommand{\arraystretch}{2.5}
	\colorbox{white}{%
		\begin{tabularx}{\textwidth}{#1}%
			\BODY%
		\end{tabularx}%
	}%
}

% Longueurs :

\newlength{\espacetitreliste}
\setlength{\espacetitreliste}{-16pt}
\newcommand{\entretitreetliste}{\vspace{\espacetitreliste}}

\begin{document}
	%<*content>
	\lesson{premiere}{probabilites}{Chapitre VII – Probabilités}

	\section{Probabilités conditionnelles}
	
	\subsection{Définition}
	
	\begin{formula}[Définition]
		Soient $A$ et $B$ deux événements avec $A$ de probabilité non nulle. Alors \textbf{la probabilité conditionnelle de $B$ sachant que $A$ est réalisé} (notée $P_{A}(B)$) est $\displaystyle{P_{A}(B) = \frac{P(A \cap B)}{P(A)}}$.
	\end{formula}
	
	\begin{tip}[Rappel]
		On rappelle que $P(A \cap B) = P(A) + P(B) - P(A \cup B)$.
	\end{tip}
	
	\begin{tip}[Différence entre conditionnelle et intersection]
		\textbf{Il faut faire attention}, à bien faire la distinction entre une probabilité conditionnelle (``\textbf{Sachant qu'on a $A$}, quelle est la probabilité d'avoir $B$ ?'') et une intersection (``Quelle est la probabilité d'avoir \textbf{$A$ et $B$ à la fois} ?'').
	\end{tip}
	
	\begin{formula}[Indépendance]
		Deux événements $A$ et $B$ sont dits \textbf{indépendants} si la réalisation de l'un n'a aucune incidence sur la réalisation de l'autre et réciproquement. C'est-à-dire si $P(A \cap B) = P(A) \times P(B)$.
	\end{formula}
	
	\begin{formula}[Propriétés]
		Pour deux événements indépendants $A$ et $B$, on a les relations suivantes :
		\begin{itemize}
			\item $P_{A}(B) = P(B)$
			\item $P_{B}(A) = P(A)$
		\end{itemize}
	\end{formula}
	
	\subsection{Arbre de probabilité}
	
	Au lycée, pour représenter visuellement des probabilités on utilise très souvent un \textbf{arbre de probabilité}. Nous nous limiterons ici au cas de deux événements, mais il est possible d'en rajouter encore d'autres.
	
	Ainsi :
	
	\begin{formula}[Définition]
		Soient $A$ et $B$ deux événements. L'arbre de probabilité décrivant la situation est le suivant :
		\includelatexpicture{arbre-1}
	\end{formula}
	
	La somme (dans le sens vertical) des probabilités de chacune des branches ayant une ``racine'' commune doit toujours faire $1$.
	
	\begin{tip}[Exemple]
		Soit $A$ et $B$ deux événements non-indépendants tels que $P(A) = \frac{4}{7}$, $P_{A}(B) = \frac{1}{4}$ et $P_{\bar{A}}(B) = \frac{5}{9}$.
		\newline
		Alors l'arbre permettant de modéliser la situation est le suivant :
		\includelatexpicture{arbre-2}
	\end{tip}
	
	\subsection{Formule des probabilités totales}
	
	Voici maintenant l'énoncé de la \textbf{formule des probabilités totales}, qui peut être très utile pour calculer des probabilités que l'on ne connaît pas (ou qui ne sont pas données dans un énoncé d'exercice) :
	
	\begin{formula}[Formule des probabilités totales]
		Soient $A_1, A_2, ..., A_n$ des événements qui partitionnent (qui recouvrent) l'univers $\Omega$, alors pour tout événement $B$ :
		\newpar
		$P(B) = P(B \cap A_1) + P(B \cap A_2) + \dots + P(B \cap A_n)$
	\end{formula}
	
	\begin{tip}[Exemple]
		En reprenant l'arbre précédent, comme $A$ et $\bar{A}$ recouvrent notre univers (en effet, soit on tombe sur $A$, soit on tombe sur $\bar{A}$ : pas d'autre issue possible), calculons $P(B)$ :
		\newpar
		\includelatexpicture{arbre-2}
		D'après la formule des probabilités totales, $P(B) = P(B \cap A) + P(B \cap \bar{A}) = \frac{107}{252}$.
	\end{tip}
	
	\section{Variables aléatoires}
	
	\subsection{Définition}
	
	\begin{formula}[Définition]
		Une \textbf{variable aléatoire} $X$ est une fonction qui, à chaque événement élémentaire de l'univers $\Omega$ y associe un nombre réel. C'est-à-dire : $X : \Omega \rightarrow \mathbb{R}$.
	\end{formula}
	
	L'ensemble des valeurs prises par $X$ est noté $X(\Omega)$.
	
	\begin{tip}
		Les variables aléatoires sont très utiles notamment pour modéliser des situations de gains ou de pertes (à un jeu d'argent par exemple).
	\end{tip}
	
	\subsection{Loi de probabilité}
	
	\begin{formula}[Définition]
		Soit $X$ une variable aléatoire. La \textbf{loi de probabilité} de $X$ attribue à chaque valeur $x_i$ la probabilité $p_i = P(X = x_i)$ de l'événement $X = x_i$ constitué de tous les événements élémentaires dont l'image par $X$ est $x_i$. 
	\end{formula}
	
	On représente généralement les lois de probabilité par un tableau.
	
	\begin{formula}[Représentation d'une loi de probabilité par un tableau]
		Soit $X$ une variable aléatoire. On peut représenter sa loi de probabilité par le tableau ci-contre :
		\newpar
		\colorbox{white}{%
			\begin{tabularx}{\textwidth}{|X|X|X|X|X|}
				\hline
				$x_i$ & $x_1$ & $x_2$ & ... & $x_n$ \\
				\hline
				$p_i$ \newline $= P(X = x_i)$ & $p_1$ \newline $= P(X = x_1)$ & $p_2$ \newline $= P(X = x_2)$ & ... & $p_n$ \newline $= P(X = x_n)$ \\
				\hline
			\end{tabularx}%
		}
		\newpar
		On a $p_1 + p_2 + \dots + p_n = 1$.
	\end{formula}
	
	\begin{tip}
		Cette définition peut sembler un peu compliquée mais elle signifie juste qu'une loi de probabilité assigne une probabilité à chaque valeur prise par notre variable aléatoire.
	\end{tip}
	
	\subsection{Espérance, variance et écart-type}
	
	\begin{formula}[Espérance]
		L'\textbf{espérance} $E(X)$ d'une variable aléatoire $X$ est le réel :
		$E(X) = x_1 \times p_1 + x_2 \times p_2 + \dots + x_n \times p_n$.
	\end{formula}
	
	\begin{formula}[Variance et écart-type]
		La \textbf{variance} $V(X)$ et l'\textbf{écart-type} $\sigma(X)$ d'une variable aléatoire $X$ sont les réels positifs suivants :
		\begin{itemize}
			\item $V(X) = E(X^2) - E(X)^2$
			\item $\sigma(X) = \sqrt{V(X)}$
		\end{itemize}
	\end{formula}
	
	\begin{tip}[Exemple]
		Calcul de l'espérance, de la variance et de l'écart-type. Soit $X$ une variable aléatoire suivant la loi de probabilité donnée par le tableau ci-dessous :
		\newpar
		\colorbox{white}{%
			\begin{tabularx}{\textwidth}{|X|X|X|X|X|}
				\hline
				$x_i$ & $-1$ & $0$ & $2$ & $6$ \\
				\hline
				$p_i$ & $\frac{1}{4}$ & $\frac{1}{2}$ & $\frac{1}{8}$ & $\frac{1}{8}$ \\
				\hline
			\end{tabularx}%
		}
		\newpar
		On a :
		\begin{itemize}
			\item $E(X) = -1 \times \frac{1}{4} + 0 \times \frac{1}{2} + 2 \times \frac{1}{8} + 6 \times \frac{1}{8} = \frac{3}{4}$
			\item $V(X) = ((-1)^2 \times \frac{1}{4} + 0^2 \times \frac{1}{2} + 2^2 \times \frac{1}{8} + 6^2 \times \frac{1}{8}) - (\frac{3}{4})^2 = \frac{75}{16}$
			\item $\sigma(X) = \sqrt{\frac{75}{16}} \approx 2.165$
		\end{itemize}
	\end{tip}
	
	Chacun de ces paramètres a une utilité bien précise. En effet :
	
	\begin{formula}[Signification des paramètres]
		\begin{itemize}
			\item L'espérance est la \textbf{valeur moyenne} prise par $X$.
			\item La variance et l'écart-type mesurent la \textbf{dispersion} des valeurs prises par $X$. Plus ces valeurs sont grandes, plus les valeurs sont dispersées autour de l'espérance.
		\end{itemize}
	\end{formula}
  %</content>
\end{document}
