\documentclass[12pt, a4paper]{report}

% Packages :

\usepackage[french]{babel}
%\usepackage[utf8]{inputenc}
%\usepackage[T1]{fontenc}
\usepackage[pdfencoding=auto, pdfauthor={Bacomathiques}]{hyperref}
\usepackage{sectsty}
\usepackage[explicit]{titlesec}
\usepackage{xcolor}
%\usepackage{amsmath}
%\usepackage{amssymb}
\usepackage{amsthm}
\usepackage{fourier-otf}
\usepackage{titlesec}
\usepackage{fancyhdr}
\usepackage{catchfilebetweentags}
\usepackage[french, capitalise, noabbrev]{cleveref}
\usepackage[fit, breakall]{truncate}
\usepackage[margin=3cm]{geometry}
\usepackage{tocloft}
\usepackage{tikz}
\usepackage{tocloft}
\usepackage{microtype}
\usepackage{listings}
\usepackage{tabularx}
\usepackage{calc}
\usepackage[export]{adjustbox}
\usepackage[most]{tcolorbox}
\usepackage{standalone}
\usepackage{xlop}
\usepackage{etoolbox}
\usepackage{environ}

\usetikzlibrary{arrows.meta}
\usetikzlibrary{trees}

% Police :

\setmathfont{Erewhon Math}

% Paramètres :

\author{Bacomathiques}
\definecolor{graphe}{HTML}{93c9ff}
\setcounter{MaxMatrixCols}{12}
\setlength{\parindent}{0pt}
\setlength{\fboxsep}{0pt}
%\pdfsuppresswarningpagegroup=1

% Code :

\lstdefinestyle{style}{
	backgroundcolor=\color{white},
	commentstyle=\em\color[HTML]{999988},
	keywordstyle=\bfseries,
	identifierstyle=\normalfont,
	stringstyle=\color[rgb]{0.87, 0.07, 0.27},
	basicstyle=\ttfamily\color{black},
	breakatwhitespace=false,
	breaklines=true,
	captionpos=b,
	keepspaces=true,
	numbers=left,
	numbersep=5pt,
	showspaces=false,
	showstringspaces=false,
	showtabs=false,
	tabsize=2,
	numbers=none
}

\lstset{style=style}
\lstset{
	literate=
	{á}{{\'a}}1
	{à}{{\`a}}1
	{ã}{{\~a}}1
	{é}{{\'e}}1
	{ê}{{\^e}}1
	{í}{{\'i}}1
	{ó}{{\'o}}1
	{õ}{{\~o}}1
	{ú}{{\'u}}1
	{ü}{{\"u}}1
	{ç}{{\c{c}}}1
}

\lstset{
	framextopmargin=10pt,
	framexrightmargin=10pt,
	framexbottommargin=10pt,
	framexleftmargin=10pt,
	xleftmargin=10pt,
	xrightmargin=10pt,
}

% Couleurs :

\definecolor{title}{HTML}{912c21}
\definecolor{section}{HTML}{1c567d}
\definecolor{subsection}{HTML}{2980b9}

\definecolor{rule}{HTML}{c4c4c4}

\definecolor{formula}{HTML}{ebf3fb}
\definecolor{formula-left}{HTML}{3583d6}

\definecolor{tip}{HTML}{dcf3d8}
\definecolor{tip-left}{HTML}{26a65b}

\definecolor{demonstration}{HTML}{fff8de}
\definecolor{demonstration-left}{HTML}{f1c40f}

\definecolor{exercise}{HTML}{e0f2f1}
\definecolor{exercise-left}{HTML}{009688}

\definecolor{correction}{HTML}{e0f7fa}
\definecolor{correction-left}{HTML}{00bcd4}

\definecolor{toc}{HTML}{fceae9}
\definecolor{toc-left}{HTML}{e74c3c}
\definecolor{toc-dark}{HTML}{87281f}

% Titres :

\renewcommand{\thesection}{\Roman{section} - }
\renewcommand{\thesubsection}{\arabic{subsection}. }

\newcommand{\sectionstyle}{\normalfont\LARGE\bfseries\color{section}}
\titleformat{\section}{\sectionstyle}{\thesection #1}{0pt}{}
\titleformat{name=\section, numberless}{\sectionstyle}{#1}{0pt}{}

\newcommand{\subsectionstyle}{\normalfont\Large\bfseries\color{subsection}}
\titleformat{\subsection}{\subsectionstyle}{\thesubsection #1}{0pt}{}
\titleformat{name=\subsection, numberless}{\subsectionstyle}{#1}{0pt}{}

\titlelabel{\thetitle\ }

% Table des matières :

\addto\captionsfrench{\renewcommand\contentsname{}}
\renewcommand{\cftsecpagefont}{\color{toc-dark}}
\renewcommand{\cftsubsecpagefont}{\color{toc-dark}}
\renewcommand{\cftsecleader}{\cftdotfill{\cftdotsep}}
\renewcommand{\cftsecfont}{\bfseries}
\renewcommand{\cftsecpagefont}{\bfseries\color{toc-dark}}
\setlength{\cftbeforetoctitleskip}{0pt}
\setlength{\cftaftertoctitleskip}{0pt}
\setlength{\cftsecindent}{0pt}
\setlength{\cftsubsecindent}{20pt}
\setlength{\cftsubsecnumwidth}{20pt}

% Commandes :

\newcommand{\newpar}{\\[\medskipamount]}
\newcommand{\lesson}[3]{%
	\newcommand{\level}{#1}%
	\newcommand{\id}{#2}%
	\hypersetup{pdftitle={#3}}
	\begin{center}%
		\includegraphics[width=150px]{\imagespath/bacomathiques}%
		
		\vspace{30pt}%
		{\Huge\color{title} #3}%
		
		\vspace{10pt}%
		{Bacomathiques --- \href{https://bacomathiqu.es/cours/#1/#2}{\color{section} https://bacomathiqu.es}}%
		
		\vspace{20pt}%
	\end{center}%
	\begin{toc}
		\tableofcontents%
	\end{toc}
	\thispagestyle{empty}%
	\newpage%
	\setcounter{page}{1}%
}
\newcommand{\imagespath}{../../images}
\newcommand{\lessonimagespath}{\imagespath/lessons/\level/\id/}
\newcommand{\includelatexpicture}[2][\textwidth - 100pt]{%
	\begin{center}%
		\resizebox{#1}{!}{%
			\input{\lessonimagespath#2}%
		}%
	\end{center}%
	\medskip%
}
\newcommand{\includeimage}[1]{%
	\begin{center}%
		\includegraphics{\lessonimagespath#1}%
	\end{center}%
	\medskip%
}
\newcommand{\includerepresentation}[1]{%
	\begin{center}%
		\setlength{\fboxrule}{0.5pt}%
		\href{https://www.geogebra.org/m/#1}{\includegraphics[width=\textwidth-1pt,fbox]{\lessonimagespath#1}}%
	\end{center}%
}
\newcommand{\floor}[1]{\lfloor #1 \rfloor}

\makeatletter
\newcommand\inputcontent{\@ifstar{\inputcontent@star}{\inputcontent@nostar}}
\newcommand{\inputcontent@star}[1]{%
	\ExecuteMetaData[#1]{content}%
}
\newcommand{\inputcontent@nostar}[1]{%
	\newpage%
	\inputcontent@star{#1}%
}
\makeatother

\let\oldsection\section
\renewcommand\section{\clearpage\oldsection}
\newcommand{\contentwidth}[1][medium]{}

% En-têtes :

\pagestyle{fancy}

\renewcommand{\sectionmark}[1]{\markboth{\thesection \ #1}{}}

\fancyhead[R]{\truncate{0.23\textwidth}{\color{title}\thepage}}
\fancyhead[L]{\truncate{0.73\textwidth}{\color{title}\leftmark}}
\fancyfoot[C]{\scriptsize \href{https://bacomathiqu.es/cours/\level/\id}{\texttt{bacomathiqu.es}}}

\makeatletter
\patchcmd{\f@nch@head}{\rlap}{\color{rule}\rlap}{}{}
\patchcmd{\headrule}{\hrule}{\color{rule}\hrule}{}{}
\makeatother

% Environnements :

\newenvironment{nosummary}{}{}
\newcommand{\tcolorboxtitle}[2]{\setlength{\fboxsep}{2.5pt}\hspace{-10pt}\colorbox{#1-left}{\hspace{8pt}\MakeUppercase{#2} \hspace{2pt} \includegraphics[height=0.8em]{\imagespath/bubbles/#1}\hspace{5pt}}}
\newcommand{\tcolorboxsubtitle}[2]{\ifstrempty{#2}{}{\textcolor{#1-left}{\large#2}\\[\medskipamount]}}
\tcbset{
	frame hidden,
	boxrule=0pt,
	boxsep=0pt,
	enlarge bottom by=8.5pt,
	enhanced jigsaw,
	boxed title style={sharp corners,boxrule=0pt,coltitle={white},titlerule=0pt},
	fonttitle=\fontsize{6pt}{6pt}\bfseries\boldmath,
	top=10pt,
	right=10pt,
	bottom=10pt,
	left=10pt,
	arc=0pt,
	outer arc=0pt,
}
\newtcolorbox{toc}[1][]{
	colback=toc,
	borderline west={3pt}{0pt}{toc-left},
	title=\tcolorboxtitle{toc}{Table des matières},
	colbacktitle=toc,
	before upper={\tcolorboxsubtitle{toc}{#1}}
}
\newtcolorbox{formula}[1][]{
	colback=formula,
	borderline west={3pt}{0pt}{formula-left},
	title=\tcolorboxtitle{formula}{À retenir},
	colbacktitle=formula,
	before upper={\tcolorboxsubtitle{formula}{#1}}
}
\newtcolorbox{tip}[1][]{
	colback=tip,
	borderline west={3pt}{0pt}{tip-left},
	title=\tcolorboxtitle{tip}{À lire},
	colbacktitle=tip,
	before upper={\tcolorboxsubtitle{tip}{#1}}
}
\newtcolorbox{demonstration}[1][]{
	colback=demonstration,
	borderline west={3pt}{0pt}{demonstration-left},
	title=\tcolorboxtitle{demonstration}{Démonstration},
	colbacktitle=demonstration,
	before upper={\tcolorboxsubtitle{demonstration}{#1}}
}

\NewEnviron{whitetabularx}[1]{%
	\renewcommand{\arraystretch}{2.5}
	\colorbox{white}{%
		\begin{tabularx}{\textwidth}{#1}%
			\BODY%
		\end{tabularx}%
	}%
}

% Longueurs :

\newlength{\espacetitreliste}
\setlength{\espacetitreliste}{-16pt}
\newcommand{\entretitreetliste}{\vspace{\espacetitreliste}}

\begin{document}
	%<*content>
	\lesson{premiere}{polynomes-second-degre}{Chapitre II – Les polynômes du second degré}

	\section{Fonctions polynômiales du second degré ?}

	\subsection{Définition}

	\begin{formula}[Définition]
		Soit $f$ une fonction. $f$ est une \textbf{fonction polynômiale du second degré} si elle est de la forme $f : x \mapsto ax^2 + bx + c$ avec $a \neq 0$, $b$ et $c$ réels qui sont les \textbf{coefficients} de $f$.
	\end{formula}

	En classe de Première, ces fonctions auront pour ensemble de départ et d'arrivée $\mathbb{R}$ mais il faut savoir qu'il est possible d'en prendre d'autres.

	\subsection{Représentation graphique}

	\begin{formula}[Parabole]
		Soit $f$ une fonction polynômiale du second degré. Alors la courbe représentative de $f$ (notée $\mathcal{C}_f$) est une \textbf{parabole}.
	\end{formula}

	\includerepresentation{drmymnkb}

	\begin{tip}[Parité d'une fonction]
		On voit sur la représentation ci-dessus que la courbe est symétrique par rapport à l'axe des ordonnées : la fonction $f$ représentée est \textbf{paire} (i.e. pour tout $x \in D_f$, $f(-x) = f(x)$).
		\newpar
		Inversement si une fonction $f$ est symétrique par rapport à l'axe des abscisses, elle est dite \textbf{impaire} (i.e. pour tout $x \in D_f$, $-f(x) = f(x)$).
	\end{tip}

	Chaque coefficient d'une fonction du second degré a un rôle dans le tracé de sa parabole.

	\begin{formula}[Rôle des coefficients dans la représentation graphique]
		Soit $f$ de la forme $f(x) = ax^2 + bx +c$ (avec $a \neq 0$, $b$ et $c$ réels). Alors on a :
		\begin{itemize}
			\item $a$ et $b$ contrôlent \textbf{l'allure générale} de la courbe (son orientation, son inclinaison, ...).
			\item $c$ contrôle l'éloignement de la courbe par rapport à \textbf{l'axe des abscisses}.
		\end{itemize}
	\end{formula}

	\begin{tip}
		Rien que le signe de $a$ peut changer toute l'allure de la courbe :
		\begin{itemize}
			\item Si $a < 0$, la fonction est croissante puis décroissante.
			\item Si $a > 0$, la fonction est décroissante puis croissante.
		\end{itemize}
	\end{tip}

	\section{Recherche de racines}

	\subsection{Qu'est-ce qu'une racine ?}

	\begin{formula}[Définition]
		Soient $f$ une fonction polynômiale du second degré et $x_0 \in \mathbb{R}$. On dit que $x_0$ est \textbf{une racine} de $f$ si $f(x_0) = 0$.
	\end{formula}

	\begin{tip}
		Autrement dit, résoudre l'équation $f(x) = 0$ revient à rechercher les racines de $f$. Pour cela il existe beaucoup de méthodes et nous en détaillerons certaines par la suite.
	\end{tip}

	\subsection{Discriminant}

	\begin{formula}[Définition]
		Soit $f$ une fonction polynômiale du second degré de la forme $f(x) = ax^2 + bx + c$ (avec $a \neq 0$, $b$ et $c$ réels). On appelle \textbf{discriminant} de $f$ le réel suivant : $\Delta = b^2 - 4ac$.
	\end{formula}

	\begin{formula}[Propriétés]
		Plusieurs propriétés découlent du signe de $\Delta$ :
		\begin{itemize}
			\item Si $\Delta < 0$ alors $f$ n'admet pas de racine réelle.
			\item Si $\Delta = 0$ alors $f$ admet une unique racine réelle : $x_0 = \displaystyle{\frac{-b}{2a}}$.
			\item Si $\Delta > 0$ alors $f$ admet deux racines réelles : $x_1 = \displaystyle{\frac{-b - \sqrt{\Delta}}{2a}}$ et $x_2 = \displaystyle{\frac{-b + \sqrt{\Delta}}{2a}}$.
		\end{itemize}
	\end{formula}

	\begin{tip}[Exemple]
		Résolvons l'équation $x^2 = 4$ pour $x \in \mathbb{R}$.
		\newpar
		On a $x^2 = 4 \iff x^2 - 4 = 0$. Il s'agit en fait de chercher les racines de la fonction du second degré définie pour tout $x \in \mathbb{R}$ par $f(x) = x^2 - 4$.
		\newline
		On identifie les coefficients : $a = 1$, $b = 0$ et $c = -4$ ; puis on calcule le discriminant $\Delta = b^2 - 4ac = 0 - 4 \times 1 \times -4 = 16$.
		\newpar
		Comme $\Delta > 0$, on a deux racines réelles :
		$\displaystyle{x_1 = \frac{-b - \sqrt{\Delta}}{2a} = -2}$ et $\displaystyle{x_2 = \frac{-b + \sqrt{\Delta}}{2a} = 2}$.
		\newpar
		Donc l'ensemble des solutions est $S = \{-2; 2\}$.
	\end{tip}

	\subsection{Racines évidentes}

	\begin{formula}[Recherche des racines rationnelles]
		Soit $f$ une fonction polynômiale du second degré de la forme $f(x) = ax^2 + bx +c$ (avec $a \neq 0$, $b$ et $c$ réels). On note $D_c$ l'ensemble des diviseurs de $c$ et $D_a$ l'ensemble des diviseurs de $a$. Alors :
		\newpar
		Pour trouver une éventuelle racine rationnelle de $f$, on calcule $\displaystyle{f\left(\frac{p}{q}\right)}$ pour tout $p \in D_c$ et $q \in D_a$, jusqu'à tomber sur $0$.
	\end{formula}

	\begin{tip}[Exemple]
		Utilisons cette méthode pour déterminer les éventuelles racines rationnelles de la fonction $f$ définie sur $\mathbb{R}$ par $f(x) = 4x^2 - 1$.
		\newpar
		On a ici $a = 4$, $b = 0$ et $c = -1$ ; la liste des diviseurs de $c$ est : $-1$ et $1$.
		\newline
		La liste des diviseurs de $a$ est : $4$, $2$, $1$, $-1$, $-2$ et $-4$.
		Il ne reste qu'à tester :
		\newpar
		$\displaystyle{f\left(\frac{-1}{4}\right)=f\left(\frac{1}{-4}\right) \neq 0}$
		\newline
		$\displaystyle{f\left(\frac{-1}{2}\right)=f\left(\frac{1}{-2}\right)=0}$ \textbf{Une racine !}
		\newline
		$\displaystyle{f\left(\frac{-1}{1}\right)=f(-1) \neq 0}$
		\newline
		$\displaystyle{f\left(\frac{-1}{-1}\right)=f(1) \neq 0}$
		\newline
		$\displaystyle{f\left(\frac{-1}{-2}\right)=f\left(\frac{1}{2}\right)=0}$ \textbf{Une racine !}
		\newpar
		On a deux racines rationnelles : $\displaystyle{-\frac{1}{2}}$ et $\displaystyle{\frac{1}{2}}$.
		\newpar
		Pas besoin d'aller plus loin car on a trouvé deux racines et un polynôme du second degré n'admet que deux racines maximum.
		\newpar
		Signalons de plus que l'on aurait pu s'arrêter après avoir trouvé la première racine car $f$ est une fonction paire.
	\end{tip}

	\subsection{Somme et produit de racines}

	\begin{formula}[Relations]
		Soit $f$ une fonction polynômiale du second degré de la forme $f(x) = ax^2 + bx +c$ (avec $a \neq 0$, $b$ et $c$ réels) admettant deux racines réelles $x_1$ et $x_2$. Alors :
		\begin{itemize}
			\item La somme $S = x_1 + x_2$ des racines vaut également $\displaystyle{-\frac{b}{a}}$.
			\item Le produit $P = x_1 \times x_2$ des racines vaut également $\displaystyle{\frac{c}{a}}$.
		\end{itemize}
	\end{formula}

	\begin{tip}[Exemple]
		Il peut être très utile de combiner cette méthode avec celle des racines évidentes !
		Par exemple, cherchons les solutions de l'équation $x^2 + 2x + 1 = 0$.
		\newpar
		Il faut donc chercher les racines de la fonction de degré 2 définie pour tout $x \in \mathbb{R}$ par $f(x) = x^2 + 2x + 1$.
		\newpar
		On a $a = 1$, $b = 2$ et $c = 1$. Avec la méthode des racines évidentes, on trouve une racine $x_1 = -1$.
		\newpar
		Or, on a $\displaystyle{x_1 \times x_2 = \frac{c}{a} \iff x_2 = -1}$. La deuxième racine vaut aussi $-1$.
		\newpar
		On dit que $-1$ est \textbf{racine double}.
	\end{tip}

	\subsection{Forme factorisée}

	\begin{formula}[Définition]
		Soit $f$ une fonction polynômiale du second degré de la forme $f(x) = ax^2 + bx +c$ (avec $a \neq 0$, $b$ et $c$ réels) admettant deux racines réelles $x_1$ et $x_2$. Alors :
		\newpar
		$f$ admet une \textbf{forme factorisée} qui vaut $f(x) = a(x-x_1)(x-x_2)$ pour tout $x \in \mathbb{R}$.
	\end{formula}

	\begin{tip}[Exemple]
		Chercher les racines de la fonction définie pour tout $x \in \mathbb{R}$ par $f(x) = x^2 - 6x + 9$.
		\newpar
		Avec une identité remarquable, on factorise $f$ : $f(x) = (x - 3)^2$.
		\newpar
		Cela correspond à la forme factorisée de $f$ et elle nous permet d'en déduire que $3$ est une racine double de $f$.
	\end{tip}

	Une propriété découle immédiatement de cette méthode :

	\begin{formula}
		Si $c = 0$, alors $\displaystyle{-\frac{b}{a}}$ et $0$ sont racines.
	\end{formula}

	\section{Étude des fonctions polynômiales du second degré}

	\subsection{Signe}

	\begin{formula}[Signe d'une fonction du second degré]
		Soit $f$ une fonction polynômiale du second degré de la forme $f(x) = ax^2 + bx +c$ (avec $a \neq 0$, $b$ et $c$ réels) admettant deux racines réelles $x_1$ et $x_2$. On suppose ici que $x_1 < x_2$, alors :
		\begin{itemize}
			\item Si $a < 0$ : $f(x) < 0$ sur $]-\infty; x_1[ \, \cup \, ]x_2; +\infty[$ et $f(x) > 0$ sur $]x_1; x_2[$.
			\item Si $a > 0$ : $f(x) > 0$ sur $]-\infty; x_1[ \, \cup \, ]x_2; +\infty[$ et $f(x) < 0$ sur $]x_1; x_2[$.
		\end{itemize}
	\end{formula}

	\begin{tip}
		Si $x_1 = x_2$ ou si $f$ n'admet pas de racine, alors $f$ est du signe de $a$.
	\end{tip}

	\subsection{Variations}
	\label{variations}

	\begin{formula}[Forme canonique]
		Soit $f$ une fonction polynômiale du second degré de la forme $f(x) = ax^2 + bx +c$ (avec $a \neq 0$, $b$ et $c$ réels), alors pour tout $x \in \mathbb{R}$, on peut écrire $f$ de la forme :
		\newpar
		$f(x) = a(x - \alpha)^2 + \beta$ avec $\displaystyle{\alpha = -\frac{b}{2a}}$ et $\beta = f(\alpha)$.
	\end{formula}

	Cette forme est appelée \textbf{forme canonique} de $f$ et elle possède de nombreuses propriétés intéressantes.

	\begin{formula}[Sommet de la parabole]
		Soit $S$ le sommet de la parabole $\mathcal{C}_f$. Alors les coordonnées de $S$ sont $(\alpha, \beta)$.
		\newline
		Si $a < 0$, ce sommet est un maximum et si $a > 0$, ce sommet est un minimum.
	\end{formula}

	\begin{tip}
		Cela veut tout simplement dire que :
		\begin{itemize}
			\item Si $a < 0$, le maximum de $f$ est atteint en $\alpha$ et vaut $\beta$ (donc pour tout $x \in \mathbb{R}$, $f(x) \leq \beta$).
			\item Si $a > 0$, le minimum de $f$ est atteint en $\alpha$ et vaut $\beta$ (donc pour tout $x \in \mathbb{R}$, $f(x) \geq \beta$).
		\end{itemize}
	\end{tip}

	Avec les remarques données précédemment, on peut en déduire les variations de la fonction $f$.

	\begin{formula}[Sens de variation]
		\begin{itemize}
			\item Si $a < 0$ : $f$ est strictement croissante sur $]-\infty; \alpha]$ et est strictement décroissante sur $]\alpha; +\infty]$.
			\item Si $a > 0$ : $f$ est strictement décroissante sur $]-\infty; \alpha]$ et est strictement croissante sur $]\alpha; +\infty]$.
		\end{itemize}
	\end{formula}

	\subsection{Axe de symétrie}

	\begin{formula}[Axe de symétrie]
		Soit $f$ une fonction polynômiale du second degré de la forme $f(x) = ax^2 + bx +c$ (avec $a \neq 0$, $b$ et $c$ réels). On note $\mathcal{C}_f$ sa courbe représentative. Alors :
		\newpar
		$\mathcal{C}_f$ possède un axe de symétrie : la droite $\mathcal{D}$ d'équation $\displaystyle{x = -\frac{b}{2a}}$.
	\end{formula}

	\begin{tip}
		En fait, $\mathcal{D}$ est juste la droite verticale passant par le \hyperref[variations]{sommet} de la parabole.
	\end{tip}
  %</content>
\end{document}
