\documentclass[12pt, a4paper]{report}

% Packages :

\usepackage[french]{babel}
%\usepackage[utf8]{inputenc}
%\usepackage[T1]{fontenc}
\usepackage[pdfencoding=auto, pdfauthor={Bacomathiques}]{hyperref}
\usepackage{sectsty}
\usepackage[explicit]{titlesec}
\usepackage{xcolor}
%\usepackage{amsmath}
%\usepackage{amssymb}
\usepackage{amsthm}
\usepackage{fourier-otf}
\usepackage{titlesec}
\usepackage{fancyhdr}
\usepackage{catchfilebetweentags}
\usepackage[french, capitalise, noabbrev]{cleveref}
\usepackage[fit, breakall]{truncate}
\usepackage[margin=3cm]{geometry}
\usepackage{tocloft}
\usepackage{tikz}
\usepackage{tocloft}
\usepackage{microtype}
\usepackage{listings}
\usepackage{tabularx}
\usepackage{calc}
\usepackage[export]{adjustbox}
\usepackage[most]{tcolorbox}
\usepackage{standalone}
\usepackage{xlop}
\usepackage{etoolbox}
\usepackage{environ}

\usetikzlibrary{arrows.meta}
\usetikzlibrary{trees}

% Police :

\setmathfont{Erewhon Math}

% Paramètres :

\author{Bacomathiques}
\definecolor{graphe}{HTML}{93c9ff}
\setcounter{MaxMatrixCols}{12}
\setlength{\parindent}{0pt}
\setlength{\fboxsep}{0pt}
%\pdfsuppresswarningpagegroup=1

% Code :

\lstdefinestyle{style}{
	backgroundcolor=\color{white},
	commentstyle=\em\color[HTML]{999988},
	keywordstyle=\bfseries,
	identifierstyle=\normalfont,
	stringstyle=\color[rgb]{0.87, 0.07, 0.27},
	basicstyle=\ttfamily\color{black},
	breakatwhitespace=false,
	breaklines=true,
	captionpos=b,
	keepspaces=true,
	numbers=left,
	numbersep=5pt,
	showspaces=false,
	showstringspaces=false,
	showtabs=false,
	tabsize=2,
	numbers=none
}

\lstset{style=style}
\lstset{
	literate=
	{á}{{\'a}}1
	{à}{{\`a}}1
	{ã}{{\~a}}1
	{é}{{\'e}}1
	{ê}{{\^e}}1
	{í}{{\'i}}1
	{ó}{{\'o}}1
	{õ}{{\~o}}1
	{ú}{{\'u}}1
	{ü}{{\"u}}1
	{ç}{{\c{c}}}1
}

\lstset{
	framextopmargin=10pt,
	framexrightmargin=10pt,
	framexbottommargin=10pt,
	framexleftmargin=10pt,
	xleftmargin=10pt,
	xrightmargin=10pt,
}

% Couleurs :

\definecolor{title}{HTML}{912c21}
\definecolor{section}{HTML}{1c567d}
\definecolor{subsection}{HTML}{2980b9}

\definecolor{rule}{HTML}{c4c4c4}

\definecolor{formula}{HTML}{ebf3fb}
\definecolor{formula-left}{HTML}{3583d6}

\definecolor{tip}{HTML}{dcf3d8}
\definecolor{tip-left}{HTML}{26a65b}

\definecolor{demonstration}{HTML}{fff8de}
\definecolor{demonstration-left}{HTML}{f1c40f}

\definecolor{exercise}{HTML}{e0f2f1}
\definecolor{exercise-left}{HTML}{009688}

\definecolor{correction}{HTML}{e0f7fa}
\definecolor{correction-left}{HTML}{00bcd4}

\definecolor{toc}{HTML}{fceae9}
\definecolor{toc-left}{HTML}{e74c3c}
\definecolor{toc-dark}{HTML}{87281f}

% Titres :

\renewcommand{\thesection}{\Roman{section} - }
\renewcommand{\thesubsection}{\arabic{subsection}. }

\newcommand{\sectionstyle}{\normalfont\LARGE\bfseries\color{section}}
\titleformat{\section}{\sectionstyle}{\thesection #1}{0pt}{}
\titleformat{name=\section, numberless}{\sectionstyle}{#1}{0pt}{}

\newcommand{\subsectionstyle}{\normalfont\Large\bfseries\color{subsection}}
\titleformat{\subsection}{\subsectionstyle}{\thesubsection #1}{0pt}{}
\titleformat{name=\subsection, numberless}{\subsectionstyle}{#1}{0pt}{}

\titlelabel{\thetitle\ }

% Table des matières :

\addto\captionsfrench{\renewcommand\contentsname{}}
\renewcommand{\cftsecpagefont}{\color{toc-dark}}
\renewcommand{\cftsubsecpagefont}{\color{toc-dark}}
\renewcommand{\cftsecleader}{\cftdotfill{\cftdotsep}}
\renewcommand{\cftsecfont}{\bfseries}
\renewcommand{\cftsecpagefont}{\bfseries\color{toc-dark}}
\setlength{\cftbeforetoctitleskip}{0pt}
\setlength{\cftaftertoctitleskip}{0pt}
\setlength{\cftsecindent}{0pt}
\setlength{\cftsubsecindent}{20pt}
\setlength{\cftsubsecnumwidth}{20pt}

% Commandes :

\newcommand{\newpar}{\\[\medskipamount]}
\newcommand{\lesson}[3]{%
	\newcommand{\level}{#1}%
	\newcommand{\id}{#2}%
	\hypersetup{pdftitle={#3}}
	\begin{center}%
		\includegraphics[width=150px]{\imagespath/bacomathiques}%
		
		\vspace{30pt}%
		{\Huge\color{title} #3}%
		
		\vspace{10pt}%
		{Bacomathiques --- \href{https://bacomathiqu.es/cours/#1/#2}{\color{section} https://bacomathiqu.es}}%
		
		\vspace{20pt}%
	\end{center}%
	\begin{toc}
		\tableofcontents%
	\end{toc}
	\thispagestyle{empty}%
	\newpage%
	\setcounter{page}{1}%
}
\newcommand{\imagespath}{../../images}
\newcommand{\lessonimagespath}{\imagespath/lessons/\level/\id/}
\newcommand{\includelatexpicture}[2][\textwidth - 100pt]{%
	\begin{center}%
		\resizebox{#1}{!}{%
			\input{\lessonimagespath#2}%
		}%
	\end{center}%
	\medskip%
}
\newcommand{\includeimage}[1]{%
	\begin{center}%
		\includegraphics{\lessonimagespath#1}%
	\end{center}%
	\medskip%
}
\newcommand{\includerepresentation}[1]{%
	\begin{center}%
		\setlength{\fboxrule}{0.5pt}%
		\href{https://www.geogebra.org/m/#1}{\includegraphics[width=\textwidth-1pt,fbox]{\lessonimagespath#1}}%
	\end{center}%
}
\newcommand{\floor}[1]{\lfloor #1 \rfloor}

\makeatletter
\newcommand\inputcontent{\@ifstar{\inputcontent@star}{\inputcontent@nostar}}
\newcommand{\inputcontent@star}[1]{%
	\ExecuteMetaData[#1]{content}%
}
\newcommand{\inputcontent@nostar}[1]{%
	\newpage%
	\inputcontent@star{#1}%
}
\makeatother

\let\oldsection\section
\renewcommand\section{\clearpage\oldsection}
\newcommand{\contentwidth}[1][medium]{}

% En-têtes :

\pagestyle{fancy}

\renewcommand{\sectionmark}[1]{\markboth{\thesection \ #1}{}}

\fancyhead[R]{\truncate{0.23\textwidth}{\color{title}\thepage}}
\fancyhead[L]{\truncate{0.73\textwidth}{\color{title}\leftmark}}
\fancyfoot[C]{\scriptsize \href{https://bacomathiqu.es/cours/\level/\id}{\texttt{bacomathiqu.es}}}

\makeatletter
\patchcmd{\f@nch@head}{\rlap}{\color{rule}\rlap}{}{}
\patchcmd{\headrule}{\hrule}{\color{rule}\hrule}{}{}
\makeatother

% Environnements :

\newenvironment{nosummary}{}{}
\newcommand{\tcolorboxtitle}[2]{\setlength{\fboxsep}{2.5pt}\hspace{-10pt}\colorbox{#1-left}{\hspace{8pt}\MakeUppercase{#2} \hspace{2pt} \includegraphics[height=0.8em]{\imagespath/bubbles/#1}\hspace{5pt}}}
\newcommand{\tcolorboxsubtitle}[2]{\ifstrempty{#2}{}{\textcolor{#1-left}{\large#2}\\[\medskipamount]}}
\tcbset{
	frame hidden,
	boxrule=0pt,
	boxsep=0pt,
	enlarge bottom by=8.5pt,
	enhanced jigsaw,
	boxed title style={sharp corners,boxrule=0pt,coltitle={white},titlerule=0pt},
	fonttitle=\fontsize{6pt}{6pt}\bfseries\boldmath,
	top=10pt,
	right=10pt,
	bottom=10pt,
	left=10pt,
	arc=0pt,
	outer arc=0pt,
}
\newtcolorbox{toc}[1][]{
	colback=toc,
	borderline west={3pt}{0pt}{toc-left},
	title=\tcolorboxtitle{toc}{Table des matières},
	colbacktitle=toc,
	before upper={\tcolorboxsubtitle{toc}{#1}}
}
\newtcolorbox{formula}[1][]{
	colback=formula,
	borderline west={3pt}{0pt}{formula-left},
	title=\tcolorboxtitle{formula}{À retenir},
	colbacktitle=formula,
	before upper={\tcolorboxsubtitle{formula}{#1}}
}
\newtcolorbox{tip}[1][]{
	colback=tip,
	borderline west={3pt}{0pt}{tip-left},
	title=\tcolorboxtitle{tip}{À lire},
	colbacktitle=tip,
	before upper={\tcolorboxsubtitle{tip}{#1}}
}
\newtcolorbox{demonstration}[1][]{
	colback=demonstration,
	borderline west={3pt}{0pt}{demonstration-left},
	title=\tcolorboxtitle{demonstration}{Démonstration},
	colbacktitle=demonstration,
	before upper={\tcolorboxsubtitle{demonstration}{#1}}
}

\NewEnviron{whitetabularx}[1]{%
	\renewcommand{\arraystretch}{2.5}
	\colorbox{white}{%
		\begin{tabularx}{\textwidth}{#1}%
			\BODY%
		\end{tabularx}%
	}%
}

% Longueurs :

\newlength{\espacetitreliste}
\setlength{\espacetitreliste}{-16pt}
\newcommand{\entretitreetliste}{\vspace{\espacetitreliste}}

\begin{document}
	%<*content>
	\lesson{premiere}{4}{fonction-exponentielle}{La fonction exponentielle}
	
	\header{caption}{Les exponentielles sont notamment utilisées dans l'étude des croissances.}

	\header{excerpt}{La fonction exponentielle est la fonction qui est sa propre dérivée et qui
		prend la valeur 1 en 0. Elle est utilisée pour modéliser des phénomènes dans lesquels
		une différence constante sur la variable conduit à un rapport constant sur les images.<br>Dans
		ce chapitre, on va étudier les propriétés de cette fonction (relations algébriques,
		dérivées, variations, signe, ...).}
	
	\header{difficulty}{3}
	
	\section{Le nombre \texorpdfstring{$e$}{e}}
	
	Le \textbf{nombre d'Euler} $e$ (également appelé constante de Neper) est une constante mathématique irrationnelle qui possède de nombreuses propriétés.
	
	\begin{formula}[Valeur approchée]
		Une valeur approchée de $e$ est $\approx 2,71828$.
	\end{formula}
	
	\begin{nosummary}
		Cependant, une définition plus exacte de $e$ existe.
		
		\begin{formula}[Autre définition]
			On définit la suite $(e_n)$ pour tout $n \in \mathbb{N}$ par $e_n = \left(1 + \frac{1}{n}\right)^n$.
			Alors la limite de la suite $(e_n)$ quand $n$ tend vers $+\infty$ est $e$.
		\end{formula}
		
		\begin{tip}
			Grâce à cette définition, il est plus facile de construire un algorithme pour approximer $e$.
		\end{tip}
	\end{nosummary}
	
	\section{La fonction exponentielle}
	
	\subsection{Définition}
	\label{definition}
	
	\begin{formula}[Définition]
		La fonction exponentielle notée pour tout $x \in \mathbb{R}$ par $e^x$ (ou parfois $\exp(x)$) est l'unique fonction $f$ définie sur $\mathbb{R}$ remplissant les critères suivants :
		\begin{itemize}
			\item $f$ est dérivable sur $\mathbb{R}$ et $f'$ = $f$
			\item $f > 0$ sur $\mathbb{R}$
			\item $f(0) = 1$
		\end{itemize}
	\end{formula}
	
	\begin{demonstration}[Existence]
		\textbf{L'existence} de cette fonction est admise, il faut cependant en démontrer \textbf{l'unicité}.
		\newpar
		Soit une autre fonction $g$ vérifiant les mêmes propriétés que notre fonction $f$. On pose pour tout $x \in \mathbb{R}$, $h(x) = \frac{f(x)}{g(x)}$.
		\newpar
		Comme $g$ ne s'annule pas et que $h$ est un quotient de fractions dérivables ne s'annulant pas sur $\mathbb{R}$, $h$ est dérivable sur $\mathbb{R}$.
		\newpar
		D'où, pour tout $x \in \mathbb{R}$, $h'(x) = \frac{f'(x)g(x) - f(x)g'(x)}{g(x)^2} = 0$ (car $f = f'$ et $g = g'$).
		\newpar
		On a donc $h$ constante sur $\mathbb{R}$ et la valeur de $h$ est $h(0) = \frac{f(0)}{g(0)} = 1$.
		\newpar
		Pour tout $x \in \mathbb{R}$, $h(x) = 1 \iff \frac{f(x)}{g(x)} = 1 \iff f(x) = g(x)$. Donc $g = f$.
	\end{demonstration}
	
	\begin{tip}[Formules]
		La fonction exponentielle, telle qu'on l'a écrite, est composée d'un réel ($e \approx 2,718 $) et d'un exposant réel $x$. \textbf{Les opérations sur les exposants} sont disponibles, par exemple, pour tout $x$, $y \in \mathbb{R}$ :
		\begin{itemize}
			\item $e^{x+y} = e^x \times e^y$
			\item $e^{x-y} = \frac{e^x}{e^y}$
			\item $e^{-x} = \frac{1}{e^x}$
			\item $(e^x)^y = e^{x \times y}$
		\end{itemize}
		Et bien entendu, $e^0 = 1$.
	\end{tip}
	
	\subsection{Relations algébriques}
	
	\begin{formula}[Relations algébriques]
		La fonction exponentielle a plusieurs propriétés algébriques qu'il faut connaître. Ainsi, pour tous réels $x$ et $y$ :
		\begin{itemize}
			\item $e^x = e^y \iff x = y$
			\item $e^x < e^y \iff x < y$
		\end{itemize}
	\end{formula}
	
	\subsection{Représentation graphique}
	
	Voici une représentation graphique de la fonction exponentielle (courbe bleue) et de sa tangente au point d'abscisse $0$ :
	
	\includerepresentation{d62ctre4}
	
	On voit plusieurs propriétés données précédemment : $e^0 = 1$, $e \approx 2,718$, etc. Mais également d'autres propriétés que nous verrons par la suite comme le fait que la fonction soit \textbf{strictement positive} sur $\mathbb{R}$. À noter que la \textbf{tangente} à sa courbe représentative en $x = 0$ est $y = x + 1$.
	
	\begin{tip}[Représentation d'une fonction exponentielle]
		Il peut être utile de savoir représenter une courbe d'une fonction du type $x \mapsto e^{kx}$ avec $k \in \mathbb{R}$ :
		\begin{itemize}
			\item L'image de $0$ par ces fonctions est toujours $1$.
			\item Plus $k$ est grand, plus la croissance est forte et rapide.
			\item Si $k$ est négatif, la courbe est symétrique à celle de $x \mapsto e^{-kx}$ par rapport à l'axe des ordonnées.
		\end{itemize}
	\end{tip}
	
	\section{Étude de la fonction}
	
	\subsection{Dérivée}
	
	\begin{formula}[Dérivée d'une composée]
		Soit une fonction $u$ dérivable sur un intervalle $I$, on a pour tout $x$ appartenant à cet intervalle : $(e^{u(x)})' = u'(x)e^{u(x)}$.
	\end{formula}
	
	\begin{formula}[Dérivée]
		Ainsi, si pour tout $x \in I$ on a $u(x) = x$, on retrouve : $({e^x})' = e^x$.
	\end{formula}
	
	Cette propriété a été donnée dans la section \hyperref[definition]{``Définition''}.
	
	\subsection{Variations}
	
	Avec la dérivée donnée précédemment, il est désormais possible d'obtenir les variations de la fonction exponentielle.
	
	\begin{formula}[Variations]
		\includelatexpicture{variations}
		On remarque sur le tableau de variation que la fonction exponentielle est strictement positive et croissante sur $\mathbb{R}$.
	\end{formula}
	
	\subsection{La suite \texorpdfstring{$(e^{na})$}{(exp(na))}}
	
	\begin{formula}
		Soit $a \in \mathbb{R}$. La suite $(e^{na})$ est une suite géométrique de raison $e^a$ et de premier terme $1$.
	\end{formula}
	
	\begin{demonstration}
		Posons pour tout $n \in \mathbb{N}$, $u_n = e^{na}$.
		\newpar
		Calculons $u_{n+1}$ :
		\newpar
		$u_{n+1} = e^{(n+1)a} = e^{na} \times e^a = u_n \times e^a$.
		\newpar
		Et on a bien $u_0 = e^0 = 1$.
	\end{demonstration}
	%</content>
\end{document}