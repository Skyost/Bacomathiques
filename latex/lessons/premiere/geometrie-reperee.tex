\documentclass[12pt, a4paper]{report}

% Packages :

\usepackage[french]{babel}
%\usepackage[utf8]{inputenc}
%\usepackage[T1]{fontenc}
\usepackage[pdfencoding=auto, pdfauthor={Bacomathiques}]{hyperref}
\usepackage{sectsty}
\usepackage[explicit]{titlesec}
\usepackage{xcolor}
%\usepackage{amsmath}
%\usepackage{amssymb}
\usepackage{amsthm}
\usepackage{fourier-otf}
\usepackage{titlesec}
\usepackage{fancyhdr}
\usepackage{catchfilebetweentags}
\usepackage[french, capitalise, noabbrev]{cleveref}
\usepackage[fit, breakall]{truncate}
\usepackage[margin=3cm]{geometry}
\usepackage{tocloft}
\usepackage{tikz}
\usepackage{tocloft}
\usepackage{microtype}
\usepackage{listings}
\usepackage{tabularx}
\usepackage{calc}
\usepackage[export]{adjustbox}
\usepackage[most]{tcolorbox}
\usepackage{standalone}
\usepackage{xlop}
\usepackage{etoolbox}
\usepackage{environ}

\usetikzlibrary{arrows.meta}
\usetikzlibrary{trees}

% Police :

\setmathfont{Erewhon Math}

% Paramètres :

\author{Bacomathiques}
\definecolor{graphe}{HTML}{93c9ff}
\setcounter{MaxMatrixCols}{12}
\setlength{\parindent}{0pt}
\setlength{\fboxsep}{0pt}
%\pdfsuppresswarningpagegroup=1

% Code :

\lstdefinestyle{style}{
	backgroundcolor=\color{white},
	commentstyle=\em\color[HTML]{999988},
	keywordstyle=\bfseries,
	identifierstyle=\normalfont,
	stringstyle=\color[rgb]{0.87, 0.07, 0.27},
	basicstyle=\ttfamily\color{black},
	breakatwhitespace=false,
	breaklines=true,
	captionpos=b,
	keepspaces=true,
	numbers=left,
	numbersep=5pt,
	showspaces=false,
	showstringspaces=false,
	showtabs=false,
	tabsize=2,
	numbers=none
}

\lstset{style=style}
\lstset{
	literate=
	{á}{{\'a}}1
	{à}{{\`a}}1
	{ã}{{\~a}}1
	{é}{{\'e}}1
	{ê}{{\^e}}1
	{í}{{\'i}}1
	{ó}{{\'o}}1
	{õ}{{\~o}}1
	{ú}{{\'u}}1
	{ü}{{\"u}}1
	{ç}{{\c{c}}}1
}

\lstset{
	framextopmargin=10pt,
	framexrightmargin=10pt,
	framexbottommargin=10pt,
	framexleftmargin=10pt,
	xleftmargin=10pt,
	xrightmargin=10pt,
}

% Couleurs :

\definecolor{title}{HTML}{912c21}
\definecolor{section}{HTML}{1c567d}
\definecolor{subsection}{HTML}{2980b9}

\definecolor{rule}{HTML}{c4c4c4}

\definecolor{formula}{HTML}{ebf3fb}
\definecolor{formula-left}{HTML}{3583d6}

\definecolor{tip}{HTML}{dcf3d8}
\definecolor{tip-left}{HTML}{26a65b}

\definecolor{demonstration}{HTML}{fff8de}
\definecolor{demonstration-left}{HTML}{f1c40f}

\definecolor{exercise}{HTML}{e0f2f1}
\definecolor{exercise-left}{HTML}{009688}

\definecolor{correction}{HTML}{e0f7fa}
\definecolor{correction-left}{HTML}{00bcd4}

\definecolor{toc}{HTML}{fceae9}
\definecolor{toc-left}{HTML}{e74c3c}
\definecolor{toc-dark}{HTML}{87281f}

% Titres :

\renewcommand{\thesection}{\Roman{section} - }
\renewcommand{\thesubsection}{\arabic{subsection}. }

\newcommand{\sectionstyle}{\normalfont\LARGE\bfseries\color{section}}
\titleformat{\section}{\sectionstyle}{\thesection #1}{0pt}{}
\titleformat{name=\section, numberless}{\sectionstyle}{#1}{0pt}{}

\newcommand{\subsectionstyle}{\normalfont\Large\bfseries\color{subsection}}
\titleformat{\subsection}{\subsectionstyle}{\thesubsection #1}{0pt}{}
\titleformat{name=\subsection, numberless}{\subsectionstyle}{#1}{0pt}{}

\titlelabel{\thetitle\ }

% Table des matières :

\addto\captionsfrench{\renewcommand\contentsname{}}
\renewcommand{\cftsecpagefont}{\color{toc-dark}}
\renewcommand{\cftsubsecpagefont}{\color{toc-dark}}
\renewcommand{\cftsecleader}{\cftdotfill{\cftdotsep}}
\renewcommand{\cftsecfont}{\bfseries}
\renewcommand{\cftsecpagefont}{\bfseries\color{toc-dark}}
\setlength{\cftbeforetoctitleskip}{0pt}
\setlength{\cftaftertoctitleskip}{0pt}
\setlength{\cftsecindent}{0pt}
\setlength{\cftsubsecindent}{20pt}
\setlength{\cftsubsecnumwidth}{20pt}

% Commandes :

\newcommand{\newpar}{\\[\medskipamount]}
\newcommand{\lesson}[3]{%
	\newcommand{\level}{#1}%
	\newcommand{\id}{#2}%
	\hypersetup{pdftitle={#3}}
	\begin{center}%
		\includegraphics[width=150px]{\imagespath/bacomathiques}%
		
		\vspace{30pt}%
		{\Huge\color{title} #3}%
		
		\vspace{10pt}%
		{Bacomathiques --- \href{https://bacomathiqu.es/cours/#1/#2}{\color{section} https://bacomathiqu.es}}%
		
		\vspace{20pt}%
	\end{center}%
	\begin{toc}
		\tableofcontents%
	\end{toc}
	\thispagestyle{empty}%
	\newpage%
	\setcounter{page}{1}%
}
\newcommand{\imagespath}{../../images}
\newcommand{\lessonimagespath}{\imagespath/lessons/\level/\id/}
\newcommand{\includelatexpicture}[2][\textwidth - 100pt]{%
	\begin{center}%
		\resizebox{#1}{!}{%
			\input{\lessonimagespath#2}%
		}%
	\end{center}%
	\medskip%
}
\newcommand{\includeimage}[1]{%
	\begin{center}%
		\includegraphics{\lessonimagespath#1}%
	\end{center}%
	\medskip%
}
\newcommand{\includerepresentation}[1]{%
	\begin{center}%
		\setlength{\fboxrule}{0.5pt}%
		\href{https://www.geogebra.org/m/#1}{\includegraphics[width=\textwidth-1pt,fbox]{\lessonimagespath#1}}%
	\end{center}%
}
\newcommand{\floor}[1]{\lfloor #1 \rfloor}

\makeatletter
\newcommand\inputcontent{\@ifstar{\inputcontent@star}{\inputcontent@nostar}}
\newcommand{\inputcontent@star}[1]{%
	\ExecuteMetaData[#1]{content}%
}
\newcommand{\inputcontent@nostar}[1]{%
	\newpage%
	\inputcontent@star{#1}%
}
\makeatother

\let\oldsection\section
\renewcommand\section{\clearpage\oldsection}
\newcommand{\contentwidth}[1][medium]{}

% En-têtes :

\pagestyle{fancy}

\renewcommand{\sectionmark}[1]{\markboth{\thesection \ #1}{}}

\fancyhead[R]{\truncate{0.23\textwidth}{\color{title}\thepage}}
\fancyhead[L]{\truncate{0.73\textwidth}{\color{title}\leftmark}}
\fancyfoot[C]{\scriptsize \href{https://bacomathiqu.es/cours/\level/\id}{\texttt{bacomathiqu.es}}}

\makeatletter
\patchcmd{\f@nch@head}{\rlap}{\color{rule}\rlap}{}{}
\patchcmd{\headrule}{\hrule}{\color{rule}\hrule}{}{}
\makeatother

% Environnements :

\newenvironment{nosummary}{}{}
\newcommand{\tcolorboxtitle}[2]{\setlength{\fboxsep}{2.5pt}\hspace{-10pt}\colorbox{#1-left}{\hspace{8pt}\MakeUppercase{#2} \hspace{2pt} \includegraphics[height=0.8em]{\imagespath/bubbles/#1}\hspace{5pt}}}
\newcommand{\tcolorboxsubtitle}[2]{\ifstrempty{#2}{}{\textcolor{#1-left}{\large#2}\\[\medskipamount]}}
\tcbset{
	frame hidden,
	boxrule=0pt,
	boxsep=0pt,
	enlarge bottom by=8.5pt,
	enhanced jigsaw,
	boxed title style={sharp corners,boxrule=0pt,coltitle={white},titlerule=0pt},
	fonttitle=\fontsize{6pt}{6pt}\bfseries\boldmath,
	top=10pt,
	right=10pt,
	bottom=10pt,
	left=10pt,
	arc=0pt,
	outer arc=0pt,
}
\newtcolorbox{toc}[1][]{
	colback=toc,
	borderline west={3pt}{0pt}{toc-left},
	title=\tcolorboxtitle{toc}{Table des matières},
	colbacktitle=toc,
	before upper={\tcolorboxsubtitle{toc}{#1}}
}
\newtcolorbox{formula}[1][]{
	colback=formula,
	borderline west={3pt}{0pt}{formula-left},
	title=\tcolorboxtitle{formula}{À retenir},
	colbacktitle=formula,
	before upper={\tcolorboxsubtitle{formula}{#1}}
}
\newtcolorbox{tip}[1][]{
	colback=tip,
	borderline west={3pt}{0pt}{tip-left},
	title=\tcolorboxtitle{tip}{À lire},
	colbacktitle=tip,
	before upper={\tcolorboxsubtitle{tip}{#1}}
}
\newtcolorbox{demonstration}[1][]{
	colback=demonstration,
	borderline west={3pt}{0pt}{demonstration-left},
	title=\tcolorboxtitle{demonstration}{Démonstration},
	colbacktitle=demonstration,
	before upper={\tcolorboxsubtitle{demonstration}{#1}}
}

\NewEnviron{whitetabularx}[1]{%
	\renewcommand{\arraystretch}{2.5}
	\colorbox{white}{%
		\begin{tabularx}{\textwidth}{#1}%
			\BODY%
		\end{tabularx}%
	}%
}

% Longueurs :

\newlength{\espacetitreliste}
\setlength{\espacetitreliste}{-16pt}
\newcommand{\entretitreetliste}{\vspace{\espacetitreliste}}

\begin{document}
	%<*content>
	\lesson{premiere}{6}{geometrie-reperee}{Géométrie repérée}
	
	\header{caption}{Impossible de donner tous les champs d'applications possibles de la géométrie
		tellement ils sont nombreux !}
	
	\header{excerpt}{La géométrie repérée est une notion très intuitive et qui consiste tout simplement
		à pratiquer la géométrie dans un repère.\newline Dans ce cours, nous allons notamment
		voir comment définir et calculer le produit scalaire dans le plan mais nous allons
		également voir la notion d'équation cartésienne de droites, de cercle, de vecteur
		directeur, etc...}
	
	\header{difficulty}{5}

	\section{Le produit scalaire}

	\subsection{Définition}
	\label{definition}

	\begin{formula}[Définition]
		Soient $\overrightarrow{u} = \begin{pmatrix} {x_1} \\ {y_1}\end{pmatrix}$ et $\overrightarrow{v} = \begin{pmatrix} {x_2} \\ {y_2}\end{pmatrix}$ deux vecteurs du plan (c'est-à-dire possédant chacun deux coordonnées).
		\newpar
		Le \textbf{produit scalaire} entre $u$ et $v$, noté $\overrightarrow{u} \cdot \overrightarrow{v}$ est le réel suivant :
		\newpar
		$\overrightarrow{u} \cdot \overrightarrow{v} = x_1 x_2 + y_1 y_2$.
	\end{formula}

	\begin{formula}[Propriétés]
		Soient $\overrightarrow{u}$, $\overrightarrow{v}$ et $\overrightarrow{w}$ des vecteurs du plan et $\lambda \in \mathbb{R}$, on a les propriétés suivantes :
		\begin{itemize}
			\item $\overrightarrow{u} \cdot \overrightarrow{v} = \overrightarrow{v} \cdot \overrightarrow{u}$
			\item $\overrightarrow{u} \cdot (\overrightarrow{v} + \overrightarrow{w}) = \overrightarrow{u} \cdot \overrightarrow{v} + \overrightarrow{u} \cdot \overrightarrow{w}$
			\item $\lambda(\overrightarrow{u} \cdot \overrightarrow{v}) = (\lambda \overrightarrow{u}) \cdot \overrightarrow{v} = \overrightarrow{u} \cdot (\lambda \overrightarrow{v})$
		\end{itemize}
	\end{formula}

	À l'aide du produit scalaire, il est possible de calculer la \textbf{norme} d'un vecteur.

	\begin{formula}[Calcul de la norme]
		Soit $\overrightarrow{u} = \begin{pmatrix} {x} \\ {y} \end{pmatrix}$ un vecteur du plan : sa norme (notée $\Vert \overrightarrow{u} \Vert$) vaut $\Vert \overrightarrow{u} \Vert = \sqrt{\overrightarrow{u} \cdot \overrightarrow{u}} = \sqrt{x^2 + y^2}$.
	\end{formula}

	\begin{tip}[Caractéristiques d'un vecteur]
		On rappelle qu'un vecteur possède 3 caractéristiques :
		\begin{itemize}
			\item Une \textbf{norme} (sa longueur, par exemple si $\overrightarrow{u} = \overrightarrow{AB}$ alors $\Vert \overrightarrow{u} \Vert = AB$)
			\item Un \textbf{sens} (exemple : ``de $A$ vers $B$'' ou ``de haut en bas'')
			\item Une \textbf{direction} (la direction de la droite que porte le vecteur, horizontale ou verticale par exemple)
		\end{itemize}
	\end{tip}

	\subsection{Calcul}

	Il existe plusieurs méthodes pour calculer le produit scalaire en fonction de la situation dans laquelle on se trouve.

	\begin{formula}[Calcul avec un angle]
		Soient $\overrightarrow{u}$, $\overrightarrow{v}$ deux vecteurs du plan et $\theta$ l'angle orienté entre les deux. On a :
		\[ \overrightarrow{u} \cdot \overrightarrow{v} = \Vert \overrightarrow{u} \Vert \times \Vert \overrightarrow{v} \Vert \times \cos(\theta) \]
	\end{formula}

	\begin{formula}[Calcul avec un projeté orthogonal]
		Soient $A$, $B$ et $C$ trois points distincts du plan. On pose $P$ le projeté orthogonal de $C$ sur $(AB)$. Alors :
		\begin{itemize}
			\item Si $P \in [AB)$ alors $\overrightarrow{AB} \cdot \overrightarrow{AC} = AB \times AP$
			\item Si $P \notin [AB)$ alors $\overrightarrow{AB} \cdot \overrightarrow{AC} = - AB \times AP$
		\end{itemize}
	\end{formula}

	Si on ne possède que les normes de nos vecteurs, il est possible d'utiliser la formule de polarisation.

	\begin{formula}[Formule de polarisation]
		Soient $\overrightarrow{u}$ et $\overrightarrow{v}$ deux vecteurs du plan. Alors :
		\[ \overrightarrow{u} \cdot \overrightarrow{v} = \frac{1}{2} \left(\Vert \overrightarrow{u} + \overrightarrow{v} \Vert^2 - \Vert \overrightarrow{u} \Vert^2 - \Vert \overrightarrow{v} \Vert^2\right) \]
	\end{formula}

	\begin{tip}[Utilisation des formules]
		Il faut vraiment trouver la formule à utiliser selon l'énoncé de l'exercice.
		\newpar
		Par exemple, si on se trouve dans un repère et que l'on a les coordonnées des vecteurs, on pourra utiliser la formule de la \hyperref[definition]{définition}. À l'inverse, si on ne possède pas les coordonnées de nos vecteurs mais que l'on possède leur normes, il est possible d'utiliser la formule de polarisation.
		\newpar
		Voici un tableau récapitulatif pour $\overrightarrow{u}$ et $\overrightarrow{v}$ vecteurs du plan :
		\newpar
		\begin{whitetabularx}{|X|X|X|}
			\hline
			\textbf{Données} & \textbf{Formule} & \textbf{À utiliser si on possède...} \\
			\hline
			$\overrightarrow{u} = \begin{pmatrix} {x_1} \\ {y_1} \end{pmatrix}$ \medskip $\overrightarrow{v} = \begin{pmatrix} {x_2} \\ {y_2} \end{pmatrix}$. & $\overrightarrow{u} \cdot \overrightarrow{v} = x_1 \times x_2 + y_1 \times y_2$ \medskip (Calcul à partir des coordonnées.) & Les coordonnées de $\overrightarrow{u}$ et $\overrightarrow{v}$. \\
			\hline
			$\theta$ est l'angle orienté entre $\overrightarrow{u}$ et $\overrightarrow{v}$. & $\overrightarrow{u} \cdot \overrightarrow{v} = \Vert \overrightarrow{u} \Vert \times \Vert \overrightarrow{v} \Vert \times \cos(\theta)$ \medskip (Calcul à partir des normes et d'un angle.) & La norme de $\overrightarrow{u}$, la norme de $\overrightarrow{v}$ et l'angle $\theta$ entre les deux vecteurs. \\
			\hline
			$A$ et $B$ sont les deux extrémités de $\overrightarrow{u}$, $A$ et $C$ sont les deux extrémités de $\overrightarrow{v}$, et $P$ est le projeté orthogonal de $C$ sur $(AB)$. & $\overrightarrow{u} \cdot \overrightarrow{v} = \overrightarrow{AB} \cdot \overrightarrow{AC} = \pm AB \times AP$ \medskip $+$ si $P \in [AB)$ et $-$ sinon. \medskip (Calcul à partir d'une projection orthogonale.) & 3 points distincts (qui sont ici $A$, $B$ et $C$). \\
			\hline
			& $\overrightarrow{u} \cdot \overrightarrow{v} = \frac{\Vert \overrightarrow{u} + \overrightarrow{v} \Vert^2 - \Vert \overrightarrow{u} \Vert^2 - \Vert \overrightarrow{v} \Vert^2}{2}$ \medskip (Calcul à partir des normes.) & On possède la norme de $\overrightarrow{u}$, celle de $\overrightarrow{v}$ mais surtout celle de $\overrightarrow{u} + \overrightarrow{v}$. \\
			\hline
		\end{whitetabularx}
	\end{tip}

	\subsection{Théorème d'Al-Kashi}

	Le \textbf{théorème d'Al-Kashi} permet de calculer la longueur des côtés de n'importe quel triangle, qu'il soit rectangle ou non. Ainsi,

	\begin{formula}[Théorème d'Al-Kashi]
		Soient $A$, $B$ et $C$ trois points du plan non alignés (formant donc un triangle). On pose $a = BC$, $b = CA$ et $c = AB$. Alors :
		\[ c^2 = a^2 + b^2 - 2 \times a \times b \times \cos(\widehat{ACB}) \]
	\end{formula}

	\begin{demonstration}[Théorème d'Al-Kashi]
		\contentwidth[big]
		En reprenant les notations de l'énoncé :
		\begin{align*}
			c^2 &= \Vert \overrightarrow{AB} \Vert^2 \\
			&= \Vert \overrightarrow{CB} - \overrightarrow{CA} \Vert^2 \text{ (par la relation de Chasles)} \\
			&= \Vert \overrightarrow{CB} \Vert^2 - 2(\overrightarrow{CB} \cdot \overrightarrow{CA}) + \Vert \overrightarrow{CA} \Vert^2 \text{ (par la formule de polarisation)} \\
			&= CB^2 - 2(CB \times CA \times \cos(\widehat{ACB})) + CA^2 \\
			&= a^2 + b^2 - 2 \times a \times b \times \cos(\widehat{ACB})
		\end{align*}
	\end{demonstration}

	\section{Géométrie}

	\subsection{Équation cartésienne d'une droite}

	\begin{formula}[Définition]
		Il est possible de décrire tous les points appartenant à une droite $\mathcal{D}$ par une équation appelée \textbf{équation cartésienne}.
		\newpar
		Une équation cartésienne de $\mathcal{D}$ est de la forme $ax + by + c = 0$ avec $a \neq 0$, $b \neq 0$ et $c$ réels, et où $x$ et $y$ sont des coordonnées de points.
	\end{formula}

	\begin{tip}
		Il est très facile de dire si oui ou non un point appartient à une droite si l'on possède l'équation cartésienne de cette droite.
		\newpar
		Par exemple, on définit la droite $\mathcal{D}$ par l'équation $y = x - 1$.
		\newpar
		Est-ce-que $A = (0; 1)$ appartient à $\mathcal{D}$ ? Remplaçons $x$ et $y$ par les coordonnées de $A$ : $1 = -1$ : c'est faux donc $A$ n'appartient pas à $\mathcal{D}$ car les coordonnées de $A$ ne vérifient par l'équation cartésienne de $\mathcal{D}$.
		\newpar
		Est-ce-que $B = (4; 3)$ appartient à $\mathcal{D}$ ? Remplaçons $x$ et $y$ par les coordonnées de $B$ : $3 = 3$ : c'est vrai donc $B$ appartient à $\mathcal{D}$ car les coordonnées de $B$ vérifient l'équation cartésienne de $\mathcal{D}$.
	\end{tip}

	\subsection{Vecteurs directeurs d'une droite}

	\begin{formula}[Définition]
		Soient $\mathcal{D}$ une droite et $\overrightarrow{u}$ un vecteur du plan non nul. Alors $\overrightarrow{u}$ est un \textbf{vecteur directeur} de $\mathcal{D}$ s'il existe deux points $A$ et $B$ appartenants à $\mathcal{D}$ et tels que $\overrightarrow{u} = \overrightarrow{AB}$.
	\end{formula}

	\includerepresentation{kccrxa9q}
	De plus, on a la propriété suivante qui peut s'avérer très utile :

	\begin{formula}[Colinéarité des vecteurs directeurs]
		$\overrightarrow{v}$ est un vecteur directeur de $\mathcal{D}$ si et seulement s'il est colinéaire au vecteur $\overrightarrow{u}$ précédent.
	\end{formula}

	Tous les vecteurs directeurs d'une droite sont donc colinéaires entre eux.

	\begin{tip}[Exemple]
		Soit $\mathcal{D}$ la droite définie par l'équation $y = 2x + 1$, montrons que $\overrightarrow{v} = \begin{pmatrix} 2 \\ 4\end{pmatrix}$ est un vecteur directeur de $\mathcal{D}$.
		\newpar
		Prenons deux points au hasard situés sur cette droite :
		\newline
		$x = 0$ donne $y = 1$, donc le point $A = (0; 1)$ appartient à $\mathcal{D}$.
		\newline
		$x = 1$ donne $y = 3$, donc le point $B = (1; 3)$ appartient à $\mathcal{D}$.
		\newpar
		Ainsi, un vecteur directeur de $\mathcal{D}$ est $\overrightarrow{u} = \overrightarrow{AB} = \begin{pmatrix} 1-0 \\ 3-1\end{pmatrix} = \begin{pmatrix} 1 \\ 2\end{pmatrix}$.
		\newpar
		Il reste à vérifier que $\overrightarrow{u}$ et $\overrightarrow{v}$ sont bien colinéaires, pour cela on peut utiliser la formule vue en seconde :
		\newline
		$2 \times 2 - 1 \times 4 = 0$ : $\overrightarrow{u}$ et $\overrightarrow{v}$ sont bien colinéaires et donc $\overrightarrow{v}$ est un vecteur directeur de $\mathcal{D}$.
	\end{tip}

	Il est facile de trouver un vecteur directeur d'une droite dont on connaît l'équation cartésienne.

	\begin{formula}[Coordonnées d'un vecteur directeur]
		Soit $\mathcal{D}$ une droite définie par l'équation $ax + by + c = 0$. Alors $\overrightarrow{u} = \begin{pmatrix} -b \\ a\end{pmatrix}$ est un vecteur directeur de $\mathcal{D}$.
	\end{formula}

	\begin{tip}[Exemple]
		Déterminons l'équation cartésienne de la droite $\mathcal{D}$ de vecteur directeur $\overrightarrow{u} = \begin{pmatrix} 1 \\ 2\end{pmatrix}$ et passant par $A = (1; 0)$.
		\newpar
		On a déjà $a$ et $b$ par la propriété précédente :
		\[ -b = 1 \iff b = -1 \text{ et } a = 2 \]
		Une équation cartésienne de la droite est $2x - y + c = 0$. Il reste à trouver $c$. Mais comme $\mathcal{D}$ passe par $A$, les coordonnées de $A$ vérifient l'équation cartésienne de $\mathcal{D}$.
		\newpar
		Remplaçons $x$ et $y$ par les coordonnées de $A$ dans l'équation cartésienne : \[ 2 + c = 0 \iff c = -2 \]
		L'équation cartésienne recherchée est donc $2x - y - 2 = 0$ ou encore $y = 2x - 2$.
	\end{tip}

	\begin{formula}[Propriétés]
		Soient $\mathcal{D}_1$ et $\mathcal{D}_2$ deux droites respectivement de vecteurs directeurs $\overrightarrow{u}$ et $\overrightarrow{v}$. Alors :
		\begin{itemize}
			\item $\mathcal{D}_1$ est parallèle à $\mathcal{D}_2$ si et seulement si $\overrightarrow{u}$ et $\overrightarrow{v}$ sont colinéaires.
			\item $\mathcal{D}_1$ est perpendiculaire à $\mathcal{D}_2$ si et seulement si $\overrightarrow{u} \cdot \overrightarrow{v} = 0$.
		\end{itemize}
	\end{formula}

	\begin{formula}[Orthogonalité]
		Si $\overrightarrow{u} \cdot \overrightarrow{v} = 0$ alors $\overrightarrow{u}$ et $\overrightarrow{v}$ sont dits \textbf{orthogonaux}.
	\end{formula}

	\subsection{Vecteurs normaux à une droite}

	\begin{formula}[Définition]
		Soient $\mathcal{D}$ une droite de vecteur directeur $\overrightarrow{u}$ et $\overrightarrow{n}$ un vecteur du plan non nul. Alors $\overrightarrow{n}$ est un \textbf{vecteur normal} à $\mathcal{D}$ si $\overrightarrow{u}$ et $\overrightarrow{n}$ sont orthogonaux entre-eux.
	\end{formula}

	\includerepresentation{ydshrp8y}

	De plus, on a la propriété suivante qui peut s'avérer très utile :

	\begin{formula}[Colinéarité des vecteurs normaux]
		$\overrightarrow{m}$ est un vecteur normal à $\mathcal{D}$ si et seulement s'il est colinéaire au vecteur $\overrightarrow{n}$ précédent.
	\end{formula}

	Tous les vecteurs normaux d'une droite sont donc colinéaires entre-eux.

	Il est facile de trouver un vecteur normal à une droite dont on connaît l'équation cartésienne.

	\begin{formula}[Coordonnées d'un vecteur normal]
		Soit $\mathcal{D}$ une droite définie par l'équation $ax + by + c = 0$. Alors $\overrightarrow{n} = \begin{pmatrix} a \\ b\end{pmatrix}$ est un vecteur normal à $\mathcal{D}$.
	\end{formula}

	Soient $\mathcal{D}_1$ et $\mathcal{D}_2$ deux droites respectivement de vecteurs directeurs $\overrightarrow{u}$ et $\overrightarrow{v}$. Alors :

	\begin{formula}
		$\mathcal{D}_1$ est perpendiculaire à $\mathcal{D}_2$ si et seulement si $\overrightarrow{u}$ est normal à $\mathcal{D}_2$.
	\end{formula}

	\begin{tip}[Exemple]
		Déterminons l'équation cartésienne de la droite $\mathcal{D}$ admettant pour vecteur normal $\overrightarrow{n} = \begin{pmatrix} -1 \\ -1\end{pmatrix}$ et passant par l'origine $O = (0; 0)$.
		\newpar
		On a déjà $a$ et $b$ par la propriété précédente :
		\newline
		$a = -1$
		\newline
		$b = -1$
		\newpar
		Une équation cartésienne de la droite est $-x - y + c = 0$. Il reste à trouver $c$. Mais comme $\mathcal{D}$ passe par l'origine, les coordonnées de $O$ vérifient l'équation cartésienne de $\mathcal{D}$.
		\newpar
		Remplaçons $x$ et $y$ par les coordonnées de $O$ dans l'équation cartésienne : $c = 0$.
		L'équation cartésienne recherchée est donc $-x - y = 0$ ou encore $y = -x$.
	\end{tip}

	\subsection{Description d'un cercle}

	De la même manière que pour les droites, il est possible de décrire l'ensemble des points appartenant à un cercle à l'aide d'une équation.

	\begin{formula}[Description par équation cartésienne]
		Soit $\mathcal{C}$ un cercle de centre $O = (x_O; y_O)$ et de rayon $R$.
		\newpar
		Une équation cartésienne de $\mathcal{C}$ est de la forme $(x - x_O)^2 + (y - y_O)^2 = R^2$ avec $x$ et $y$ qui sont des coordonnées de points.
	\end{formula}

	On peut de manière équivalente, décrire un cercle à l'aide du produit scalaire.

	\begin{formula}[Description par produit scalaire]
		Soient $A$ et $B$ deux points du plan. Alors l'ensemble des points $M$ tels que $\overrightarrow{MA} \cdot \overrightarrow{MB} = 0$ est le cercle de diamètre $[AB]$.
	\end{formula}

	\begin{demonstration}[Description par produit scalaire]
		\contentwidth[big]
		On pose $A = (x_A; y_A)$, $B = (x_B; y_B)$ et on cherche les points $M = (x; y)$ tels que $\overrightarrow{MA} \cdot \overrightarrow{MB} = 0$.
		\newpar
		Soit $O$ le milieu de $[AB]$ :
		\begin{align*}
			\overrightarrow{MA} \cdot \overrightarrow{MB} = 0 &\iff (\overrightarrow{MO} + \overrightarrow{OA}) \cdot (\overrightarrow{MO} + \overrightarrow{OB}) = 0 \\
			&\iff (\overrightarrow{MO} + \overrightarrow{OA}) \cdot (\overrightarrow{MO} - \overrightarrow{OA}) = 0 \\
			&\iff (\overrightarrow{MO} \cdot \overrightarrow{MO}) - (\overrightarrow{OA} \cdot \overrightarrow{OA}) = 0 \\
			&\iff MO^2 - OA^2 = 0 \\
			&\iff MO = OA
		\end{align*}
		Donc l'ensemble cherché est l'ensemble des points situés à une distance $OA$ du point $O$, c'est bien le cercle de centre $O$ et de diamètre $[AB]$.
	\end{demonstration}

	En réalité, les deux points précédents sont deux manières différentes de décrire un cercle.
	%</content>
\end{document}
