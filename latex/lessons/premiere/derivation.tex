\documentclass[12pt, a4paper]{report}

% Packages :

\usepackage[french]{babel}
%\usepackage[utf8]{inputenc}
%\usepackage[T1]{fontenc}
\usepackage[pdfencoding=auto, pdfauthor={Bacomathiques}]{hyperref}
\usepackage{sectsty}
\usepackage[explicit]{titlesec}
\usepackage{xcolor}
%\usepackage{amsmath}
%\usepackage{amssymb}
\usepackage{amsthm}
\usepackage{fourier-otf}
\usepackage{titlesec}
\usepackage{fancyhdr}
\usepackage{catchfilebetweentags}
\usepackage[french, capitalise, noabbrev]{cleveref}
\usepackage[fit, breakall]{truncate}
\usepackage[margin=3cm]{geometry}
\usepackage{tocloft}
\usepackage{tikz}
\usepackage{tocloft}
\usepackage{microtype}
\usepackage{listings}
\usepackage{tabularx}
\usepackage{calc}
\usepackage[export]{adjustbox}
\usepackage[most]{tcolorbox}
\usepackage{standalone}
\usepackage{xlop}
\usepackage{etoolbox}
\usepackage{environ}

\usetikzlibrary{arrows.meta}
\usetikzlibrary{trees}

% Police :

\setmathfont{Erewhon Math}

% Paramètres :

\author{Bacomathiques}
\definecolor{graphe}{HTML}{93c9ff}
\setcounter{MaxMatrixCols}{12}
\setlength{\parindent}{0pt}
\setlength{\fboxsep}{0pt}
%\pdfsuppresswarningpagegroup=1

% Code :

\lstdefinestyle{style}{
	backgroundcolor=\color{white},
	commentstyle=\em\color[HTML]{999988},
	keywordstyle=\bfseries,
	identifierstyle=\normalfont,
	stringstyle=\color[rgb]{0.87, 0.07, 0.27},
	basicstyle=\ttfamily\color{black},
	breakatwhitespace=false,
	breaklines=true,
	captionpos=b,
	keepspaces=true,
	numbers=left,
	numbersep=5pt,
	showspaces=false,
	showstringspaces=false,
	showtabs=false,
	tabsize=2,
	numbers=none
}

\lstset{style=style}
\lstset{
	literate=
	{á}{{\'a}}1
	{à}{{\`a}}1
	{ã}{{\~a}}1
	{é}{{\'e}}1
	{ê}{{\^e}}1
	{í}{{\'i}}1
	{ó}{{\'o}}1
	{õ}{{\~o}}1
	{ú}{{\'u}}1
	{ü}{{\"u}}1
	{ç}{{\c{c}}}1
}

\lstset{
	framextopmargin=10pt,
	framexrightmargin=10pt,
	framexbottommargin=10pt,
	framexleftmargin=10pt,
	xleftmargin=10pt,
	xrightmargin=10pt,
}

% Couleurs :

\definecolor{title}{HTML}{912c21}
\definecolor{section}{HTML}{1c567d}
\definecolor{subsection}{HTML}{2980b9}

\definecolor{rule}{HTML}{c4c4c4}

\definecolor{formula}{HTML}{ebf3fb}
\definecolor{formula-left}{HTML}{3583d6}

\definecolor{tip}{HTML}{dcf3d8}
\definecolor{tip-left}{HTML}{26a65b}

\definecolor{demonstration}{HTML}{fff8de}
\definecolor{demonstration-left}{HTML}{f1c40f}

\definecolor{exercise}{HTML}{e0f2f1}
\definecolor{exercise-left}{HTML}{009688}

\definecolor{correction}{HTML}{e0f7fa}
\definecolor{correction-left}{HTML}{00bcd4}

\definecolor{toc}{HTML}{fceae9}
\definecolor{toc-left}{HTML}{e74c3c}
\definecolor{toc-dark}{HTML}{87281f}

% Titres :

\renewcommand{\thesection}{\Roman{section} - }
\renewcommand{\thesubsection}{\arabic{subsection}. }

\newcommand{\sectionstyle}{\normalfont\LARGE\bfseries\color{section}}
\titleformat{\section}{\sectionstyle}{\thesection #1}{0pt}{}
\titleformat{name=\section, numberless}{\sectionstyle}{#1}{0pt}{}

\newcommand{\subsectionstyle}{\normalfont\Large\bfseries\color{subsection}}
\titleformat{\subsection}{\subsectionstyle}{\thesubsection #1}{0pt}{}
\titleformat{name=\subsection, numberless}{\subsectionstyle}{#1}{0pt}{}

\titlelabel{\thetitle\ }

% Table des matières :

\addto\captionsfrench{\renewcommand\contentsname{}}
\renewcommand{\cftsecpagefont}{\color{toc-dark}}
\renewcommand{\cftsubsecpagefont}{\color{toc-dark}}
\renewcommand{\cftsecleader}{\cftdotfill{\cftdotsep}}
\renewcommand{\cftsecfont}{\bfseries}
\renewcommand{\cftsecpagefont}{\bfseries\color{toc-dark}}
\setlength{\cftbeforetoctitleskip}{0pt}
\setlength{\cftaftertoctitleskip}{0pt}
\setlength{\cftsecindent}{0pt}
\setlength{\cftsubsecindent}{20pt}
\setlength{\cftsubsecnumwidth}{20pt}

% Commandes :

\newcommand{\newpar}{\\[\medskipamount]}
\newcommand{\lesson}[3]{%
	\newcommand{\level}{#1}%
	\newcommand{\id}{#2}%
	\hypersetup{pdftitle={#3}}
	\begin{center}%
		\includegraphics[width=150px]{\imagespath/bacomathiques}%
		
		\vspace{30pt}%
		{\Huge\color{title} #3}%
		
		\vspace{10pt}%
		{Bacomathiques --- \href{https://bacomathiqu.es/cours/#1/#2}{\color{section} https://bacomathiqu.es}}%
		
		\vspace{20pt}%
	\end{center}%
	\begin{toc}
		\tableofcontents%
	\end{toc}
	\thispagestyle{empty}%
	\newpage%
	\setcounter{page}{1}%
}
\newcommand{\imagespath}{../../images}
\newcommand{\lessonimagespath}{\imagespath/lessons/\level/\id/}
\newcommand{\includelatexpicture}[2][\textwidth - 100pt]{%
	\begin{center}%
		\resizebox{#1}{!}{%
			\input{\lessonimagespath#2}%
		}%
	\end{center}%
	\medskip%
}
\newcommand{\includeimage}[1]{%
	\begin{center}%
		\includegraphics{\lessonimagespath#1}%
	\end{center}%
	\medskip%
}
\newcommand{\includerepresentation}[1]{%
	\begin{center}%
		\setlength{\fboxrule}{0.5pt}%
		\href{https://www.geogebra.org/m/#1}{\includegraphics[width=\textwidth-1pt,fbox]{\lessonimagespath#1}}%
	\end{center}%
}
\newcommand{\floor}[1]{\lfloor #1 \rfloor}

\makeatletter
\newcommand\inputcontent{\@ifstar{\inputcontent@star}{\inputcontent@nostar}}
\newcommand{\inputcontent@star}[1]{%
	\ExecuteMetaData[#1]{content}%
}
\newcommand{\inputcontent@nostar}[1]{%
	\newpage%
	\inputcontent@star{#1}%
}
\makeatother

\let\oldsection\section
\renewcommand\section{\clearpage\oldsection}
\newcommand{\contentwidth}[1][medium]{}

% En-têtes :

\pagestyle{fancy}

\renewcommand{\sectionmark}[1]{\markboth{\thesection \ #1}{}}

\fancyhead[R]{\truncate{0.23\textwidth}{\color{title}\thepage}}
\fancyhead[L]{\truncate{0.73\textwidth}{\color{title}\leftmark}}
\fancyfoot[C]{\scriptsize \href{https://bacomathiqu.es/cours/\level/\id}{\texttt{bacomathiqu.es}}}

\makeatletter
\patchcmd{\f@nch@head}{\rlap}{\color{rule}\rlap}{}{}
\patchcmd{\headrule}{\hrule}{\color{rule}\hrule}{}{}
\makeatother

% Environnements :

\newenvironment{nosummary}{}{}
\newcommand{\tcolorboxtitle}[2]{\setlength{\fboxsep}{2.5pt}\hspace{-10pt}\colorbox{#1-left}{\hspace{8pt}\MakeUppercase{#2} \hspace{2pt} \includegraphics[height=0.8em]{\imagespath/bubbles/#1}\hspace{5pt}}}
\newcommand{\tcolorboxsubtitle}[2]{\ifstrempty{#2}{}{\textcolor{#1-left}{\large#2}\\[\medskipamount]}}
\tcbset{
	frame hidden,
	boxrule=0pt,
	boxsep=0pt,
	enlarge bottom by=8.5pt,
	enhanced jigsaw,
	boxed title style={sharp corners,boxrule=0pt,coltitle={white},titlerule=0pt},
	fonttitle=\fontsize{6pt}{6pt}\bfseries\boldmath,
	top=10pt,
	right=10pt,
	bottom=10pt,
	left=10pt,
	arc=0pt,
	outer arc=0pt,
}
\newtcolorbox{toc}[1][]{
	colback=toc,
	borderline west={3pt}{0pt}{toc-left},
	title=\tcolorboxtitle{toc}{Table des matières},
	colbacktitle=toc,
	before upper={\tcolorboxsubtitle{toc}{#1}}
}
\newtcolorbox{formula}[1][]{
	colback=formula,
	borderline west={3pt}{0pt}{formula-left},
	title=\tcolorboxtitle{formula}{À retenir},
	colbacktitle=formula,
	before upper={\tcolorboxsubtitle{formula}{#1}}
}
\newtcolorbox{tip}[1][]{
	colback=tip,
	borderline west={3pt}{0pt}{tip-left},
	title=\tcolorboxtitle{tip}{À lire},
	colbacktitle=tip,
	before upper={\tcolorboxsubtitle{tip}{#1}}
}
\newtcolorbox{demonstration}[1][]{
	colback=demonstration,
	borderline west={3pt}{0pt}{demonstration-left},
	title=\tcolorboxtitle{demonstration}{Démonstration},
	colbacktitle=demonstration,
	before upper={\tcolorboxsubtitle{demonstration}{#1}}
}

\NewEnviron{whitetabularx}[1]{%
	\renewcommand{\arraystretch}{2.5}
	\colorbox{white}{%
		\begin{tabularx}{\textwidth}{#1}%
			\BODY%
		\end{tabularx}%
	}%
}

% Longueurs :

\newlength{\espacetitreliste}
\setlength{\espacetitreliste}{-16pt}
\newcommand{\entretitreetliste}{\vspace{\espacetitreliste}}

\begin{document}
	%<*content>
	\lesson{premiere}{3}{derivation}{Dérivation}

	\header{caption}{La dérivation de fonctions possède des applications fondamentales en mécanique
		des fluides.}

	\header{excerpt}{En analyse, la dérivation élémentaire est le calcul permettant de définir
		une variation de phénomène par unité de temps ou par unité géométrique. \newpar Au cours
		de ce chapitre nous verrons ce qu'est une dérivée (au sens local comme au sens global)
		puis nous ferons le lien avec les variations d'une fonction. Les différentes tables
		de dérivation sont également au programme de ce cours.}

	\header{difficulty}{2}

	\section{Nombre et fonction dérivés}

	\subsection{Nombre dérivé}

	\begin{formula}[Définition]
		Soient $f$ une fonction définie sur un intervalle $I$ et deux réels $a \in I$ et $h \neq 0$ tels que $(a + h) \in I$.
		\newpar
		La fonction $f$ est \textbf{dérivable} en $a$ si la limite ci-dessous existe et est finie :
		\[ \lim\limits_{h \rightarrow 0} \frac{f(a + h) - f(a)}{h} \]
		Ou en posant $x = a + h$ :
		\[ \lim\limits_{x \rightarrow a} \frac{f(x) - f(a)}{x-a} \]
		Si cette limite existe et est finie, alors elle est égale au \textbf{nombre dérivé} de $f$ en $a$, noté $f'(a)$.
	\end{formula}

	\begin{tip}[Limite d'une fonction]
		La notation $\lim\limits_{h \rightarrow 0}$ veut simplement dire que l'on rend $h$ aussi proche de $0$ que possible (sans pour autant que $h$ soit égal à $0$). On dit que l'on ``fait tendre $h$ vers $0$'' et on appelle cela \textbf{une limite}.
		\newpar
		\textbf{Attention !} Il arrive que cette limite n'existe pas ou ne soit pas finie. Dans ce cas-là, $f'(a)$ n'existe pas et on dit que $f$ n'est pas dérivable en $a$.
	\end{tip}

	\subsection{Tangente en un point}

	\begin{formula}[Équation de la tangente]
		Soient $f$ une fonction définie sur un intervalle $I$ et un réel $a \in I$. Si $f$ est dérivable en $a$, alors la courbe représentative de $f$ admet une tangente $\mathcal{T}$ au point de coordonnées $(a; f(a))$.
		\newpar
		De plus, $f'(a)$ est le coefficient directeur de $\mathcal{T}$, et une équation de $\mathcal{T}$ est $y = f'(a)(x-a)+f(a)$.
	\end{formula}

	\includerepresentation{znryeret}

	\begin{demonstration}[Équation de la tangente]
		La tangente $\mathcal{T}$ en un point d'une courbe est une droite. Une équation de droite est de la forme $y = mx + p$ avec $m$ le coefficient directeur et $p$ l'ordonnée à l'origine.
		\newpar
		On a déjà le coefficient directeur de $\mathcal{T}$ par la propriété précédente : $m = f'(a)$.
		\newpar
		De plus, on sait que $\mathcal{T}$ passe par le point $(a, f(a))$ (car c'est la tangente à $\mathcal{C}_f$ au point d'abscisse $a$).
		\newpar
		Donc l'équation de droite vérifie $f(a) = f'(a)a + p$. Ce qui donne $p = f(a) - af'(a)$.
		\newline
		Au final notre équation est la suivante : $y = xf'(a) + f(a) - af'(a) \iff y = f(a) + (x - a)f'(a)$.
	\end{demonstration}

	\subsection{Fonction dérivée}

	\begin{formula}[Définition]
		Soit $f$ une fonction dérivable sur un intervalle $I$ de $\mathbb{R}$.
		\newpar
		On appelle fonction dérivée (ou plus simplement \textbf{dérivée}) de $f$ la fonction $g$ qui à tout réel $x$ de $I$, associe le nombre dérivé $f'(x)$ (i.e. $g(x) = f'(x)$).
	\end{formula}

	Très souvent, la fonction $g$ sera notée $f'$.

	\section{Tables de dérivation}

	\subsection{Dérivées usuelles}

	Le tableau suivant est à connaître et nous donne la dérivée de la plupart des fonctions usuelles :

	\begin{formula}
		Soient $\lambda$ une constante réelle et $n$ un entier.
		\newpar
		\begin{whitetabularx}{|X|X|l|}
			\hline
			\textbf{Fonction} & \textbf{Dérivée} & \textbf{Domaine de dérivabilité} \\
			\hline
			$x \mapsto \lambda$ & $x \mapsto 0$ & $\mathbb{R}$ \\
			\hline
			$x \mapsto x^n$ & $x \mapsto nx^{n-1}$ & $\mathbb{R}$ \\
			\hline
			$x \mapsto \frac{1}{x}$ & $x \mapsto -\frac{1}{x^2}$ & $\mathbb{R}^*$ \\
			\hline
			$x \mapsto \sqrt{x}$ & $x \mapsto \frac{1}{2\sqrt{x}}$ & $\mathbb{R}^+_*$ \\
			\hline
			$x \mapsto e^x$ & $x \mapsto e^x$ & $\mathbb{R}$ \\
			\hline
			$x \mapsto \sin(x)$ & $x \mapsto \cos(x)$ & $\mathbb{R}$ \\
			\hline
			$x \mapsto \cos(x)$ & $x \mapsto -\sin(x)$ & $\mathbb{R}$ \\
			\hline
		\end{whitetabularx}
	\end{formula}

	\subsection{Opérations sur les dérivées}

	Le tableau suivant est également à connaître et nous donne la dérivée qui dépend des opérations sur certaines fonctions :

	\begin{formula}
		Soient deux fonctions $u$ et $v$ et soit $\lambda$ une constante réelle.
		\newpar
		\begin{whitetabularx}{|X|X|l|}
			\hline
			\textbf{Fonction} & \textbf{Dérivée} & \textbf{Domaine de dérivabilité} \\
			\hline
			$\lambda \times u$ & $\lambda \times u'$ & En tout point où $u$ est dérivable. \\
			\hline
			$u + v$ & $u' + v'$ & En tout point où $u$ et $v$ sont dérivables. \\
			\hline
			$u \times v$ & $u' \times v + u \times v'$ & En tout point où $u$ et $v$ sont dérivables. \\
			\hline
			$\frac{1}{v}$ & $-\frac{v'}{v^2}$ & En tout point où $v$ est dérivable et non nulle. \\
			\hline
			$\frac{u}{v}$ & $\frac{u' \times v - u \times v'}{v^2}$ & En tout point où $u$ et $v$ sont dérivables et non nulles. \\
			\hline
		\end{whitetabularx}
	\end{formula}

	\subsection{Dérivées de composées}

	Le tableau suivant, toujours à connaître, nous donne la dérivée des fonctions composées usuelles (i.e. ``$f$ de $g$ de $x$'') :

	\begin{formula}
		Soit $u$ une fonction.
		\newpar
		\begin{whitetabularx}{|X|X|l|}
			\hline
			\textbf{Fonction} & \textbf{Dérivée} & \textbf{Domaine de dérivabilité} \\
			\hline
			$u^n$ avec $n \in \mathbb{N}^*$ & $nu'u^{n-1}$ & En tout point où $u$ est dérivable. \\
			\hline
			$\frac{1}{u}$ & $-\frac{u'}{u^2}$ & En tout point où $u$ est dérivable et non nulle. \\
			\hline
			$\sqrt{u}$ & $\frac{u'}{2\sqrt{u}}$ & En tout point où $u$ est dérivable et strictement positive. \\
			\hline
			$e^u$ & $u'e^u$ & En tout point où $u$ est dérivable. \\
			\hline
			$\sin(u)$ & $u'\cos(u)$ & En tout point où $u$ est dérivable. \\
			\hline
			$\cos(u)$ & $-u'\sin(u)$ & En tout point où $u$ est dérivable. \\
			\hline
		\end{whitetabularx}
	\end{formula}

	Il est cependant possible de donner une formule plus générale.

	\begin{formula}[Dérivée d'une composée]
		Soient $f$ dérivable sur $I$ et $g$ dérivable sur l'ensemble des valeurs prises par $f$ sur $I$. On a alors $(g \circ f)' = (g' \circ f) \times f'$.
	\end{formula}

	\begin{tip}[Fonction composée]
		On rappelle que la fonction $g \circ f$ est la fonction définie pour tout $x$ par $(g \circ f)(x) = g(f(x))$.
	\end{tip}

	\section{Étude des variations d'une fonction}

	\subsection{Lien dérivée - variations d'une fonction}

	Avec le signe de la dérivée d'une fonction, il est possible d'obtenir le sens de variation de cette fonction.

	\begin{formula}[Variations d'une fonction]
		Soit une fonction $f$ dérivable sur un intervalle $I$.
		\begin{itemize}
			\item Si $f' > 0$ sur $I$, alors $f$ est strictement croissante sur $I$.
			\item Si $f' < 0$ sur $I$, alors $f$ est strictement décroissante sur $I$.
			\item Si $f' = 0$ sur $I$, alors $f$ est constante sur $I$.
		\end{itemize}
	\end{formula}

	\includerepresentation{sjeph3eh}

	\subsection{Extrema}

	\begin{formula}[Étude des extrema]
		Soient $f$ dérivable sur un intervalle $I$, et $a \in I$ :
		\begin{itemize}
			\item Si $f$ admet un extremum local en $a$, alors on a $f'(a) = 0$.
			\item Si $f'(a) = 0$ et que le signe de $f'$ est différent avant et après $a$, alors $f'(a)$ est un extremum local de $f$.
			\item Si $f'(a) = 0$ et qu'on est négatif avant $a$ et positif après, cet extremum local est un minimum local.
			\item Si $f'(a) = 0$ et qu'on est positif avant $a$ et négatif après, cet extremum local est un maximum local.
		\end{itemize}
	\end{formula}

	\begin{tip}
		Avec ceci, il est possible de retrouver la plupart des formules que nous avons vues sur les \href{https://bacomathiqu.es/cours/premiere/polynomes-second-degre/}{fonctions du second degré} (sens de variation, sommet de la parabole, ...).
	\end{tip}
	%</content>
\end{document}
