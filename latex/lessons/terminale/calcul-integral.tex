\documentclass[12pt, a4paper]{report}

% Packages :

\usepackage[french]{babel}
%\usepackage[utf8]{inputenc}
%\usepackage[T1]{fontenc}
\usepackage[pdfencoding=auto, pdfauthor={Bacomathiques}]{hyperref}
\usepackage{sectsty}
\usepackage[explicit]{titlesec}
\usepackage{xcolor}
%\usepackage{amsmath}
%\usepackage{amssymb}
\usepackage{amsthm}
\usepackage{fourier-otf}
\usepackage{titlesec}
\usepackage{fancyhdr}
\usepackage{catchfilebetweentags}
\usepackage[french, capitalise, noabbrev]{cleveref}
\usepackage[fit, breakall]{truncate}
\usepackage[margin=3cm]{geometry}
\usepackage{tocloft}
\usepackage{tikz}
\usepackage{tocloft}
\usepackage{microtype}
\usepackage{listings}
\usepackage{tabularx}
\usepackage{calc}
\usepackage[export]{adjustbox}
\usepackage[most]{tcolorbox}
\usepackage{standalone}
\usepackage{xlop}
\usepackage{etoolbox}
\usepackage{environ}

\usetikzlibrary{arrows.meta}
\usetikzlibrary{trees}

% Police :

\setmathfont{Erewhon Math}

% Paramètres :

\author{Bacomathiques}
\definecolor{graphe}{HTML}{93c9ff}
\setcounter{MaxMatrixCols}{12}
\setlength{\parindent}{0pt}
\setlength{\fboxsep}{0pt}
%\pdfsuppresswarningpagegroup=1

% Code :

\lstdefinestyle{style}{
	backgroundcolor=\color{white},
	commentstyle=\em\color[HTML]{999988},
	keywordstyle=\bfseries,
	identifierstyle=\normalfont,
	stringstyle=\color[rgb]{0.87, 0.07, 0.27},
	basicstyle=\ttfamily\color{black},
	breakatwhitespace=false,
	breaklines=true,
	captionpos=b,
	keepspaces=true,
	numbers=left,
	numbersep=5pt,
	showspaces=false,
	showstringspaces=false,
	showtabs=false,
	tabsize=2,
	numbers=none
}

\lstset{style=style}
\lstset{
	literate=
	{á}{{\'a}}1
	{à}{{\`a}}1
	{ã}{{\~a}}1
	{é}{{\'e}}1
	{ê}{{\^e}}1
	{í}{{\'i}}1
	{ó}{{\'o}}1
	{õ}{{\~o}}1
	{ú}{{\'u}}1
	{ü}{{\"u}}1
	{ç}{{\c{c}}}1
}

\lstset{
	framextopmargin=10pt,
	framexrightmargin=10pt,
	framexbottommargin=10pt,
	framexleftmargin=10pt,
	xleftmargin=10pt,
	xrightmargin=10pt,
}

% Couleurs :

\definecolor{title}{HTML}{912c21}
\definecolor{section}{HTML}{1c567d}
\definecolor{subsection}{HTML}{2980b9}

\definecolor{rule}{HTML}{c4c4c4}

\definecolor{formula}{HTML}{ebf3fb}
\definecolor{formula-left}{HTML}{3583d6}

\definecolor{tip}{HTML}{dcf3d8}
\definecolor{tip-left}{HTML}{26a65b}

\definecolor{demonstration}{HTML}{fff8de}
\definecolor{demonstration-left}{HTML}{f1c40f}

\definecolor{exercise}{HTML}{e0f2f1}
\definecolor{exercise-left}{HTML}{009688}

\definecolor{correction}{HTML}{e0f7fa}
\definecolor{correction-left}{HTML}{00bcd4}

\definecolor{toc}{HTML}{fceae9}
\definecolor{toc-left}{HTML}{e74c3c}
\definecolor{toc-dark}{HTML}{87281f}

% Titres :

\renewcommand{\thesection}{\Roman{section} - }
\renewcommand{\thesubsection}{\arabic{subsection}. }

\newcommand{\sectionstyle}{\normalfont\LARGE\bfseries\color{section}}
\titleformat{\section}{\sectionstyle}{\thesection #1}{0pt}{}
\titleformat{name=\section, numberless}{\sectionstyle}{#1}{0pt}{}

\newcommand{\subsectionstyle}{\normalfont\Large\bfseries\color{subsection}}
\titleformat{\subsection}{\subsectionstyle}{\thesubsection #1}{0pt}{}
\titleformat{name=\subsection, numberless}{\subsectionstyle}{#1}{0pt}{}

\titlelabel{\thetitle\ }

% Table des matières :

\addto\captionsfrench{\renewcommand\contentsname{}}
\renewcommand{\cftsecpagefont}{\color{toc-dark}}
\renewcommand{\cftsubsecpagefont}{\color{toc-dark}}
\renewcommand{\cftsecleader}{\cftdotfill{\cftdotsep}}
\renewcommand{\cftsecfont}{\bfseries}
\renewcommand{\cftsecpagefont}{\bfseries\color{toc-dark}}
\setlength{\cftbeforetoctitleskip}{0pt}
\setlength{\cftaftertoctitleskip}{0pt}
\setlength{\cftsecindent}{0pt}
\setlength{\cftsubsecindent}{20pt}
\setlength{\cftsubsecnumwidth}{20pt}

% Commandes :

\newcommand{\newpar}{\\[\medskipamount]}
\newcommand{\lesson}[3]{%
	\newcommand{\level}{#1}%
	\newcommand{\id}{#2}%
	\hypersetup{pdftitle={#3}}
	\begin{center}%
		\includegraphics[width=150px]{\imagespath/bacomathiques}%
		
		\vspace{30pt}%
		{\Huge\color{title} #3}%
		
		\vspace{10pt}%
		{Bacomathiques --- \href{https://bacomathiqu.es/cours/#1/#2}{\color{section} https://bacomathiqu.es}}%
		
		\vspace{20pt}%
	\end{center}%
	\begin{toc}
		\tableofcontents%
	\end{toc}
	\thispagestyle{empty}%
	\newpage%
	\setcounter{page}{1}%
}
\newcommand{\imagespath}{../../images}
\newcommand{\lessonimagespath}{\imagespath/lessons/\level/\id/}
\newcommand{\includelatexpicture}[2][\textwidth - 100pt]{%
	\begin{center}%
		\resizebox{#1}{!}{%
			\input{\lessonimagespath#2}%
		}%
	\end{center}%
	\medskip%
}
\newcommand{\includeimage}[1]{%
	\begin{center}%
		\includegraphics{\lessonimagespath#1}%
	\end{center}%
	\medskip%
}
\newcommand{\includerepresentation}[1]{%
	\begin{center}%
		\setlength{\fboxrule}{0.5pt}%
		\href{https://www.geogebra.org/m/#1}{\includegraphics[width=\textwidth-1pt,fbox]{\lessonimagespath#1}}%
	\end{center}%
}
\newcommand{\floor}[1]{\lfloor #1 \rfloor}

\makeatletter
\newcommand\inputcontent{\@ifstar{\inputcontent@star}{\inputcontent@nostar}}
\newcommand{\inputcontent@star}[1]{%
	\ExecuteMetaData[#1]{content}%
}
\newcommand{\inputcontent@nostar}[1]{%
	\newpage%
	\inputcontent@star{#1}%
}
\makeatother

\let\oldsection\section
\renewcommand\section{\clearpage\oldsection}
\newcommand{\contentwidth}[1][medium]{}

% En-têtes :

\pagestyle{fancy}

\renewcommand{\sectionmark}[1]{\markboth{\thesection \ #1}{}}

\fancyhead[R]{\truncate{0.23\textwidth}{\color{title}\thepage}}
\fancyhead[L]{\truncate{0.73\textwidth}{\color{title}\leftmark}}
\fancyfoot[C]{\scriptsize \href{https://bacomathiqu.es/cours/\level/\id}{\texttt{bacomathiqu.es}}}

\makeatletter
\patchcmd{\f@nch@head}{\rlap}{\color{rule}\rlap}{}{}
\patchcmd{\headrule}{\hrule}{\color{rule}\hrule}{}{}
\makeatother

% Environnements :

\newenvironment{nosummary}{}{}
\newcommand{\tcolorboxtitle}[2]{\setlength{\fboxsep}{2.5pt}\hspace{-10pt}\colorbox{#1-left}{\hspace{8pt}\MakeUppercase{#2} \hspace{2pt} \includegraphics[height=0.8em]{\imagespath/bubbles/#1}\hspace{5pt}}}
\newcommand{\tcolorboxsubtitle}[2]{\ifstrempty{#2}{}{\textcolor{#1-left}{\large#2}\\[\medskipamount]}}
\tcbset{
	frame hidden,
	boxrule=0pt,
	boxsep=0pt,
	enlarge bottom by=8.5pt,
	enhanced jigsaw,
	boxed title style={sharp corners,boxrule=0pt,coltitle={white},titlerule=0pt},
	fonttitle=\fontsize{6pt}{6pt}\bfseries\boldmath,
	top=10pt,
	right=10pt,
	bottom=10pt,
	left=10pt,
	arc=0pt,
	outer arc=0pt,
}
\newtcolorbox{toc}[1][]{
	colback=toc,
	borderline west={3pt}{0pt}{toc-left},
	title=\tcolorboxtitle{toc}{Table des matières},
	colbacktitle=toc,
	before upper={\tcolorboxsubtitle{toc}{#1}}
}
\newtcolorbox{formula}[1][]{
	colback=formula,
	borderline west={3pt}{0pt}{formula-left},
	title=\tcolorboxtitle{formula}{À retenir},
	colbacktitle=formula,
	before upper={\tcolorboxsubtitle{formula}{#1}}
}
\newtcolorbox{tip}[1][]{
	colback=tip,
	borderline west={3pt}{0pt}{tip-left},
	title=\tcolorboxtitle{tip}{À lire},
	colbacktitle=tip,
	before upper={\tcolorboxsubtitle{tip}{#1}}
}
\newtcolorbox{demonstration}[1][]{
	colback=demonstration,
	borderline west={3pt}{0pt}{demonstration-left},
	title=\tcolorboxtitle{demonstration}{Démonstration},
	colbacktitle=demonstration,
	before upper={\tcolorboxsubtitle{demonstration}{#1}}
}

\NewEnviron{whitetabularx}[1]{%
	\renewcommand{\arraystretch}{2.5}
	\colorbox{white}{%
		\begin{tabularx}{\textwidth}{#1}%
			\BODY%
		\end{tabularx}%
	}%
}

% Longueurs :

\newlength{\espacetitreliste}
\setlength{\espacetitreliste}{-16pt}
\newcommand{\entretitreetliste}{\vspace{\espacetitreliste}}

\begin{document}
	%<*content>
	\lesson{terminale}{calcul-integral}{Chapitre VII – Calcul intégral}
	
	\section{Calcul d'aire}
	
	\subsection{Aires et intégrales}
	
	Dans un repère orthogonal $(O; I; J)$, on prend un point $A = (1; 1)$ et on appelle \textbf{Unité d'Aire} (U.A.) l'aire du rectangle formé par les points $O$, $I$, $A$ et $J$.
	
	\includerepresentation{mxqfqsm5}
	
	Soient $a$ et $b$ deux réels avec $a \leq b$ et $f$ une fonction continue sur $[a;b]$. L'\textbf{intégrale} de la fonction $f$ sur $[a;b]$ notée
	$\displaystyle{\int_{a}^{b} f(x) \, \mathrm{d}x}$ représente l'aire entre la courbe de $f$ et l'axe des abscisses délimitée par les droites d'équation $x = a$ et $x = b$
	et est exprimée en \textbf{U.A.}.
	
	\includerepresentation{txmfnhst}
	
	On dit que les réels $a$ et $b$ sont les \textbf{bornes} de l'intégrale.
	
	\subsection{Théorème fondamental de l'analyse}
	
	Pour calculer l'intégrale d'une fonction, il faut d'abord trouver la primitive de celle-ci (voir le cours sur les \href{https://bacomathiqu.es/cours/terminale/primitives-equations-differentielles/}{primitives}).
	
	\begin{formula}[Théorème fondamental de l'analyse]
		Soient une fonction $f$ continue sur un intervalle $I$ et deux réels $a$ et $b$ appartenant à $I$.
		\newpar
		Alors $\displaystyle{\int_{a}^{b} f(x) \, \mathrm{d}x = \left[ F(x) \right]_a^b = F(b) - F(a)}$ où $F$ est une primitive de $f$ sur $I$.
	\end{formula}
	
	\begin{tip}[Exemple]
		On veut calculer l'aire entre la courbe d'une fonction $f$ définie pour tout $x \in \mathbb{R}$ par $f(x) = 2x + 1$, et l'axe des abscisses sur l'intervalle $[1;4]$ :
		\newpar
		\textbf{1\iere{} étape :} On cherche une primitive de $f$. On trouve $F(x) = x^2 + x = x(x + 1)$.
		\newpar
		\textbf{2\ieme{} étape :} On calcule l'intégrale.
		On a $\displaystyle{\int_{1}^{4} 2x + 1 \, \mathrm{d}x} = \left[ x(x + 1) \right]_1^4 = 4(4 + 1) - 1(1 + 1) = 20 - 2 = 18$ U.A.
		\newpar
	\end{tip}
	
	\begin{tip}[Autre exemple]
		On veut calculer l'aire entre la courbe d'une fonction $f$ définie pour tout $x \in \mathbb{R}$ par $f(x) = x$, et l'axe des abscisses sur l'intervalle $[-2;2]$ :
		\newpar
		\textbf{1\iere{} étape :} On cherche une primitive de $f$. On trouve pour tout $x \in \mathbb{R}$, $F(x) = \frac{x^2}{2}$.
		\newpar
		\textbf{2\ieme{} étape :} On calcule l'intégrale. On a $\displaystyle{\int_{-2}^{2} x \, \mathrm{d}x = \left[ \frac{x^2}{2} \right]_{-2}^2 = \frac{4}{2} - \frac{4}{2} = 0}$ U.A.
		\newpar
		Ce résultat est logique car l'aire au-dessus de la courbe de la fonction $f$ sur $[-2;0]$ est égale à l'aire sous la courbe de $f$ sur $[0;2]$ (voir \hyperref[integrales-paires-impaires]{les propriétés sur les intégrales des fonctions impaires}).
	\end{tip}
	
	\subsection{Signe de l'intégrale}
	
	De manière générale, le signe de l'intégrale d'une fonction sur un intervalle dépend du signe de cette fonction sur cet intervalle.
	
	\begin{formula}[Relation signe de l'intégrale - signe de la fonction]
		Soient une fonction $f$ continue sur un intervalle $I = [a; b]$.
		\begin{itemize}
			\item Si $f > 0$ sur $I$, alors $\displaystyle{\int_{a}^{b} f(x) \, \mathrm{d}x > 0}$.
			\item Si $f < 0$ sur $I$, alors $\displaystyle{\int_{a}^{b} f(x) \, \mathrm{d}x < 0}$.
			\item Si $f$ change de signe sur $I$, on ne connaît pas directement le signe de l'intégrale. Le signe dépend de la partie de l'aire qui est la plus ``grande''.
			\item Soit $g$ une fonction définie sur $I$ avec $f > g$ sur $I$, alors $\displaystyle{\int_{a}^{b} f(x) \, \mathrm{d}x > \int_{a}^{b} g(x) \, \mathrm{d}x}$.
		\end{itemize}
	\end{formula}
	
	Ainsi, cette intégrale sera positive :
	
	\includerepresentation{egjpfkzq}
	
	Et cette intégrale sera négative :
	
	\includerepresentation{zyjkgrkc}
	
	\section{Propriétés de l'intégrale}
	
	\subsection{Propriétés algébriques}
	
	\begin{formula}[Propriétés]
		Soient une fonction $f$ continue sur un intervalle $I$ et deux réels $a$ et $b$ appartenant à $I$.
		\begin{itemize}
			\item $\displaystyle{\int_{a}^{b} f(x) \, \mathrm{d}x = - \int_{b}^{a} f(x) \, \mathrm{d}x}$
			\item $\displaystyle{\int_{a}^{a} f(x) \, \mathrm{d}x = 0}$
		\end{itemize}
	\end{formula}
	
	\subsection{Linéarité}
	
	\begin{formula}[Linéarité de l'intégrale]
		Soient une fonction $f$ continue sur un intervalle $I$ et deux réels $a$ et $b$ appartenant à $I$. Soit $\lambda$ un réel quelconque.
		\begin{itemize}
			\item $\displaystyle{\int_{a}^{b} f(x) + g(x) \, \mathrm{d}x = \int_{b}^{a} f(x) \, \mathrm{d}x + \int_{b}^{a} g(x) \, \mathrm{d}x}$
			\item $\displaystyle{\int_{a}^{b} \lambda f(x) \, \mathrm{d}x = \lambda \int_{b}^{a} f(x) \, \mathrm{d}x}$
		\end{itemize}
	\end{formula}
	
	\subsection{Relation de Chasles}
	
	\begin{formula}[Relation de Chasles]
		Soient une fonction $f$ continue sur un intervalle $I$ et deux réels $a$ et $b$ appartenant à $I$.
		Pour tout $c \in I$, on a $\displaystyle{\int_{a}^{b} f(x) \, \mathrm{d}x = \int_{a}^{c} f(x) \, \mathrm{d}x + \int_{c}^{b} f(x) \, \mathrm{d}x}$.
	\end{formula}
	
	\begin{tip}[Exemple]
		On veut calculer $I = \displaystyle{\int_{-2}^4 f(x) \, \mathrm{d}x}$ où $f(x) = |x| = \begin{cases} -x \text{ si } x < 0 \\ x \text{ si } x \geq 0 \end{cases}$.
		\newpar
		\textbf{1\iere{} étape :} On sépare l'intégrale à l'aide de la relation de Chasles :
		\newpar
		$\displaystyle{I = \int_{-2}^{4} f(x) \, \mathrm{d}x = \int_{-2}^{0} -x \, \mathrm{d}x + \int_{0}^{4} x \, \mathrm{d}x}$.
		\newpar
		\textbf{2\ieme{} étape :} On calcule l'intégrale :
		\newpar
		$\displaystyle{I = \int_{-2}^{0} -x \, \mathrm{d}x + \int_{0}^{4} x \, \mathrm{d}x = \left[ -\frac{x^2}{2} \right]_{-2}^0 + \left[ \frac{x^2}{2} \right]_0^4 = 0 - (-\frac{2^2}{2}) + ((\frac{4^2}{2}) - 0) = 10}$ U.A.
	\end{tip}
	
	\section{Calculs d'intégrale}
	
	\subsection{Intégration par parties}
	
	Il peut arriver que vous ayez à intégrer un produit de fonctions. En classe de Terminale, il est possible de faire appel à une technique appelée \textbf{intégration par parties} pour en venir à bout.
	
	\begin{formula}[Intégration par parties]
		Soient $u$ et $v$ deux fonctions dérivables sur un intervalle $I$ et soient $a$ et $b$ appartenant à $I$.
		\newpar
		Alors $\displaystyle{\int_a^b u'(x) v(x) \, \mathrm{d}x = \left[u(x) v(x)\right]_a^b - \int_a^b u(x) v'(x) \, \mathrm{d}x}$.
	\end{formula}
	
	\begin{demonstration}[Intégration par parties]
		Comme $(u \times v)' = u'v + uv'$, la fonction $u \times v$ est une primitive de la fonction $u'v + uv'$ sur $I$. Or, par la relation de Chasles :
		\newpar
		$\displaystyle{\int_a^b u'(x) v(x) + u(x) v'(x) \, \mathrm{d}x = \int_a^b u'(x) v(x) \, \mathrm{d}x + \int_a^b u(x) v'(x) \, \mathrm{d}x}$
		\newpar
		Donc, avec ce que l'on a fait au tout début, on a bien :
		\newpar
		$\displaystyle{\int_a^b u'(x) v(x) \, \mathrm{d}x + \int_a^b u(x) v'(x) \, \mathrm{d}x = \left[u(x) v(x)\right]_a^b}$
		\newpar
		C'est-à-dire :
		\newpar
		$\displaystyle{\int_a^b u'(x) v(x) \, \mathrm{d}x = \left[u(x) v(x)\right]_a^b - \int_a^b u(x) v'(x) \, \mathrm{d}x}$
	\end{demonstration}
	
	\begin{tip}[Exemple]
		En utilisant cette technique, calculons $\displaystyle{I = \int_{0}^1 xe^x \, \mathrm{d}x}$. Nous souhaitons faire disparaître le ``$x$'', on va donc poser $u'(x) = e^x$ et $v(x) = x$ (afin de dériver $x$).
		\newpar
		Donc par la formule d'intégration par parties :
		\newpar
		$\displaystyle{I = \left[\underbrace{e^x}_{= u} \underbrace{x}_{= v}\right]_0^1 - \int_{0}^1 \underbrace{e^x}_{= u} \times \underbrace{1}_{= v'} \, \mathrm{d}x = e - \left[ e^x \right]_0^1 = 1}$.
		\newpar
		Il vous faudra un peu de pratique pour savoir quelle fonction il faut dériver et quelle fonction il faut primitiver.
	\end{tip}
	
	\subsection{Intégrales de fonctions paires et impaires}
	\label{integrales-paires-impaires}
	
	\begin{formula}[Intégrale d'une fonction paire]
		Soit $f$ une \textbf{fonction paire} continue sur un intervalle $I$ (comme $x \mapsto x^2$).
		\newpar
		On a la relation suivante pour tout $a \in I$ ($-a$ doit aussi être dans $I$) :
		\newpar
		$\displaystyle{\int_{-a}^{a} f(x) \, \mathrm{d}x = 2 \times \int_{0}^{a} f(x) \, \mathrm{d}x = 2 \times \int_{-a}^{0} f(x) \, \mathrm{d}x}$.
	\end{formula}
	
	\begin{tip}[Exemple]
		Cette relation peut se retrouver visuellement, l'aire du côté gauche par rapport à $(Oy)$ est égale à l'aire de l'autre côté de $(Oy)$, et les deux sont positives ; on peut donc les additionner pour retrouver l'aire totale :
		\includerepresentation{txmfnhst}
	\end{tip}
	
	\begin{formula}[Intégrale d'une fonction impaire]
		Soit $f$ une \textbf{fonction impaire} continue sur un intervalle $I$ (comme $x \mapsto x^3$).
		\newpar
		On a la relation suivante pour tout $a \in I$ ($-a$ doit aussi être dans $I$) :
		\newpar
		$\displaystyle{\int_{-a}^{a} f(x) \, \mathrm{d}x = 0}$.
	\end{formula}
	
	\begin{tip}[Exemple]
		De même, on peut retrouver cette relation visuellement, l'aire du côté gauche par rapport à $(Oy)$ est négative et égale à l'aire de l'autre côté de $(Oy)$ qui est positive, les deux s'annulent donc :
		\includerepresentation{jde5vc4m}
	\end{tip}
	
	\subsection{Intégrales de fonctions périodiques}
	
	\begin{formula}[Intégrale d'une fonction périodique]
		Soit $f$ une \textbf{fonction périodique} de période $T$ (comme $\cos$ avec $T = 2\pi$) continue sur chacune de ses périodes, on a la relation suivante pour tout $a \in \mathbb{R}$ :
		\newpar
		$\displaystyle{\int_{0}^{T} f(x) \, \mathrm{d}x = \int_{a}^{a + T} f(x) \, \mathrm{d}x}$
	\end{formula}
	
	\subsection{Valeur moyenne d'une fonction}
	
	\begin{formula}[Valeur moyenne]
		Soient $f$ une fonction continue sur un intervalle $[a;b]$. La valeur moyenne $M$ de $f$ sur $[a;b]$ est donnée par $\displaystyle{M = \frac{1}{b-a}\int_{a}^{b} f(x) \, \mathrm{d}x}$.
	\end{formula}
	
	\subsection{Aire entre deux courbes}
	
	\begin{formula}[Différence d'aires]
		Soient $f$ et $g$ deux fonctions continues sur un intervalle $[a;b]$. Si on a $f \geq g$ sur cet intervalle, alors l'aire entre les deux courbes est donnée par $\displaystyle{\int_{a}^{b} f(x) - g(x) \, \mathrm{d}x}$.
	\end{formula}
	
	\subsection{Primitive s'annulant en \texorpdfstring{$a$}{a}}
	
	\begin{formula}[Existence d'une primitive s'annulant en un point]
		Soient une fonction $f$ continue sur un intervalle $I$ et un réel $a \in I$. La primitive de $f$ sur $I$ qui vaut $0$ en $a$ (notée $F_a$) est donnée par $\displaystyle{F_a : x \mapsto \int_{a}^{x} f(t) \, \mathrm{d}t}$.
	\end{formula}
	
	\begin{demonstration}[Existence d'une primitive]
		Soit $F$ une autre primitive de $f$. Alors on a pour tout $x \in I$, $\displaystyle{F_a(x) = \int_a^x f(t) \, \mathrm{d}t} = F(x) - F(a)$ par le théorème fondamental de l'analyse.
		\newpar
		Donc pour tout $x \in I$, $F_a'(x) = F'(x) - 0 = f(x)$, donc $F_a$ est bien une primitive de $f$.
		\newpar
		De plus, $\displaystyle{F_a(a) = \int_{a}^{a} f(t) \, \mathrm{d}t = 0}$.
		\newpar
		Enfin, comme les primitives d'une fonction continue sur un intervalle diffèrent d'une constante près, on a bien l'unicité de $F_a$.
	\end{demonstration}
	%</content>
\end{document}