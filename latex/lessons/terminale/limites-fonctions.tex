\documentclass[12pt, a4paper]{report}

% Packages :

\usepackage[french]{babel}
%\usepackage[utf8]{inputenc}
%\usepackage[T1]{fontenc}
\usepackage[pdfencoding=auto, pdfauthor={Bacomathiques}]{hyperref}
\usepackage{sectsty}
\usepackage[explicit]{titlesec}
\usepackage{xcolor}
%\usepackage{amsmath}
%\usepackage{amssymb}
\usepackage{amsthm}
\usepackage{fourier-otf}
\usepackage{titlesec}
\usepackage{fancyhdr}
\usepackage{catchfilebetweentags}
\usepackage[french, capitalise, noabbrev]{cleveref}
\usepackage[fit, breakall]{truncate}
\usepackage[margin=3cm]{geometry}
\usepackage{tocloft}
\usepackage{tikz}
\usepackage{tocloft}
\usepackage{microtype}
\usepackage{listings}
\usepackage{tabularx}
\usepackage{calc}
\usepackage[export]{adjustbox}
\usepackage[most]{tcolorbox}
\usepackage{standalone}
\usepackage{xlop}
\usepackage{etoolbox}
\usepackage{environ}

\usetikzlibrary{arrows.meta}
\usetikzlibrary{trees}

% Police :

\setmathfont{Erewhon Math}

% Paramètres :

\author{Bacomathiques}
\definecolor{graphe}{HTML}{93c9ff}
\setcounter{MaxMatrixCols}{12}
\setlength{\parindent}{0pt}
\setlength{\fboxsep}{0pt}
%\pdfsuppresswarningpagegroup=1

% Code :

\lstdefinestyle{style}{
	backgroundcolor=\color{white},
	commentstyle=\em\color[HTML]{999988},
	keywordstyle=\bfseries,
	identifierstyle=\normalfont,
	stringstyle=\color[rgb]{0.87, 0.07, 0.27},
	basicstyle=\ttfamily\color{black},
	breakatwhitespace=false,
	breaklines=true,
	captionpos=b,
	keepspaces=true,
	numbers=left,
	numbersep=5pt,
	showspaces=false,
	showstringspaces=false,
	showtabs=false,
	tabsize=2,
	numbers=none
}

\lstset{style=style}
\lstset{
	literate=
	{á}{{\'a}}1
	{à}{{\`a}}1
	{ã}{{\~a}}1
	{é}{{\'e}}1
	{ê}{{\^e}}1
	{í}{{\'i}}1
	{ó}{{\'o}}1
	{õ}{{\~o}}1
	{ú}{{\'u}}1
	{ü}{{\"u}}1
	{ç}{{\c{c}}}1
}

\lstset{
	framextopmargin=10pt,
	framexrightmargin=10pt,
	framexbottommargin=10pt,
	framexleftmargin=10pt,
	xleftmargin=10pt,
	xrightmargin=10pt,
}

% Couleurs :

\definecolor{title}{HTML}{912c21}
\definecolor{section}{HTML}{1c567d}
\definecolor{subsection}{HTML}{2980b9}

\definecolor{rule}{HTML}{c4c4c4}

\definecolor{formula}{HTML}{ebf3fb}
\definecolor{formula-left}{HTML}{3583d6}

\definecolor{tip}{HTML}{dcf3d8}
\definecolor{tip-left}{HTML}{26a65b}

\definecolor{demonstration}{HTML}{fff8de}
\definecolor{demonstration-left}{HTML}{f1c40f}

\definecolor{exercise}{HTML}{e0f2f1}
\definecolor{exercise-left}{HTML}{009688}

\definecolor{correction}{HTML}{e0f7fa}
\definecolor{correction-left}{HTML}{00bcd4}

\definecolor{toc}{HTML}{fceae9}
\definecolor{toc-left}{HTML}{e74c3c}
\definecolor{toc-dark}{HTML}{87281f}

% Titres :

\renewcommand{\thesection}{\Roman{section} - }
\renewcommand{\thesubsection}{\arabic{subsection}. }

\newcommand{\sectionstyle}{\normalfont\LARGE\bfseries\color{section}}
\titleformat{\section}{\sectionstyle}{\thesection #1}{0pt}{}
\titleformat{name=\section, numberless}{\sectionstyle}{#1}{0pt}{}

\newcommand{\subsectionstyle}{\normalfont\Large\bfseries\color{subsection}}
\titleformat{\subsection}{\subsectionstyle}{\thesubsection #1}{0pt}{}
\titleformat{name=\subsection, numberless}{\subsectionstyle}{#1}{0pt}{}

\titlelabel{\thetitle\ }

% Table des matières :

\addto\captionsfrench{\renewcommand\contentsname{}}
\renewcommand{\cftsecpagefont}{\color{toc-dark}}
\renewcommand{\cftsubsecpagefont}{\color{toc-dark}}
\renewcommand{\cftsecleader}{\cftdotfill{\cftdotsep}}
\renewcommand{\cftsecfont}{\bfseries}
\renewcommand{\cftsecpagefont}{\bfseries\color{toc-dark}}
\setlength{\cftbeforetoctitleskip}{0pt}
\setlength{\cftaftertoctitleskip}{0pt}
\setlength{\cftsecindent}{0pt}
\setlength{\cftsubsecindent}{20pt}
\setlength{\cftsubsecnumwidth}{20pt}

% Commandes :

\newcommand{\newpar}{\\[\medskipamount]}
\newcommand{\lesson}[3]{%
	\newcommand{\level}{#1}%
	\newcommand{\id}{#2}%
	\hypersetup{pdftitle={#3}}
	\begin{center}%
		\includegraphics[width=150px]{\imagespath/bacomathiques}%
		
		\vspace{30pt}%
		{\Huge\color{title} #3}%
		
		\vspace{10pt}%
		{Bacomathiques --- \href{https://bacomathiqu.es/cours/#1/#2}{\color{section} https://bacomathiqu.es}}%
		
		\vspace{20pt}%
	\end{center}%
	\begin{toc}
		\tableofcontents%
	\end{toc}
	\thispagestyle{empty}%
	\newpage%
	\setcounter{page}{1}%
}
\newcommand{\imagespath}{../../images}
\newcommand{\lessonimagespath}{\imagespath/lessons/\level/\id/}
\newcommand{\includelatexpicture}[2][\textwidth - 100pt]{%
	\begin{center}%
		\resizebox{#1}{!}{%
			\input{\lessonimagespath#2}%
		}%
	\end{center}%
	\medskip%
}
\newcommand{\includeimage}[1]{%
	\begin{center}%
		\includegraphics{\lessonimagespath#1}%
	\end{center}%
	\medskip%
}
\newcommand{\includerepresentation}[1]{%
	\begin{center}%
		\setlength{\fboxrule}{0.5pt}%
		\href{https://www.geogebra.org/m/#1}{\includegraphics[width=\textwidth-1pt,fbox]{\lessonimagespath#1}}%
	\end{center}%
}
\newcommand{\floor}[1]{\lfloor #1 \rfloor}

\makeatletter
\newcommand\inputcontent{\@ifstar{\inputcontent@star}{\inputcontent@nostar}}
\newcommand{\inputcontent@star}[1]{%
	\ExecuteMetaData[#1]{content}%
}
\newcommand{\inputcontent@nostar}[1]{%
	\newpage%
	\inputcontent@star{#1}%
}
\makeatother

\let\oldsection\section
\renewcommand\section{\clearpage\oldsection}
\newcommand{\contentwidth}[1][medium]{}

% En-têtes :

\pagestyle{fancy}

\renewcommand{\sectionmark}[1]{\markboth{\thesection \ #1}{}}

\fancyhead[R]{\truncate{0.23\textwidth}{\color{title}\thepage}}
\fancyhead[L]{\truncate{0.73\textwidth}{\color{title}\leftmark}}
\fancyfoot[C]{\scriptsize \href{https://bacomathiqu.es/cours/\level/\id}{\texttt{bacomathiqu.es}}}

\makeatletter
\patchcmd{\f@nch@head}{\rlap}{\color{rule}\rlap}{}{}
\patchcmd{\headrule}{\hrule}{\color{rule}\hrule}{}{}
\makeatother

% Environnements :

\newenvironment{nosummary}{}{}
\newcommand{\tcolorboxtitle}[2]{\setlength{\fboxsep}{2.5pt}\hspace{-10pt}\colorbox{#1-left}{\hspace{8pt}\MakeUppercase{#2} \hspace{2pt} \includegraphics[height=0.8em]{\imagespath/bubbles/#1}\hspace{5pt}}}
\newcommand{\tcolorboxsubtitle}[2]{\ifstrempty{#2}{}{\textcolor{#1-left}{\large#2}\\[\medskipamount]}}
\tcbset{
	frame hidden,
	boxrule=0pt,
	boxsep=0pt,
	enlarge bottom by=8.5pt,
	enhanced jigsaw,
	boxed title style={sharp corners,boxrule=0pt,coltitle={white},titlerule=0pt},
	fonttitle=\fontsize{6pt}{6pt}\bfseries\boldmath,
	top=10pt,
	right=10pt,
	bottom=10pt,
	left=10pt,
	arc=0pt,
	outer arc=0pt,
}
\newtcolorbox{toc}[1][]{
	colback=toc,
	borderline west={3pt}{0pt}{toc-left},
	title=\tcolorboxtitle{toc}{Table des matières},
	colbacktitle=toc,
	before upper={\tcolorboxsubtitle{toc}{#1}}
}
\newtcolorbox{formula}[1][]{
	colback=formula,
	borderline west={3pt}{0pt}{formula-left},
	title=\tcolorboxtitle{formula}{À retenir},
	colbacktitle=formula,
	before upper={\tcolorboxsubtitle{formula}{#1}}
}
\newtcolorbox{tip}[1][]{
	colback=tip,
	borderline west={3pt}{0pt}{tip-left},
	title=\tcolorboxtitle{tip}{À lire},
	colbacktitle=tip,
	before upper={\tcolorboxsubtitle{tip}{#1}}
}
\newtcolorbox{demonstration}[1][]{
	colback=demonstration,
	borderline west={3pt}{0pt}{demonstration-left},
	title=\tcolorboxtitle{demonstration}{Démonstration},
	colbacktitle=demonstration,
	before upper={\tcolorboxsubtitle{demonstration}{#1}}
}

\NewEnviron{whitetabularx}[1]{%
	\renewcommand{\arraystretch}{2.5}
	\colorbox{white}{%
		\begin{tabularx}{\textwidth}{#1}%
			\BODY%
		\end{tabularx}%
	}%
}

% Longueurs :

\newlength{\espacetitreliste}
\setlength{\espacetitreliste}{-16pt}
\newcommand{\entretitreetliste}{\vspace{\espacetitreliste}}

\begin{document}
	%<*content>
	\lesson{terminale}{limites-fonctions}{Chapitre II – Limites de fonctions}

	\section{Limite d'une fonction en un point}

	\subsection{Limite infinie}

	\begin{formula}[Fonction tendant vers $+\infty$ en un point]
		Soit $f$ une fonction (en classe de Terminale, on se limite aux fonctions réelles) d'ensemble de définition $\mathcal{D}_f$. Soit $a$ un réel appartenant à $\mathcal{D}_f$ ou étant une borne de $\mathcal{D}_f$.
		\newpar
		On dit que $f(x)$ \textbf{tend vers $+\infty$} quand $x$ tend vers $a$ si $f(x)$ est aussi grand que l'on veut pourvu que $x$ soit suffisamment proche de $a$.
		\newpar
		On note ceci $\lim_{x \rightarrow a} f(x) = +\infty$.
	\end{formula}

	\begin{tip}[Exemple]
		La fonction $f$ définie sur $\mathbb{R}^*$ par $f(x) = \frac{1}{x^2}$, tend vers $+\infty$ quand $x$ tend vers $0$.
		\includerepresentation{wkysdds8}
	\end{tip}

	Il est tout à fait possible d'établir une définition similaire pour une fonction tendant vers $-\infty$ en un point.

	\begin{tip}[Fonction tendant vers $-\infty$ en un point]
		En reprenant les notations précédentes, on dit que $f(x)$ \textbf{tend vers $-\infty$} quand $x$ tend vers $a$ si $f(x)$ est aussi petit que l'on veut pourvu que $x$ suffisamment proche de $a$.
		\newpar
		On note ceci $\lim_{x \rightarrow a} f(x) = -\infty$.
	\end{tip}

	\begin{tip}[Exemple]
		La fonction $f$ définie sur $]-\infty, 3[ \, \cup \, ]3, +\infty[$ par $f(x) = -\frac{1}{x^2-6x+9}$, tend vers $-\infty$ quand $x$ tend vers $3$.
		\includerepresentation{fu58s6je}
	\end{tip}

	\subsection{Limite finie}

	\begin{formula}[Définition]
		Soit $f$ une fonction d'ensemble de définition $\mathcal{D}_f$. Soit $a$ un réel appartenant à $\mathcal{D}_f$ ou étant une borne de $\mathcal{D}_f$.
		\newpar
		On dit que $f(x)$ \textbf{tend vers $\ell$} quand $x$ tend vers $a$ si $f(x)$ est aussi proche de $\ell$ que l'on veut pourvu que $x$ soit suffisamment proche de $a$.
		\newpar
		On note ceci $\lim_{x \rightarrow a} f(x) = \ell$.
	\end{formula}

	\begin{tip}[Exemple]
		La fonction $f$ définie sur $\mathbb{R}^*$ par $f(x) = \frac{\sin(x)}{x}$, tend vers $1$ quand $x$ tend vers $0$.
		\includerepresentation{fq28tqng}
		Une petite remarque cependant : cette limite n'est pas triviale à démontrer. On peut cependant en proposer une preuve à l'aide de la dérivée de la fonction $\sin$ (qui est $\cos$) : $\lim_{x \rightarrow 0} \frac{\sin(x)}{x} = \lim_{x \rightarrow 0} \frac{\sin(x) - \sin(0)}{x - 0} = \sin'(0) = \cos(0) = 1$.
	\end{tip}

	\subsection{Limites à gauche et à droite}

	\begin{formula}[Définition]
		Soit $f$ une fonction d'ensemble de définition $\mathcal{D}_f$. Soit $a$ un réel appartenant à $\mathcal{D}_f$ ou étant une borne de $\mathcal{D}_f$.
		\begin{itemize}
			\item On dit que $f(x)$ admet une \textbf{limite à gauche} quand $x$ tend vers $a$ si $f(x)$ admet une limite quand $x$ tend vers $a$ avec $x \lt a$. On la note $\lim_{x \rightarrow a^-} f(x)$.
			\item On dit que $f(x)$ admet une \textbf{limite à droite} quand $x$ tend vers $a$ si $f(x)$ admet une limite quand $x$ tend vers $a$ avec $x \gt a$. On la note $\lim_{x \rightarrow a^+} f(x)$.
		\end{itemize}
	\end{formula}

	\begin{tip}[Exemple]
		La fonction $f$ définie sur $\mathbb{R}^*$ par $f(x) = \frac{1}{x}$, admet deux limites différentes à gauche et à droite de $0$ :
		\begin{itemize}
			\item $\lim_{x \rightarrow 0^-} h(x) = -\infty$
			\item $\lim_{x \rightarrow 0^+} h(x) = +\infty$
		\end{itemize}
		\includerepresentation{p5pedmuw}
	\end{tip}

	\subsection{Asymptote verticale}

	\begin{formula}[Définition]
		Soit $f$ une fonction d'ensemble de définition $\mathcal{D}_f$. Soit $a$ un réel appartenant à $\mathcal{D}_f$ ou étant une borne de $\mathcal{D}_f$.
		\newpar
		Alors si $f(x)$ admet une limite infinie quand $x$ tend vers $a$, alors la droite d'équation $x = a$ est une \textbf{asymptote verticale} à la courbe représentative de $f$.
	\end{formula}

	\begin{tip}[Exemple]
		En reprenant les exemples précédents :
		\begin{itemize}
			\item Les courbes représentatives des fonctions $x \mapsto \frac{1}{x}$ et $x \mapsto \frac{1}{x^2}$ admettent toutes deux une asymptote verticale d'équation $x = 0$.
			\item La courbe de la fonction $x \mapsto \frac{1}{x^2-6x+9}$ admet une asymptote verticale d'équation $x = 3$.
		\end{itemize}
	\end{tip}

	\section{Limite d'une fonction en l'infini}

	\subsection{Limite infinie}

	\begin{formula}[Fonction tendant vers $+\infty$ en $+\infty$]
		Soit $f$ une fonction d'ensemble de définition $\mathcal{D}_f$. On suppose qu'une des bornes de $\mathcal{D}_f$ est $+\infty$.
		\newpar
		On dit que $f(x)$ \textbf{tend vers $+\infty$} si $f(x)$ est aussi grand que l'on veut pourvu que $x$ soit suffisamment grand.
	\end{formula}

	Comme précédemment, on peut écrire des définitions similaires pour dire que $f$ tend vers $-\infty$ quand $x$ tend vers $+\infty$.

	\begin{tip}[Fonction tendant vers $-\infty$ en $+\infty$]
		En reprenant les notations précédentes, on dit que $f(x)$ \textbf{tend vers $-\infty$} quand $x$ tend vers $+\infty$ si $f(x)$ est aussi petit que l'on veut pourvu que $x$ soit suffisamment grand.
	\end{tip}

	\begin{tip}[Fonction tendant vers $\pm \infty$ en $-\infty$]
		Pour avoir les définitions quand $x$ tend vers $-\infty$, il suffit de remplacer ``$x$ suffisamment grand'' par ``$x$ suffisamment petit'' et il faut qu'une des bornes de $\mathcal{D}_f$ soit $-\infty$.
	\end{tip}

	\begin{tip}[Exemple]
		La fonction $f$ définie sur $\mathbb{R}$ par $f(x) = 2x+1$, tend vers $+\infty$ quand $x$ tend vers $+\infty$. Cependant, la fonction $-f : x \mapsto -2x - 1$ tend vers $-\infty$ quand $x$ tend vers $+\infty$.
		\includerepresentation{afrnasga}
	\end{tip}

	\subsection{Limite finie}

	\begin{formula}[Limite finie en $+\infty$]
		Soit $f$ une fonction d'ensemble de définition $\mathcal{D}_f$. On suppose qu'une des bornes de $\mathcal{D}_f$ est $+\infty$.
		\newpar
		On dit que $f(x)$ \textbf{tend vers $\ell$} quand $x$ tend vers $+\infty$ si $f(x)$ est aussi proche de $\ell$ que l'on veut pourvu que $x$ soit suffisamment grand.
	\end{formula}

	De même, on peut écrire une définition semblable quand $x$ tend vers $-\infty$.

	\begin{tip}[Limite finie en $-\infty$]
		En reprenant les notations précédentes et en supposant qu'une des bornes de $\mathcal{D}_f$ soit $-\infty$, on dit que $f(x)$ \textbf{tend vers $\ell$} quand $x$ tend vers $-\infty$ si $f(x)$ est aussi proche de $\ell$ que l'on veut pourvu que $x$ soit suffisamment petit.
	\end{tip}

	\begin{tip}[Exemple]
		La fonction $f$ définie sur $\mathbb{R}^+$ par $f(x) = \frac{9x}{3x+1}$ tend vers $3$ quand $x$ tend vers $+\infty$.
		\includerepresentation{rs8mkymv}
	\end{tip}

	\subsection{Asymptote horizontale}

	\begin{formula}[Définition en $+\infty$]
		Soit $f$ une fonction d'ensemble de définition $\mathcal{D}_f$. On suppose qu'une des bornes de $\mathcal{D}_f$ est $+\infty$.
		\newpar
		Alors si $f(x)$ admet une limite finie $\ell$ quand $x$ tend vers $+\infty$, alors la droite d'équation $y = \ell$ est une \textbf{asymptote horizontale} en $+\infty$ à la courbe représentative de $f$.
	\end{formula}

	Comme tout ce que l'on a vu avant, il existe une définition semblable en $-\infty$.

	\begin{tip}[Définition en $-\infty$]
		Soit $f$ une fonction d'ensemble de définition $\mathcal{D}_f$. On suppose qu'une des bornes de $\mathcal{D}_f$ est $-\infty$.
		\newpar
		Alors si $f(x)$ admet une limite finie $\ell$ quand $x$ tend vers $-\infty$, alors la droite d'équation $y = \ell$ est une \textbf{asymptote horizontale} en $-\infty$ à la courbe représentative de $f$.
	\end{tip}

	\begin{tip}[Exemple]
		En reprenant l'exemple précédent, la courbe représentative de la fonction $x \mapsto \frac{9x}{3x+1}$ admet une asymptote horizontale d'équation $y=3$ en $+\infty$.
		\newpar
		De plus, elle admet une asymptote verticale d'équation $x=-\frac{1}{3}$.
	\end{tip}

	\section{Calcul de limites}

	\subsection{Limites de fonctions de référence}

	Nous allons donner quelques fonctions ``classiques'' avec leur limite en quelques points.

	\begin{formula}[Limites de fonctions usuelles]
    \begin{whitetabularx}{|X|X|X|X|}
				\hline
				& $a = -\infty$ & $a = 0$ & $a = +\infty$ \\
				\hline
				$\lim_{x \rightarrow a} \frac{1}{x}$ & $0$ & $-\infty$ si $a = 0^-$, $+\infty$ si $a = 0^+$ & $0$ \\
				\hline
				$\lim_{x \rightarrow a} \sqrt{x}$ & \textbf{Non définie} & $0$ si $a = 0^+$ & $+\infty$ \\
				\hline
				$\lim_{x \rightarrow a} x^k$ & $-\infty$ si $k$ est impair, $+\infty$ si $k$ est pair & $0$ & $+\infty$ \\
				\hline
				$\lim_{x \rightarrow a} e^x$ & $0$ & $e^0 = 1$ & $+\infty$ \\
				\hline
    \end{whitetabularx}
	\end{formula}

	\begin{tip}[Rappel]
		On rappelle que $0^-$ signifie ``tend vers $0$ mais en restant inférieur à $0$'' et $0^+$ signifie ``tend vers $0$ mais en restant supérieur à $0$''.
	\end{tip}

	\subsection{Opérations sur les limites}

	Dans tout ce qui suit, $f$ et $g$ sont deux fonctions de domaines de définition $\mathcal{D}_f$ et $\mathcal{D}_g$. Soit $a$ un nombre réel appartenant à $\mathcal{D}_f \, \cap \, \mathcal{D}_g$ (ou qui est au moins une borne des deux à la fois). Les tableaux suivants ressemblent beaucoup à ceux qui sont disponibles dans le cours sur \href{https://bacomathiqu.es/cours/terminale/suites/}{les suites} donc vous pouvez bien-sûr n'en retenir qu'un des deux, et tenter à partir de là de retrouver l'autre.

	\medskip
	\begin{formula}[Limite d'une somme]
    \begin{whitetabularx}{|X|l|l|l|l|l|l|}
				\hline
				\multicolumn{7}{|l|}{\textbf{Limite d'une somme}} \\
				\hline
				Si la limite de $f(x)$ quand $x$ tend vers $a$ est... & $\ell$ & $\ell$ & $\ell$ & $+\infty$ & $-\infty$ & $+\infty$ \\
				\hline
				Et la limite de $g$ quand $x$ tend vers $a$ est... & $\ell'$ & $+\infty$ & $-\infty$ & $+\infty$ & $-\infty$ & $-\infty$ \\
				\hline
				Alors la limite de $f + g$ quand $x$ tend vers $a$ est... & $\ell + \ell'$ & $+\infty$ & $-\infty$ & $+\infty$ & $-\infty$ & \textbf{?} \\
				\hline
			\end{whitetabularx}
	\end{formula}

	\begin{formula}[Limite d'un produit]
      \begin{whitetabularx}{|X|l|l|l|l|l|l|l|l|l|}
				\hline
				\multicolumn{10}{|l|}{\textbf{Limite d'un produit}} \\
				\hline
				Si la limite de $f(x)$ quand $x$ tend vers $a$ est... & $\ell$ & $\ell \gt 0$ & $\ell \gt 0$ & $\ell \lt 0$ & $\ell \lt 0$ & $+\infty$ & $+\infty$ & $-\infty$ & $0$ \\
				\hline
				Et la limite de $g$ quand $x$ tend vers $a$ est... & $\ell'$ & $+\infty$ & $-\infty$ & $+\infty$ & $-\infty$ & $+\infty$ & $-\infty$ & $-\infty$ & $\pm \infty$ \\
				\hline
				Alors la limite de $f \times g$ quand $x$ tend vers $a$ est... & $\ell \times \ell'$ & $+\infty$ & $-\infty$ & $-\infty$ & $+\infty$ & $+\infty$ & $-\infty$ & $+\infty$ & \textbf{?} \\
				\hline
			\end{whitetabularx}
	\end{formula}

	\begin{formula}[Limite d'un quotient]
    \begin{whitetabularx}{|X|l|l|l|l|l|l|l|l|l|}
				\hline
				\multicolumn{10}{|l|}{\textbf{Limite d'un quotient}} \\
				\hline
				Si la limite de $f(x)$ quand $x$ tend vers $a$ est... & $\ell$ & $\ell$ & $+\infty$ & $+\infty$ & $-\infty$ & $-\infty$ & $\pm \infty$ & $\ell$ & $0$ \\
				\hline
				Et la limite de $g$ quand $x$ tend vers $a$ est... & $\ell' \neq 0$ & $\pm \infty$ & $\ell' \gt 0$ & $\ell' \lt 0$ & $\ell' \gt 0$ & $\ell' \lt 0$ & $\pm \infty$ & $0$ & $0$ \\
				\hline
				Alors la limite de $\frac{f}{g}$ quand $x$ tend vers $a$ est... & $\displaystyle{\frac{\ell}{\ell'}}$ & $0$ & $+\infty$ & $-\infty$ & $-\infty$ & $+\infty$ & \textbf{?} & $\pm \infty$ & \textbf{?} \\
				\hline
			\end{whitetabularx}
	\end{formula}

	\begin{formula}[Limite d'une composée]
		Si on pose $\lim_{x \rightarrow a} f(x) = b$ et $\lim_{x \rightarrow b} g(x) = c$. Alors $\lim_{x \rightarrow} (g \circ f)(x) = c$.
	\end{formula}

	\begin{tip}[Formes indéterminées]
		À noter qu'il n'existe que 4 formes indéterminées : ``$+\infty - \infty$'', ``$0 \times \pm \infty$'', ``$\frac{\pm \infty}{\pm \infty}$'' et ``$\frac{0}{0}$''.
	\end{tip}

	\subsection{Comparaisons et encadrements}

	\begin{formula}[Théorèmes de comparaison]
		Soient deux fonctions $f$ et $g$.
		\begin{itemize}
			\item Si $\lim_{x \rightarrow +\infty} f(x) = +\infty$ et si $f \leq g$ à partir d'un certain point,
			\newline
			alors $\lim_{x \rightarrow +\infty} g(x) = +\infty$.
			\item Si $\lim_{x \rightarrow +\infty} f(x) = -\infty$ et si $f \geq g$ à partir d'un certain point,
			\newline
			alors $\lim_{x \rightarrow +\infty} g(x) = -\infty$.
		\end{itemize}
	\end{formula}

	\begin{formula}[Théorème des gendarmes]
		Soient trois fonctions $f$, $g$ et $h$. Si on a $f \leq g \leq h$ à partir d'un certain point, et qu'il existe $\ell$ tel que $\lim_{x \rightarrow +\infty} f(x) = \ell$ et $\lim_{x \rightarrow +\infty} h(x) = \ell$, alors $\lim_{x \rightarrow +\infty} g(x) = \ell$.
	\end{formula}

	\begin{tip}[Exemple]
		Utilisons ce théorème pour montrer que la fonction $f : x \mapsto \frac{\sin(x)}{x}$ tend vers $0$ quand $x$ tend vers $+\infty$.
		\newpar
		Tout d'abord, pour tout $x \in \mathbb{R}$, $-1 \leq \sin(x) \leq 1$.
		\newpar
		Donc, pour tout $x \gt 0$, $\frac{-1}{x} \leq \underbrace{\frac{\sin(x)}{x}}_{= f(x)} \leq \frac{1}{x}$.
		\newpar
		Comme, $\lim_{x \rightarrow +\infty} \frac{-1}{x} = 0$ et $\lim_{x \rightarrow +\infty} \frac{1}{x} = 0$, alors $\lim_{x \rightarrow +\infty} f(x) = 0$.
	\end{tip}

	Le dernier théorème est la ``version fonctions'' du théorèmes des gendarmes (que l'on a vu lors du cours sur \href{https://bacomathiqu.es/cours/terminale/suites/}{les suites}). Ils permettent notamment de démontrer une partie du \textbf{théorème des croissances comparées}.

	\begin{formula}[Croissances comparées]
		$\lim_{x \rightarrow +\infty} \frac{e^x}{x^n} = +\infty$ pour tout $n \in \mathbb{N}$.
	\end{formula}

	\begin{demonstration}[Croissances comparées]
		Commençons tout d'abord par montrer que pour tout $x \geq 0$, $e^x \geq 1 + x$. Pour cela, posons $f : x \mapsto e^x - 1 - x$. On a pour tout $x \in \mathbb{R}$, $f'(x) = e^x - 1$. Donc $f'(x)$ est positif si et seulement si $e^x - 1 \geq 0$, c'est-à-dire $e^x \geq 1$.
		\newpar
		En regardant le graphique de la fonction exponentielle, on trouve que cela est équivalent à $x \geq 0$.
		\newpar
		Notre fonction est donc croissante sur l'intervalle $[0, +\infty[$, et son minimum est donc atteint en $x = 0$ et vaut $f(0) = 0$. Ainsi, pour tout $x \geq 0$, $f(x) \geq 0 \iff e^x - 1 - x \geq 0 \iff e^x \geq 1 + x$ : ce que l'on cherchait.
		\newpar
		Pour conclure, on utilise une petite astuce. Soit $n \in \mathbb{N}$ :
		\newpar
		D'après ce que l'on vient de faire, pour tout $x \gt 0$, $e^{\frac{x}{n+1}} \geq 1 + \frac{x}{n+1} \gt \frac{x}{n+1}$. Ainsi, en mettant à la puissance $n + 1$ (qui ne change pas le sens de l'inégalité car les deux membres sont positifs), on a :
		\newpar
		$e^x \gt (\frac{x}{n+1})^{n+1} = \frac{x^{n+1}}{(n+1)^{n+1}}$
		Maintenant, on divise les deux côtés par $x^n$ (qui est un nombre strictement positif) et on obtient :
		\newpar
		$\frac{e^x}{x^n} \gt \frac{x}{(n+1)^{n+1}}$
		\newpar
		Or, le membre de droite tend vers $+\infty$ quand $x$ tend vers $+\infty$ donc le membre de gauche aussi d'après les théorèmes de comparaison.
	\end{demonstration}
    %</content>
\end{document}
