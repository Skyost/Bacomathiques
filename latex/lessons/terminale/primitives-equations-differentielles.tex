\documentclass[12pt, a4paper]{report}

% Packages :

\usepackage[french]{babel}
%\usepackage[utf8]{inputenc}
%\usepackage[T1]{fontenc}
\usepackage[pdfencoding=auto, pdfauthor={Bacomathiques}]{hyperref}
\usepackage{sectsty}
\usepackage[explicit]{titlesec}
\usepackage{xcolor}
%\usepackage{amsmath}
%\usepackage{amssymb}
\usepackage{amsthm}
\usepackage{fourier-otf}
\usepackage{titlesec}
\usepackage{fancyhdr}
\usepackage{catchfilebetweentags}
\usepackage[french, capitalise, noabbrev]{cleveref}
\usepackage[fit, breakall]{truncate}
\usepackage[margin=3cm]{geometry}
\usepackage{tocloft}
\usepackage{tikz}
\usepackage{tocloft}
\usepackage{microtype}
\usepackage{listings}
\usepackage{tabularx}
\usepackage{calc}
\usepackage[export]{adjustbox}
\usepackage[most]{tcolorbox}
\usepackage{standalone}
\usepackage{xlop}
\usepackage{etoolbox}
\usepackage{environ}

\usetikzlibrary{arrows.meta}
\usetikzlibrary{trees}

% Police :

\setmathfont{Erewhon Math}

% Paramètres :

\author{Bacomathiques}
\definecolor{graphe}{HTML}{93c9ff}
\setcounter{MaxMatrixCols}{12}
\setlength{\parindent}{0pt}
\setlength{\fboxsep}{0pt}
%\pdfsuppresswarningpagegroup=1

% Code :

\lstdefinestyle{style}{
	backgroundcolor=\color{white},
	commentstyle=\em\color[HTML]{999988},
	keywordstyle=\bfseries,
	identifierstyle=\normalfont,
	stringstyle=\color[rgb]{0.87, 0.07, 0.27},
	basicstyle=\ttfamily\color{black},
	breakatwhitespace=false,
	breaklines=true,
	captionpos=b,
	keepspaces=true,
	numbers=left,
	numbersep=5pt,
	showspaces=false,
	showstringspaces=false,
	showtabs=false,
	tabsize=2,
	numbers=none
}

\lstset{style=style}
\lstset{
	literate=
	{á}{{\'a}}1
	{à}{{\`a}}1
	{ã}{{\~a}}1
	{é}{{\'e}}1
	{ê}{{\^e}}1
	{í}{{\'i}}1
	{ó}{{\'o}}1
	{õ}{{\~o}}1
	{ú}{{\'u}}1
	{ü}{{\"u}}1
	{ç}{{\c{c}}}1
}

\lstset{
	framextopmargin=10pt,
	framexrightmargin=10pt,
	framexbottommargin=10pt,
	framexleftmargin=10pt,
	xleftmargin=10pt,
	xrightmargin=10pt,
}

% Couleurs :

\definecolor{title}{HTML}{912c21}
\definecolor{section}{HTML}{1c567d}
\definecolor{subsection}{HTML}{2980b9}

\definecolor{rule}{HTML}{c4c4c4}

\definecolor{formula}{HTML}{ebf3fb}
\definecolor{formula-left}{HTML}{3583d6}

\definecolor{tip}{HTML}{dcf3d8}
\definecolor{tip-left}{HTML}{26a65b}

\definecolor{demonstration}{HTML}{fff8de}
\definecolor{demonstration-left}{HTML}{f1c40f}

\definecolor{exercise}{HTML}{e0f2f1}
\definecolor{exercise-left}{HTML}{009688}

\definecolor{correction}{HTML}{e0f7fa}
\definecolor{correction-left}{HTML}{00bcd4}

\definecolor{toc}{HTML}{fceae9}
\definecolor{toc-left}{HTML}{e74c3c}
\definecolor{toc-dark}{HTML}{87281f}

% Titres :

\renewcommand{\thesection}{\Roman{section} - }
\renewcommand{\thesubsection}{\arabic{subsection}. }

\newcommand{\sectionstyle}{\normalfont\LARGE\bfseries\color{section}}
\titleformat{\section}{\sectionstyle}{\thesection #1}{0pt}{}
\titleformat{name=\section, numberless}{\sectionstyle}{#1}{0pt}{}

\newcommand{\subsectionstyle}{\normalfont\Large\bfseries\color{subsection}}
\titleformat{\subsection}{\subsectionstyle}{\thesubsection #1}{0pt}{}
\titleformat{name=\subsection, numberless}{\subsectionstyle}{#1}{0pt}{}

\titlelabel{\thetitle\ }

% Table des matières :

\addto\captionsfrench{\renewcommand\contentsname{}}
\renewcommand{\cftsecpagefont}{\color{toc-dark}}
\renewcommand{\cftsubsecpagefont}{\color{toc-dark}}
\renewcommand{\cftsecleader}{\cftdotfill{\cftdotsep}}
\renewcommand{\cftsecfont}{\bfseries}
\renewcommand{\cftsecpagefont}{\bfseries\color{toc-dark}}
\setlength{\cftbeforetoctitleskip}{0pt}
\setlength{\cftaftertoctitleskip}{0pt}
\setlength{\cftsecindent}{0pt}
\setlength{\cftsubsecindent}{20pt}
\setlength{\cftsubsecnumwidth}{20pt}

% Commandes :

\newcommand{\newpar}{\\[\medskipamount]}
\newcommand{\lesson}[3]{%
	\newcommand{\level}{#1}%
	\newcommand{\id}{#2}%
	\hypersetup{pdftitle={#3}}
	\begin{center}%
		\includegraphics[width=150px]{\imagespath/bacomathiques}%
		
		\vspace{30pt}%
		{\Huge\color{title} #3}%
		
		\vspace{10pt}%
		{Bacomathiques --- \href{https://bacomathiqu.es/cours/#1/#2}{\color{section} https://bacomathiqu.es}}%
		
		\vspace{20pt}%
	\end{center}%
	\begin{toc}
		\tableofcontents%
	\end{toc}
	\thispagestyle{empty}%
	\newpage%
	\setcounter{page}{1}%
}
\newcommand{\imagespath}{../../images}
\newcommand{\lessonimagespath}{\imagespath/lessons/\level/\id/}
\newcommand{\includelatexpicture}[2][\textwidth - 100pt]{%
	\begin{center}%
		\resizebox{#1}{!}{%
			\input{\lessonimagespath#2}%
		}%
	\end{center}%
	\medskip%
}
\newcommand{\includeimage}[1]{%
	\begin{center}%
		\includegraphics{\lessonimagespath#1}%
	\end{center}%
	\medskip%
}
\newcommand{\includerepresentation}[1]{%
	\begin{center}%
		\setlength{\fboxrule}{0.5pt}%
		\href{https://www.geogebra.org/m/#1}{\includegraphics[width=\textwidth-1pt,fbox]{\lessonimagespath#1}}%
	\end{center}%
}
\newcommand{\floor}[1]{\lfloor #1 \rfloor}

\makeatletter
\newcommand\inputcontent{\@ifstar{\inputcontent@star}{\inputcontent@nostar}}
\newcommand{\inputcontent@star}[1]{%
	\ExecuteMetaData[#1]{content}%
}
\newcommand{\inputcontent@nostar}[1]{%
	\newpage%
	\inputcontent@star{#1}%
}
\makeatother

\let\oldsection\section
\renewcommand\section{\clearpage\oldsection}
\newcommand{\contentwidth}[1][medium]{}

% En-têtes :

\pagestyle{fancy}

\renewcommand{\sectionmark}[1]{\markboth{\thesection \ #1}{}}

\fancyhead[R]{\truncate{0.23\textwidth}{\color{title}\thepage}}
\fancyhead[L]{\truncate{0.73\textwidth}{\color{title}\leftmark}}
\fancyfoot[C]{\scriptsize \href{https://bacomathiqu.es/cours/\level/\id}{\texttt{bacomathiqu.es}}}

\makeatletter
\patchcmd{\f@nch@head}{\rlap}{\color{rule}\rlap}{}{}
\patchcmd{\headrule}{\hrule}{\color{rule}\hrule}{}{}
\makeatother

% Environnements :

\newenvironment{nosummary}{}{}
\newcommand{\tcolorboxtitle}[2]{\setlength{\fboxsep}{2.5pt}\hspace{-10pt}\colorbox{#1-left}{\hspace{8pt}\MakeUppercase{#2} \hspace{2pt} \includegraphics[height=0.8em]{\imagespath/bubbles/#1}\hspace{5pt}}}
\newcommand{\tcolorboxsubtitle}[2]{\ifstrempty{#2}{}{\textcolor{#1-left}{\large#2}\\[\medskipamount]}}
\tcbset{
	frame hidden,
	boxrule=0pt,
	boxsep=0pt,
	enlarge bottom by=8.5pt,
	enhanced jigsaw,
	boxed title style={sharp corners,boxrule=0pt,coltitle={white},titlerule=0pt},
	fonttitle=\fontsize{6pt}{6pt}\bfseries\boldmath,
	top=10pt,
	right=10pt,
	bottom=10pt,
	left=10pt,
	arc=0pt,
	outer arc=0pt,
}
\newtcolorbox{toc}[1][]{
	colback=toc,
	borderline west={3pt}{0pt}{toc-left},
	title=\tcolorboxtitle{toc}{Table des matières},
	colbacktitle=toc,
	before upper={\tcolorboxsubtitle{toc}{#1}}
}
\newtcolorbox{formula}[1][]{
	colback=formula,
	borderline west={3pt}{0pt}{formula-left},
	title=\tcolorboxtitle{formula}{À retenir},
	colbacktitle=formula,
	before upper={\tcolorboxsubtitle{formula}{#1}}
}
\newtcolorbox{tip}[1][]{
	colback=tip,
	borderline west={3pt}{0pt}{tip-left},
	title=\tcolorboxtitle{tip}{À lire},
	colbacktitle=tip,
	before upper={\tcolorboxsubtitle{tip}{#1}}
}
\newtcolorbox{demonstration}[1][]{
	colback=demonstration,
	borderline west={3pt}{0pt}{demonstration-left},
	title=\tcolorboxtitle{demonstration}{Démonstration},
	colbacktitle=demonstration,
	before upper={\tcolorboxsubtitle{demonstration}{#1}}
}

\NewEnviron{whitetabularx}[1]{%
	\renewcommand{\arraystretch}{2.5}
	\colorbox{white}{%
		\begin{tabularx}{\textwidth}{#1}%
			\BODY%
		\end{tabularx}%
	}%
}

% Longueurs :

\newlength{\espacetitreliste}
\setlength{\espacetitreliste}{-16pt}
\newcommand{\entretitreetliste}{\vspace{\espacetitreliste}}

\begin{document}
	%<*content>
	\lesson{terminale}{6}{primitives-equations-differentielles}{Primitives et équations différentielles}

	\header{caption}{Les primitives sont notamment utilisées dans dans le cadre de la dynamique
		Newtonienne.}

	\header{excerpt}{Une primitive d'une fonction d'une variable réelle définie sur un intervalle
		est une autre fonction, définie et dérivable sur cet intervalle, et dont la dérivée
		et notre fonction de départ. \newpar Ce cours contient essentiellement (en plus de la
		définition d'une primitive) que les tableaux à connaître pour pouvoir trouver les
		primitives des fonctions usuelles, mais également des méthodes de résolution pour
		certaines équations différentielles !}

	\header{difficulty}{3}

	\section{Primitives de fonctions continues}

	\subsection{Définition}

	\begin{formula}[Définition]
		Soit $f$ une fonction définie et continue sur un intervalle $I$. On appelle \textbf{primitive} de $f$, toute fonction $F$ définie sur $I$ et qui vérifie pour tout $x \in I$ :
		\[ F'(x) = f(x) \]
	\end{formula}

	\begin{tip}[Note]
		Une primitive est toujours définie à une constante près.
		\newpar
		En effet. On considère la fonction $f$ définie pour tout $x \in \mathbb{R}$ par $f(x) = 2x$. Alors, $F_{1} : x \mapsto x^2 + 1$ est une primitive de la fonction $f$ (car pour tout $x$, $F'(x) = 2x = f(x)$).
		\newpar
		Mais $F_{1}$ \textbf{n'est pas la seule primitive} de $f$ ! On peut citer par exemple $F_{2} : x \mapsto x^2 + 10$ et $F_{3} : x \mapsto x^2 + 3$ qui sont également des primitives de $f$.
		\newpar
		C'est pour cette raison que l'on dit que les primitives sont définies à une constante près (lorsque l'on dérive, la constante devient nulle).
	\end{tip}

	Ainsi, toute \textbf{fonction continue} sur un intervalle admet \textbf{une infinité de primitives} d'une forme particulière sur cet intervalle. Plus formellement :

	\begin{formula}[Infinité de primitives]
		Une fonction continue $f$ sur un intervalle $I$ admet une infinité de primitives sur $I$ de la forme $x \mapsto F_0(x) + c$ avec $c \in \mathbb{R}$ (où $F_0$ est une primitive de $f$).
	\end{formula}

	\begin{demonstration}[Infinité de primitives]
		Soit $F$ une autre primitive de $f$ sur $I$. On a pour tout $x \in I$ :
		\newpar
		$(F - F_0)'(x) = F'(x) - F_0'(x) = f(x) - f(x) = 0$ (car $F_0$ et $F$ sont deux primitives de $f$).
		\newpar
		Donc il existe une constante réelle $c$ telle que $F - F_0 = c$. D'où pour tout $x \in I$, $F(x) = F_0(x) + c$ : ce qu'il fallait démontrer.
	\end{demonstration}

	\subsection{Primitive de fonctions usuelles}

	Le tableau suivant est à connaître (mais il peut être obtenu en prenant celui des dérivées usuelles à l'envers) :

	\begin{formula}
		Soient $\lambda$ et $a$ deux constantes réelles avec $a \neq 1$.
		\newpar
		\begin{whitetabularx}{|X|X|X|}
			\hline
			\textbf{Fonction} & \textbf{Primitive} & \textbf{Domaine de définition de la primitive} \\
			\hline
			$x \mapsto \lambda$ & $x \mapsto \lambda x$ & $\mathbb{R}$ \\
			\hline
			$x \mapsto e^x$ & $x \mapsto e^x$ & $\mathbb{R}$ \\
			\hline
      		$x \mapsto \frac{1}{x}$ & $x \mapsto \ln(x)$ & $\mathbb{R}^{*}_{+}$ \\
			\hline
			$x \mapsto \frac{1}{\sqrt{x}}$ & $x \mapsto 2\sqrt{x}$ & $\mathbb{R}^{*}_{+}$ \\
			\hline
			$x \mapsto x^a$ & $x \mapsto \frac{1}{a + 1} x^{a + 1}$ & $\mathbb{R}^{*}_{+}$ \\
			\hline
			$x \mapsto \sin(x)$ & $x \mapsto -\cos(x)$ & $\mathbb{R}$ \\
			\hline
			$x \mapsto \cos(x)$ & $x \mapsto \sin(x)$ & $\mathbb{R}$ \\
			\hline
		\end{whitetabularx}
	\end{formula}

	\subsection{Opérations sur les primitives}

	Le tableau suivant est également à connaître (mais il peut être obtenu en prenant celui des dérivées usuelles à l'envers) :

	\begin{formula}
		Soit $u$ une fonction continue.
		\newpar
		\begin{whitetabularx}{|X|X|X|}
			\hline
			\textbf{Fonction} & \textbf{Primitive} & \textbf{Domaine de définition de la primitive} \\
			\hline
			$u'e^u$ & $e^u$ & En tout point où $u$ est définie. \\
			\hline
			$\frac{u'}{u}$ & $\ln(|u|)$ & En tout point où $u$ est définie et est non-nulle. On peut retirer la valeur absolue si $u$ est positive. \\
			\hline
			$\frac{u'}{\sqrt{u}}$ & $2\sqrt{u}$ & En tout point où $u$ est définie et est strictement positive. \\
			\hline
			$u' (u)^a$ avec $a \in \mathbb{R}$ et $a \neq -1$ & $\frac{1}{a + 1} u^{a + 1}$ & En tout point où $u$ est définie. \\
			\hline
			$u' \sin(u)$ & $-\cos(u)$ & En tout point où $u$ est définie. \\
			\hline
			$u' \cos(u)$ & $\sin(u)$ & En tout point où $u$ est définie. \\
			\hline
		\end{whitetabularx}
	\end{formula}

	\section{Équations différentielles}

	\subsection{Qu'est-ce-qu'une équation différentielle ?}

	Commençons cette partie par quelques définitions.

	\begin{formula}[Définition]
		\entretitreetliste
		\begin{itemize}
			\item Une \textbf{équation différentielle} est une égalité liant une fonction inconnue $y$ à ses dérivées successives ($y'$, $y''$, ...) contenant éventuellement d'autres fonctions connues.
			\item Une \textbf{solution} d'une équation différentielle est une fonction vérifiant l'égalité décrite précédemment.
		\end{itemize}
	\end{formula}

	\begin{tip}[Exemple]
		La fonction logarithme est une solution de l'équation différentielle $y' = \frac{1}{x}$.
		\newpar
		La fonction exponentielle est une solution de l'équation différentielle $y' = y$, mais aussi de l'équation différentielle $y'' = y$, etc.
	\end{tip}

	\subsection{Résolution d'équations différentielles de la forme $y'=ay$}

	Nous allons donner une formule permettant de résoudre des équations différentielles de la forme $y' = ay$.

	\begin{formula}[Formule]
		On pose $(E) : y'=ay$ (où $a$ est un réel). Alors l'ensemble des solutions de $(E)$ est l'ensemble des fonctions $x \mapsto c e^{ax}$ où $c \in \mathbb{R}$.
	\end{formula}

	\begin{demonstration}
		Vérifions tout d'abord que les fonctions $x \mapsto k e^{ax}$ sont solutions de $(E)$. Soit $c \in \mathbb{R}$, posons pour tout $x \in \mathbb{R}$, $y_c(x) = c e^{ax}$.
		\newpar
		Alors pour tout $x \in \mathbb{R}$, $y'_c(x) = ac e^{ax}$ et $ay_c(x) = ac e^{ax}$. Donc $y'_c = a y_c$ : $y_c$ est bien solution de $(E)$.
		\newpar
		Montrons que les fonctions $y_c$ sont les seules solutions de $(E)$. Soit $y$ une solution quelconque de $(E)$ sur $\mathbb{R}$. Pour tout $x \in \mathbb{R}$, on pose $z(x) = y(x) e^{-ax}$. En dérivant :
		\newpar
		$z'(x) = y'(x) e^{-ax} + y(x) (-ae^{-ax}) = e^{-ax}(y'(x) - ay(x))$
		\newpar
		De plus, comme $y$ est solution de $(E)$, on a $y' - ay = 0$, donc $z' = 0$.
		\newpar
		Ainsi, il existe une constante réelle $c$ telle que $z = c$. C'est-à-dire que pour tout $x \in \mathbb{R}$ :
		\newpar
		$c = y(x) e^{-ax} \iff y(x) = c e^{ax}$. Ce qui termine la preuve.
	\end{demonstration}

	\begin{formula}[Théorème]
		Pour tout réels $x_0$ et $y_0$, il existe une \textbf{unique} fonction $y$ solution de l'équation différentielle $(E)$ telle que $y(x_0) = y_0$.
	\end{formula}

	\begin{tip}[Exemple]
		Résolvons l'équation différentielle $(E) : y' - 5y = 0$ sous condition d'avoir $y(0) = 1$.
		\newpar
		Dans un premier temps, on écrit l'équation sous une meilleure forme : $y' - 5y = 0 \iff y' = 5y$. On a donc $a = 5$. Les solutions de l'équation $(E)$ sont les fonctions définies $x \mapsto c e^{5x}$ où $c \in \mathbb{R}$.
		\newpar
		Maintenant, il faut trouver la fonction $y$ qui vaut $1$ en $0$. Soit donc $y$ une telle solution de $(E)$. Alors :
		\newpar
		$y(0) = 1 \iff c e^{5 \times 0} = 1 \iff c = e^{-1}$. La solution recherchée est donc la fonction $y : x \mapsto e^{-1} e^{5x}$.
	\end{tip}

	\subsection{Résolution d'équations différentielles de la forme \texorpdfstring{$y'=ay+b$}{y'=ay+b}}

	Nous allons donner une formule permettant de résoudre des équations différentielles de la forme $y' = ay+b$.

	\begin{formula}[Formule]
		On pose $(E) : y'=ay+b$ (où $a$ est un réel non-nul et $b$ est un réel). Alors l'ensemble des solutions de $(E)$ est l'ensemble des fonctions $x \mapsto c e^{ax} - \frac{b}{a}$ où $c \in \mathbb{R}$.
	\end{formula}

	\begin{formula}[Théorème]
		Pour tout réels $x_0$ et $y_0$, il existe une \textbf{unique} fonction $y$ solution de l'équation différentielle $(E)$ telle que $y(x_0) = y_0$.
	\end{formula}

	\begin{tip}[Exemple]
		Résolvons l'équation différentielle $(E) : y'=2y-1$ sous condition d'avoir $y(1) = 0$.
		\newpar
		On a donc $a = 2$ et $b = -1$. Les solutions de l'équation $(E)$ sont les fonctions définies $x \mapsto c e^{2x} + \frac{1}{2}$ où $c \in \mathbb{R}$.
		\newpar
		Maintenant, il faut trouver la fonction $y$ qui vaut $0$ en $1$. Soit donc $y$ une telle solution de $(E)$. Alors :
		\[ y(1) =  0 \iff c e^{2 \times 1} + \frac{1}{2} = 0 \iff c = -\frac{1}{2e^2} \]
		La solution recherchée est donc la fonction $y : x \mapsto -\frac{e^{2x}}{2e^2} + \frac{1}{2}$.
	\end{tip}
	%</content>
\end{document}
