\documentclass[12pt, a4paper]{report}

% Packages :

\usepackage[french]{babel}
%\usepackage[utf8]{inputenc}
%\usepackage[T1]{fontenc}
\usepackage[pdfencoding=auto, pdfauthor={Bacomathiques}]{hyperref}
\usepackage{sectsty}
\usepackage[explicit]{titlesec}
\usepackage{xcolor}
%\usepackage{amsmath}
%\usepackage{amssymb}
\usepackage{amsthm}
\usepackage{fourier-otf}
\usepackage{titlesec}
\usepackage{fancyhdr}
\usepackage{catchfilebetweentags}
\usepackage[french, capitalise, noabbrev]{cleveref}
\usepackage[fit, breakall]{truncate}
\usepackage[margin=3cm]{geometry}
\usepackage{tocloft}
\usepackage{tikz}
\usepackage{tocloft}
\usepackage{microtype}
\usepackage{listings}
\usepackage{tabularx}
\usepackage{calc}
\usepackage[export]{adjustbox}
\usepackage[most]{tcolorbox}
\usepackage{standalone}
\usepackage{xlop}
\usepackage{etoolbox}
\usepackage{environ}

\usetikzlibrary{arrows.meta}
\usetikzlibrary{trees}

% Police :

\setmathfont{Erewhon Math}

% Paramètres :

\author{Bacomathiques}
\definecolor{graphe}{HTML}{93c9ff}
\setcounter{MaxMatrixCols}{12}
\setlength{\parindent}{0pt}
\setlength{\fboxsep}{0pt}
%\pdfsuppresswarningpagegroup=1

% Code :

\lstdefinestyle{style}{
	backgroundcolor=\color{white},
	commentstyle=\em\color[HTML]{999988},
	keywordstyle=\bfseries,
	identifierstyle=\normalfont,
	stringstyle=\color[rgb]{0.87, 0.07, 0.27},
	basicstyle=\ttfamily\color{black},
	breakatwhitespace=false,
	breaklines=true,
	captionpos=b,
	keepspaces=true,
	numbers=left,
	numbersep=5pt,
	showspaces=false,
	showstringspaces=false,
	showtabs=false,
	tabsize=2,
	numbers=none
}

\lstset{style=style}
\lstset{
	literate=
	{á}{{\'a}}1
	{à}{{\`a}}1
	{ã}{{\~a}}1
	{é}{{\'e}}1
	{ê}{{\^e}}1
	{í}{{\'i}}1
	{ó}{{\'o}}1
	{õ}{{\~o}}1
	{ú}{{\'u}}1
	{ü}{{\"u}}1
	{ç}{{\c{c}}}1
}

\lstset{
	framextopmargin=10pt,
	framexrightmargin=10pt,
	framexbottommargin=10pt,
	framexleftmargin=10pt,
	xleftmargin=10pt,
	xrightmargin=10pt,
}

% Couleurs :

\definecolor{title}{HTML}{912c21}
\definecolor{section}{HTML}{1c567d}
\definecolor{subsection}{HTML}{2980b9}

\definecolor{rule}{HTML}{c4c4c4}

\definecolor{formula}{HTML}{ebf3fb}
\definecolor{formula-left}{HTML}{3583d6}

\definecolor{tip}{HTML}{dcf3d8}
\definecolor{tip-left}{HTML}{26a65b}

\definecolor{demonstration}{HTML}{fff8de}
\definecolor{demonstration-left}{HTML}{f1c40f}

\definecolor{exercise}{HTML}{e0f2f1}
\definecolor{exercise-left}{HTML}{009688}

\definecolor{correction}{HTML}{e0f7fa}
\definecolor{correction-left}{HTML}{00bcd4}

\definecolor{toc}{HTML}{fceae9}
\definecolor{toc-left}{HTML}{e74c3c}
\definecolor{toc-dark}{HTML}{87281f}

% Titres :

\renewcommand{\thesection}{\Roman{section} - }
\renewcommand{\thesubsection}{\arabic{subsection}. }

\newcommand{\sectionstyle}{\normalfont\LARGE\bfseries\color{section}}
\titleformat{\section}{\sectionstyle}{\thesection #1}{0pt}{}
\titleformat{name=\section, numberless}{\sectionstyle}{#1}{0pt}{}

\newcommand{\subsectionstyle}{\normalfont\Large\bfseries\color{subsection}}
\titleformat{\subsection}{\subsectionstyle}{\thesubsection #1}{0pt}{}
\titleformat{name=\subsection, numberless}{\subsectionstyle}{#1}{0pt}{}

\titlelabel{\thetitle\ }

% Table des matières :

\addto\captionsfrench{\renewcommand\contentsname{}}
\renewcommand{\cftsecpagefont}{\color{toc-dark}}
\renewcommand{\cftsubsecpagefont}{\color{toc-dark}}
\renewcommand{\cftsecleader}{\cftdotfill{\cftdotsep}}
\renewcommand{\cftsecfont}{\bfseries}
\renewcommand{\cftsecpagefont}{\bfseries\color{toc-dark}}
\setlength{\cftbeforetoctitleskip}{0pt}
\setlength{\cftaftertoctitleskip}{0pt}
\setlength{\cftsecindent}{0pt}
\setlength{\cftsubsecindent}{20pt}
\setlength{\cftsubsecnumwidth}{20pt}

% Commandes :

\newcommand{\newpar}{\\[\medskipamount]}
\newcommand{\lesson}[3]{%
	\newcommand{\level}{#1}%
	\newcommand{\id}{#2}%
	\hypersetup{pdftitle={#3}}
	\begin{center}%
		\includegraphics[width=150px]{\imagespath/bacomathiques}%
		
		\vspace{30pt}%
		{\Huge\color{title} #3}%
		
		\vspace{10pt}%
		{Bacomathiques --- \href{https://bacomathiqu.es/cours/#1/#2}{\color{section} https://bacomathiqu.es}}%
		
		\vspace{20pt}%
	\end{center}%
	\begin{toc}
		\tableofcontents%
	\end{toc}
	\thispagestyle{empty}%
	\newpage%
	\setcounter{page}{1}%
}
\newcommand{\imagespath}{../../images}
\newcommand{\lessonimagespath}{\imagespath/lessons/\level/\id/}
\newcommand{\includelatexpicture}[2][\textwidth - 100pt]{%
	\begin{center}%
		\resizebox{#1}{!}{%
			\input{\lessonimagespath#2}%
		}%
	\end{center}%
	\medskip%
}
\newcommand{\includeimage}[1]{%
	\begin{center}%
		\includegraphics{\lessonimagespath#1}%
	\end{center}%
	\medskip%
}
\newcommand{\includerepresentation}[1]{%
	\begin{center}%
		\setlength{\fboxrule}{0.5pt}%
		\href{https://www.geogebra.org/m/#1}{\includegraphics[width=\textwidth-1pt,fbox]{\lessonimagespath#1}}%
	\end{center}%
}
\newcommand{\floor}[1]{\lfloor #1 \rfloor}

\makeatletter
\newcommand\inputcontent{\@ifstar{\inputcontent@star}{\inputcontent@nostar}}
\newcommand{\inputcontent@star}[1]{%
	\ExecuteMetaData[#1]{content}%
}
\newcommand{\inputcontent@nostar}[1]{%
	\newpage%
	\inputcontent@star{#1}%
}
\makeatother

\let\oldsection\section
\renewcommand\section{\clearpage\oldsection}
\newcommand{\contentwidth}[1][medium]{}

% En-têtes :

\pagestyle{fancy}

\renewcommand{\sectionmark}[1]{\markboth{\thesection \ #1}{}}

\fancyhead[R]{\truncate{0.23\textwidth}{\color{title}\thepage}}
\fancyhead[L]{\truncate{0.73\textwidth}{\color{title}\leftmark}}
\fancyfoot[C]{\scriptsize \href{https://bacomathiqu.es/cours/\level/\id}{\texttt{bacomathiqu.es}}}

\makeatletter
\patchcmd{\f@nch@head}{\rlap}{\color{rule}\rlap}{}{}
\patchcmd{\headrule}{\hrule}{\color{rule}\hrule}{}{}
\makeatother

% Environnements :

\newenvironment{nosummary}{}{}
\newcommand{\tcolorboxtitle}[2]{\setlength{\fboxsep}{2.5pt}\hspace{-10pt}\colorbox{#1-left}{\hspace{8pt}\MakeUppercase{#2} \hspace{2pt} \includegraphics[height=0.8em]{\imagespath/bubbles/#1}\hspace{5pt}}}
\newcommand{\tcolorboxsubtitle}[2]{\ifstrempty{#2}{}{\textcolor{#1-left}{\large#2}\\[\medskipamount]}}
\tcbset{
	frame hidden,
	boxrule=0pt,
	boxsep=0pt,
	enlarge bottom by=8.5pt,
	enhanced jigsaw,
	boxed title style={sharp corners,boxrule=0pt,coltitle={white},titlerule=0pt},
	fonttitle=\fontsize{6pt}{6pt}\bfseries\boldmath,
	top=10pt,
	right=10pt,
	bottom=10pt,
	left=10pt,
	arc=0pt,
	outer arc=0pt,
}
\newtcolorbox{toc}[1][]{
	colback=toc,
	borderline west={3pt}{0pt}{toc-left},
	title=\tcolorboxtitle{toc}{Table des matières},
	colbacktitle=toc,
	before upper={\tcolorboxsubtitle{toc}{#1}}
}
\newtcolorbox{formula}[1][]{
	colback=formula,
	borderline west={3pt}{0pt}{formula-left},
	title=\tcolorboxtitle{formula}{À retenir},
	colbacktitle=formula,
	before upper={\tcolorboxsubtitle{formula}{#1}}
}
\newtcolorbox{tip}[1][]{
	colback=tip,
	borderline west={3pt}{0pt}{tip-left},
	title=\tcolorboxtitle{tip}{À lire},
	colbacktitle=tip,
	before upper={\tcolorboxsubtitle{tip}{#1}}
}
\newtcolorbox{demonstration}[1][]{
	colback=demonstration,
	borderline west={3pt}{0pt}{demonstration-left},
	title=\tcolorboxtitle{demonstration}{Démonstration},
	colbacktitle=demonstration,
	before upper={\tcolorboxsubtitle{demonstration}{#1}}
}

\NewEnviron{whitetabularx}[1]{%
	\renewcommand{\arraystretch}{2.5}
	\colorbox{white}{%
		\begin{tabularx}{\textwidth}{#1}%
			\BODY%
		\end{tabularx}%
	}%
}

% Longueurs :

\newlength{\espacetitreliste}
\setlength{\espacetitreliste}{-16pt}
\newcommand{\entretitreetliste}{\vspace{\espacetitreliste}}

\begin{document}
	%<*content>
	\lesson{terminale}{fonction-logarithme}{Chapitre V – La fonction logarithme népérien}

	\section{Propriétés du logarithme népérien}

	\subsection{Définition}

	\begin{formula}[Définition]
		Le \textbf{logarithme népérien} est la fonction définie sur $]0;+\infty[$ par $x \mapsto \ln(x)$.
	\end{formula}

	\begin{formula}
		On a la relation fondamentale suivante pour tout $x \gt 0$ et $y$ réels :
		\newpar
		$\ln(x) = y \iff x = e^y$.
	\end{formula}

	Ainsi, a tout réel \textbf{strictement positif} $x$, la fonction logarithme népérien y associe \textbf{son unique antécédent} $y$ par rapport à \href{https://bacomathiqu.es/cours/premiere/fonction-exponentielle/}{la fonction exponentielle}. De même pour la fonction exponentielle.

	On dit que ces fonctions sont des \textbf{fonctions réciproques} (à la manière de $\sin$ et $\arcsin$ ou $\cos$ et $\arccos$).

	\begin{tip}[Exemple]
		Cette relation peut sembler compliquer à assimiler mais il n'en est rien ! Prenons $x = 0$, on a :
		\newpar
		$e^0 = 1$ (tout réel mis à la puissance zéro vaut un), la relation précédente nous donne $\ln(1) = 0$.
		\newpar
		Si on prend maintenant $x = 1$, on a :
		\newpar
		$e^1 = e$, on a donc $\ln(e) = 1$.
	\end{tip}

	Les relations suivantes sont par conséquent disponibles :

	\begin{formula}[Relations entre fonctions réciproques]
		Pour tout réel $x$ \textbf{strictement positif}, on a $e^{\ln(x)} = x$.
		\newpar
		Et pour tout réel $x$, on a $\ln(e^x) = x$.
	\end{formula}

	\subsection{Relations algébriques}

	Le logarithme népérien a plusieurs propriétés intéressantes qu'il faut connaître.

	\begin{formula}[Formules]
		Pour tous réels $x$ et $y$ \textbf{strictement positifs} :
		\begin{itemize}
			\item $\ln(x \times y) = \ln(x) + \ln(y)$
			\item $\ln(x^n) = n \times \ln(x)$ pour $n \in \mathbb{Z}$
			\item $\displaystyle{\ln\left(\frac{x}{y}\right) = \ln(x) - \ln(y)}$
			\item $\displaystyle{\ln\left(\frac{1}{y}\right) = -\ln(y)}$
			\item $\displaystyle{\ln(\sqrt[p]{x}) = \frac{1}{p} \times \ln(x)}$ pour $p \in \mathbb{N}^*$
		\end{itemize}
	\end{formula}

	Certaines de ces propriétés peuvent se déduire les unes des autres.

	\subsection{Représentation graphique}
	\label{representation-graphique}

	Voici une représentation graphique de la fonction logarithme népérien :

	\includerepresentation{wsntfeab}

	On voit sur ce graphique plusieurs propriétés données précédemment : $\ln(1) = 0$ et $\ln(e) = 1$ par exemple.
	On trace maintenant le graphe de la fonction logarithme népérien, avec celui de la fonction exponentielle. On trace également la droite d'équation $y = x$ :

	\includerepresentation{aymmb94w}

	On remarque plusieurs choses : le graphe de la fonction logarithme népérien est le symétrique de celui de la fonction exponentielle par rapport à la
	droite $y = x$ et on voit que la fonction logarithme népérien croît moins vite que la fonction puissance qui elle-même croît moins vite que la fonction exponentielle. Cette propriété est importante : c'est la \textbf{croissance comparée}.

	\section{Étude de la fonction}

	\subsection{Limites}

	\begin{formula}[Limites]
		Les limites de la fonction logarithme népérien aux bornes de son ensemble de définition sont :
		\begin{itemize}
			\item $\lim_{x \rightarrow 0^+} \ln(x) = -\infty$
			\item $\lim_{x \rightarrow +\infty} \ln(x) = +\infty$
		\end{itemize}
	\end{formula}

	Il faut aussi savoir que la fonction puissance ``l'emporte'' sur le logarithme népérien (voir la partie \hyperref[representation-graphique]{``Représentation graphique''}).

	\begin{formula}[Croissances comparées]
		Pour tout $n \in \mathbb{N}$ :
		\begin{itemize}
			\item $\displaystyle{\lim_{x \rightarrow +\infty} \frac{\ln(x)}{x^n} = 0}$.
			\item $\displaystyle{\lim_{x \rightarrow 0^+} x^n \ln(x) = 0}$.
		\end{itemize}
	\end{formula}

	\begin{demonstration}[Croissances comparées]
		Nous allons démontrer le second point en utilisant le premier (qui n'est pas éligible à une démonstration au lycée) dans le cas $n = 1$. Pour tout $x \gt 0$, posons $y = \frac{1}{x}$.
		\newpar
		On a donc pour tout, $x \ln(x) = \frac{1}{y} \ln\left(\frac{1}{y}\right) = -\frac{\ln(y)}{y}$.
		\newpar
		Or, quand $x$ tend vers $0^+$, $y$ tend vers $+\infty$. Par le premier point :
		\newpar
		$\displaystyle{\lim\limits_{\substack{y \rightarrow +\infty}} \frac{\ln(y)}{y} = 0 \iff \lim\limits_{\substack{y \rightarrow +\infty}} -\frac{\ln(y)}{y} = 0}$.
		\newpar
		Et en remplaçant $y$ par $\frac{1}{x}$ dans le résultat ci-dessus, on a bien ce que l'on cherchait.
	\end{demonstration}

	Pour finir, on donne une limite qu'il peut être utile de savoir redémontrer.

	\begin{formula}
		$\displaystyle{\lim\limits_{\substack{x \rightarrow 0}} \frac{\ln(1 + x)}{x} = 1}$
	\end{formula}

	\begin{demonstration}
		La fonction logarithme népérien est dérivable en $1$ (voir sous-section suivante), on peut donc écrire :
		\newpar
		$\displaystyle{\ln'(1) = \lim\limits_{\substack{x \rightarrow 1}} \frac{\ln(x) - \ln(1)}{x - 1}}$
		\newpar
		Ce qui est équivalent à (car on a $\ln(1) = 0$ et $\ln'(1) = 1$) :
		\newpar
		$\displaystyle{1 = \lim\limits_{\substack{x \rightarrow 1}} \frac{\ln(x)}{x - 1}}$
		\newpar
		On pose $y = x-1$ ce qui nous donne finalement :
		\newpar
		$\displaystyle{\lim\limits_{\substack{y \rightarrow 0}} \frac{\ln(y - 1)}{y} = 1}$
	\end{demonstration}

	\subsection{Dérivée}

	\begin{formula}[Dérivée d'une composée]
		Soit une fonction $u$ dérivable et \textbf{strictement positive} sur un intervalle $I$, on a pour tout $x$ appartenant à cet intervalle :
		\newpar
		$\displaystyle{\ln'(u(x)) = \frac{u'(x)}{u(x)}}$.
	\end{formula}

	\begin{formula}[Dérivée]
		Ainsi, si pour tout $x \in I$ on a $u(x) = x$, on trouve :
		\newpar
		$\displaystyle{\ln'(x) = \frac{1}{x}}$.
	\end{formula}

	\subsection{Variations}

	Avec la dérivée donnée précédemment ainsi que les limites données, il est désormais possible d'obtenir les variations de la fonction logarithme népérien.

	\begin{formula}[Signe et variations]
		\contentwidth[big]
		\includelatexpicture{variations}

		On remarque qu'avec le tableau de variation, il est possible d'obtenir le signe de la fonction (avec le théorème des valeurs
		intermédiaires).
		\newpar
		Ainsi, sur $]0;1[$, $\ln$ est \textbf{strictement négative} et sur $]1;+\infty[$,
		$\ln(x)$ est \textbf{strictement positive} et, comme vu précédemment, $\ln(1) = 0$.
		\newpar
		On observe également les variations de la fonction : strictement croissante sur son ensemble de définition.
	\end{formula}
    %</content>
\end{document}
