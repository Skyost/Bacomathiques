\documentclass[tikz]{standalone}

% Load all required packages for my graphics.
\usepackage{fourier-otf}
\usepackage{fontspec}
\usepackage{tkz-euclide}
\usepackage{catchfilebetweentags}

% 1.5x scale.
\tikzset{
  graphfonctionlabel/.style args={at #1 #2 with #3}{
    postaction={
      decorate, decoration={markings, mark= at position #1 with \node [#2] {#3};}
    }
  },
  every picture/.append style={scale=2, every node/.style={scale=2}}
}

% Input content.
\newcommand{\inputcontent}[1]{\ExecuteMetaData[#1]{content}}



\begin{document}
	%<*content>
	\lesson{terminale}{8}{geometrie-espace}{Géométrie dans l'espace}

	\header{caption}{La géométrie dans l'espace s'applique dans beaucoup de domaines : la modélisation 3D par exemple.}

	\header{description}{La géométrie dans l'espace est une forme de géométrie dans laquelle les objets
		peuvent notamment être des solides. Ce chapitre va vous servir à mieux comprendre
		différentes notions comme la coplanarité, le produit scalaire dans l'espace mais
		aussi les représentations paramétriques ou encore les intersections et orthogonalités.}

	\header{difficulty}{2}

	\section{Vecteurs de l'espace}

	\subsection{Lien avec la géométrie plane}

	\begin{formula}[Définition]
		Un \textbf{vecteur de l'espace} est un vecteur possédant trois coordonnées. De même, un \textbf{point de l'espace} est un point possédant trois coordonnées.
	\end{formula}

	Comme dans le plan, un vecteur de l'espace est caractérisé par \textbf{une norme} (sa ``longueur''), \textbf{un sens}, et \textbf{une direction}.

	\begin{formula}[Opérations sur les vecteurs]
		Soient $\overrightarrow{u} = \begin{pmatrix} x_1 \\ y_1 \\ z_1 \end{pmatrix}$ et $\overrightarrow{v} = \begin{pmatrix} x_2 \\ y_2 \\ z_2 \end{pmatrix}$ deux vecteurs de l'espace et $\lambda$ un réel quelconque.
		\begin{itemize}
			\item On peut additionner deux vecteurs : $\overrightarrow{u} + \overrightarrow{v} = \begin{pmatrix} x_1 + x_2 \\ y_1 + y_2 \\ z_1 + z_2 \end{pmatrix}$.
			\item On peut multiplier un vecteur par un réel : $\lambda \overrightarrow{u} =  \begin{pmatrix} \lambda x_1 \\ \lambda y_1 \\ \lambda z_1 \end{pmatrix}$.
		\end{itemize}
	\end{formula}

	La \textbf{relation de Chasles} est également disponible dans l'espace.

	\begin{formula}[Relation de Chasles]
		Soient $A$, $B$ et $C$ trois points de l'espace. Alors $\overrightarrow{AC} = \overrightarrow{AB} + \overrightarrow{BC}$.
	\end{formula}

	\subsection{Coplanarité}

	Soient $\overrightarrow{u}$ et $\overrightarrow{v}$ deux vecteurs non colinéaires de l'espace et $\overrightarrow{w}$ un autre vecteur de l'espace. Ces vecteurs sont dits \textbf{coplanaires} s'il existe des représentants de ces trois vecteurs dans un même plan. De manière plus formelle :

	\begin{formula}[Définition]
		$\overrightarrow{u}$, $\overrightarrow{v}$ et $\overrightarrow{w}$ sont coplanaires s'il existe deux réels $\alpha$ et $\beta$ tels que $\overrightarrow{w} = \alpha \overrightarrow{u} + \beta \overrightarrow{v}$.
	\end{formula}

	\subsection{Repérage dans l'espace}

	\begin{formula}[Repère de l'espace]
		Soient $\overrightarrow{i}$, $\overrightarrow{j}$ et $\overrightarrow{k}$ trois vecteurs non coplanaires de l'espace et $O$ un point de l'espace.
		Tout point $M$ peut alors être identifié dans ce repère par un unique triplet de réels $(x;y;z)$ tel que :
		\[ \overrightarrow{OM} = x\overrightarrow{i} + y\overrightarrow{j} + z\overrightarrow{k} \]
		On dit alors $(O ; \overrightarrow{i} ; \overrightarrow{j} ; \overrightarrow{k})$ est un \textbf{repère de l'espace} et les \textbf{coordonnées} de $M$ dans ce repère sont alors $(x; y; z)$. Par abus de langage, on notera cela $M = (x; y; z)$.
	\end{formula}

	\begin{tip}[Nom des coordonnées]
		En reprenant les notations précédentes, $x$ est \textbf{l'abscisse}, $y$ est \textbf{l'ordonnée} et $z$ est \textbf{la côte} de $M$.
	\end{tip}

	Dans toute la suite du chapitre, on se placera dans le repère précédent $(O ; \overrightarrow{i} ; \overrightarrow{j} ; \overrightarrow{k})$.

	\begin{formula}[Types de repères]
		\entretitreetliste
		\begin{itemize}
			\item Le repère est dit \textbf{orthogonal} si $\overrightarrow{i}$, $\overrightarrow{j}$ et $\overrightarrow{k}$ sont orthogonaux les uns par rapport aux autres.
			\item Le repère est dit \textbf{normé} si $\overrightarrow{i}$, $\overrightarrow{j}$ et $\overrightarrow{k}$ sont de norme $1$.
			\item Le repère est dit \textbf{orthonormé} si les deux conditions précédentes sont réunies.
		\end{itemize}
	\end{formula}

	\section{Produit scalaire dans l'espace}

	\subsection{Caractérisation}

	Soient $\overrightarrow{u}$ et $\overrightarrow{v}$ deux vecteurs de l'espace et $A$, $B$ et $C$ trois points de l'espace. Il existe un plan qui contient les points $A$, $B$ et $C$ tels que $\overrightarrow{u} = \overrightarrow{AB}$ et $\overrightarrow{v} = \overrightarrow{AC}$. Le produit scalaire $\overrightarrow{u} \cdot \overrightarrow{v}$ est alors égal au produit scalaire $\overrightarrow{AB} \cdot \overrightarrow{AC}$ dans ce plan.
	\newpar
	Toutes les propriétés du produit scalaire du plan sont par conséquent également applicables dans l'espace. Il vous est donc conseillé de relire \href{https://bacomathiqu.es/cours/premiere/geometrie-reperee/#le-produit-scalaire}{le cours de Première} sur le produit scalaire.

	\subsection{Calcul du produit scalaire}

	\begin{formula}[Calcul avec les coordonnées]
		Soient $\overrightarrow{u} = \begin{pmatrix} {x_1} \\ {y_1} \\ {z_1} \end{pmatrix}$ et $\overrightarrow{v} = \begin{pmatrix} {x_2} \\ {y_2} \\ {z_2} \end{pmatrix}$ deux vecteurs de l'espace.
		\newpar
		On a :
		\[ \overrightarrow{u} \cdot \overrightarrow{v} = x_1x_2 + y_1y_2 + z_1z_2 \]
	\end{formula}

	\begin{formula}[Calcul avec un angle]
		Soient $\overrightarrow{u}$ et $\overrightarrow{v}$ deux vecteurs du plan et $\theta$ l'angle orienté entre les deux. On a :
		\[ \overrightarrow{u} \cdot \overrightarrow{v} = \Vert \overrightarrow{u} \Vert \times \Vert \overrightarrow{v} \Vert \times \cos(\theta) \]
	\end{formula}

	\begin{formula}[Calcul un projeté orthogonal]
		Soient $A$, $B$ et $C$ trois points distincts de l'espace. On se place dans le plan défini par ces points. On pose $P$ le projeté orthogonal de $C$ sur $(AB)$. Alors :
		\begin{itemize}
			\item Si $P \in [AB)$ alors $\overrightarrow{AB} \cdot \overrightarrow{AC} = AB \times AP$
			\item Si $P \notin [AB)$ alors $\overrightarrow{AB} \cdot \overrightarrow{AC} = - AB \times AP$
		\end{itemize}
	\end{formula}

	Si on ne possède que les normes de nos vecteurs, il est possible d'utiliser la formule de polarisation.

	\begin{formula}[Formule de polarisation]
		Soient $\overrightarrow{u}$ et $\overrightarrow{v}$ deux vecteurs de l'espace :
		\[ \overrightarrow{u} \cdot \overrightarrow{v} = \frac{1}{2} \left(\Vert \overrightarrow{u} + \overrightarrow{v} \Vert^2 - \Vert \overrightarrow{u} \Vert^2 - \Vert \overrightarrow{v} \Vert^2\right) \]
	\end{formula}

	\begin{tip}[Utilisation des formules]
		Il faut vraiment trouver la formule à utiliser selon l'énoncé de l'exercice.
		\newpar
		Par exemple, si on se trouve dans un repère et que l'on a les coordonnées des vecteurs, on pourra utiliser la formule analytique (la première formule donnée).
		À l'inverse, si on ne possède pas les coordonnées de nos vecteurs mais que l'on possède leur normes, il est possible d'utiliser la formule de polarisation.
		\newpar
		Voici un tableau récapitulatif pour $\overrightarrow{u}$ et $\overrightarrow{v}$ vecteurs de l'espace :
		\newpar
		\begin{whitetabularx}{|X|X|X|}
			\hline
			\textbf{Paramètres} & \textbf{Formule} & \textbf{À utiliser si on possède...} \\
			\hline
			$\overrightarrow{u} = \begin{pmatrix} {x_1} \\ {y_1} \\ {z_1} \end{pmatrix}$ \medskip $\overrightarrow{v} = \begin{pmatrix} {x_2} \\ {y_2} \\ {z_2} \end{pmatrix}$. & $\overrightarrow{u} \cdot \overrightarrow{v} = x_1 \times x_2 + y_1 \times y_2 + z_1 \times z_2$ \medskip (Calcul à partir des coordonnées.) & Les coordonnées de $\overrightarrow{u}$ et $\overrightarrow{v}$. \\
			\hline
			$\theta$ est l'angle orienté entre $\overrightarrow{u}$ et $\overrightarrow{v}$. & $\overrightarrow{u} \cdot \overrightarrow{v} = \Vert \overrightarrow{u} \Vert \times \Vert \overrightarrow{v} \Vert \times \cos(\theta)$ \medskip (Calcul à partir des normes et d'un angle.) & La norme de $\overrightarrow{u}$, la norme de $\overrightarrow{v}$ et l'angle $\theta$ entre les deux vecteurs. \\
			\hline
			$A$ et $B$ sont les deux extrémités de $\overrightarrow{u}$, $A$ et $C$ sont les deux extrémités de $\overrightarrow{v}$, et $P$ est le projeté orthogonal de $C$ sur $(AB)$. & $\overrightarrow{u} \cdot \overrightarrow{v} = \overrightarrow{AB} \cdot \overrightarrow{AC} = \pm AB \times AP$ \medskip $+$ si $P \in [AB)$ et $-$ sinon. \medskip (Calcul à partir d'une projection orthogonale.) & 3 points distincts (qui sont ici $A$, $B$ et $C$). \\
			\hline
			& $\overrightarrow{u} \cdot \overrightarrow{v} = \frac{\Vert \overrightarrow{u} + \overrightarrow{v} \Vert^2 - \Vert \overrightarrow{u} \Vert^2 - \Vert \overrightarrow{v} \Vert^2}{2}$ \medskip (Calcul à partir des normes.) & On possède la norme de $\overrightarrow{u}$, celle de $\overrightarrow{v}$ mais surtout celle de $\overrightarrow{u} + \overrightarrow{v}$. \\
			\hline
		\end{whitetabularx}
	\end{tip}

	\section{Droites de l'espace}

	\subsection{Définition}

	Une droite passant par deux points de l'espace différents $A$ et $B$ peut être définie par ces points. Ainsi la droite de l'espace contenant les points $A$ et $B$ peut se nommer la droite $(AB)$.

	\subsection{Caractérisation}

	\begin{formula}[Vecteur directeur d'une droite]
		Le \textbf{vecteur directeur} d'une droite de l'espace est le vecteur qui ``porte'' (ou qui ``suit'') cette droite.
	\end{formula}

	Plusieurs manières existent pour caractériser une droite de l'espace.

	\begin{formula}[Caractérisation d'une droite de l'espace]
		Soit $\mathcal{D}$ une droite de l'espace passant par un point $A = (x_A, y_A, y_C)$ de vecteur directeur $\overrightarrow{u} = \begin{pmatrix} a \\ b \\ c \end{pmatrix}$.
		\newpar
		Soit $M = (x; y; z)$. On peut caractériser $\mathcal{D}$ de deux manières :
		\begin{itemize}
			\item \textbf{Caractérisation vectorielle :}
			\[ M \in \mathcal{D} \iff \overrightarrow{AM} = \lambda \overrightarrow{u} \]
			où $\lambda \in \mathbb{R}$
			\item \textbf{Caractérisation par système d'équations paramétriques :}
			\[ M \in \mathcal{D} \iff \text{il existe } \lambda \in \mathbb{R} \text{ tel que } \begin{cases} x = x_A + \lambda a \\ y = y_A + \lambda b \\ z = z_A + \lambda c \end{cases} \]
		\end{itemize}
	\end{formula}

	\begin{tip}[Comment déterminer une représentation paramétrique d'une droite ?]
		On a deux cas : soit on a directement un point $A$ appartenant à la droite ainsi qu'un vecteur directeur de cette droite. Dans ce cas, il faut appliquer la deuxième formule donnée précédemment (en remplaçant $x_A$, $y_A$ et $z_A$ par les coordonnées de $A$ et $a$, $b$ et $c$ par les coordonnées du vecteur directeur).
		\newpar
		Soit on nous donne deux points de la droite, disons $A$ et $B$. Ce qui signifie que la droite est de vecteur directeur $\overrightarrow{AB}$ et passe par le point $A$. Encore une fois on utilise la deuxième formule donnée précédemment une fois le vecteur $\overrightarrow{AB}$ calculé (il suffit alors de remplacer $x_A$, $y_A$ et $z_A$ par les coordonnées de $A$ et $a$, $b$ et $c$ par les coordonnées de $\overrightarrow{AB}$).
	\end{tip}

	\subsection{Intersection de deux droites}

	\begin{formula}[Intersection de deux droites]
		Soient $\mathcal{D}_1$ et $\mathcal{D}_2$ deux droites. On a les relations suivantes :
		\begin{itemize}
			\item Si $\mathcal{D}_1$ et $\mathcal{D}_2$ ne sont pas coplanaires, leur intersection \textbf{est vide}.
			\item Si $\mathcal{D}_1$ et $\mathcal{D}_2$ sont coplanaires et parallèles mais pas confondues, leur intersection \textbf{est vide}.
			\item Si $\mathcal{D}_1$ et $\mathcal{D}_2$ sont coplanaires et confondues, leur intersection \textbf{est la droite $\mathcal{D}_1$}.
			\item Si $\mathcal{D}_1$ et $\mathcal{D}_2$ sont coplanaires et non parallèles, leur intersection \textbf{est un point}.
		\end{itemize}
	\end{formula}

	\subsection{Orthogonalité de deux droites}

	\begin{formula}[Définition]
		Deux droites $\mathcal{D}_1$ et $\mathcal{D}_2$ sont \textbf{orthogonales} s'il existe une parallèle à $\mathcal{D}_1$ qui est perpendiculaire à $\mathcal{D}_2$.
	\end{formula}

	\begin{tip}[Relation avec les vecteurs directeurs]
		Ainsi, $\mathcal{D}_1$ et $\mathcal{D}_2$ sont orthogonales si un vecteur directeur de $\mathcal{D}_1$ est orthogonal à un vecteur directeur de $\mathcal{D}_2$ (c'est-à-dire, si leur produit scalaire vaut $0$).
	\end{tip}

	\section{Plans de l'espace}

	\subsection{Définition}

	Soient trois points $A$, $B$ et $C$ non alignés (i.e. tels que les vecteurs $\overrightarrow{AB}$ et $\overrightarrow{AC}$ ne sont pas colinéaires). Alors ces points forment un plan de l'espace qui peut se nommer $(ABC)$.

	\subsection{Caractérisation}

	\begin{formula}[Vecteur normal à un plan]
		On dit qu'un vecteur est \textbf{normal} à un plan s'il est orthogonal à tous les vecteurs de ce plan.
	\end{formula}

	Plusieurs manières existent pour caractériser un plan de l'espace.

	\begin{formula}[Caractérisation d'un plan de l'espace]
		Soit $\mathcal{P}$ un plan de l'espace contenant un point $A$ et soient $\overrightarrow{u}$ et $\overrightarrow{v}$ deux vecteurs de l'espace non-colinéaires mais qui appartiennent à $\mathcal{P}$.
		\newpar
		On se donne un vecteur de l'espace $\overrightarrow{n} = \begin{pmatrix} a \\ b \\ c \end{pmatrix}$ orthogonal à $\overrightarrow{u}$ et $\overrightarrow{v}$ (qui est donc normal à $\mathcal{P}$).
		\newpar
		Soit $M = (x; y; z)$. On peut caractériser $\mathcal{P}$ de deux manières :
		\begin{itemize}
			\item \textbf{Caractérisation vectorielle :}
			\[ M \in \mathcal{P} \iff \text{il existe } \lambda \text{ et } \mu \in \mathbb{R} \text{ tels que } \overrightarrow{AM} = \lambda \overrightarrow{u} + \mu \overrightarrow{v} \]
			\item \textbf{Caractérisation par une équation cartésienne :}
			\newline
			\[ M \in \mathcal{P} \iff \text{il existe } d \in \mathbb{R} \text{ tel que } ax + by + cz + d = 0 \]
		\end{itemize}
	\end{formula}

	\begin{tip}[Comment déterminer une équation cartésienne d'un plan ?]
		Deux cas : soit on a directement un point du plan ainsi qu'un vecteur normal à ce plan. Dans ce cas, on remplace $a$, $b$ et $c$ par les coordonnées de ce vecteur normal (dans la deuxième formule). Il reste à trouver $d$ et pour cela on remplace $x$, $y$ et $z$ par les coordonnées du point donné.
		\newpar
		Soit on a un point du plan et on précise que le plan doit être perpendiculaire à une droite dont la représentation paramétrique est donnée. Dans ce cas le vecteur normal au plan sera un vecteur directeur de cette droite et il faudra encore une fois appliquer la deuxième formule donnée précédemment.
	\end{tip}

	\subsection{Intersections}

	\begin{formula}[Intersection d'un plan et d'une droite]
		Soient $\mathcal{P}$ un plan de l'espace et $\mathcal{D}$ une droite de l'espace :
		\begin{itemize}
			\item Si $\mathcal{P}$ contient $\mathcal{D}$, leur intersection \textbf{est la droite $\mathcal{D}$}.
			\item Si $\mathcal{P}$ ne contient pas $\mathcal{D}$ et que $\mathcal{P}$ et $\mathcal{D}$ sont parallèles, leur intersection \textbf{est vide}.
			\item Si $\mathcal{P}$ ne contient pas $\mathcal{D}$ et que $\mathcal{P}$ et $\mathcal{D}$ ne sont pas parallèles, leur intersection \textbf{est un point}.
		\end{itemize}
	\end{formula}

	\begin{formula}[Intersection de deux plans]
		Soient $\mathcal{P}_1$ et $\mathcal{P}_2$ deux plans de l'espace :
		\begin{itemize}
			\item Si $\mathcal{P}_1$ et $\mathcal{P}_2$ sont confondus, leur intersection \textbf{est le plan $\mathcal{P}_1$}.
			\item Si $\mathcal{P}_1$ et $\mathcal{P}_2$ sont parallèles mais pas confondus, leur intersection \textbf{est vide}.
			\item Si $\mathcal{P}_1$ et $\mathcal{P}_2$ ne sont ni parallèles ni confondus, leur intersection \textbf{est une droite}.
		\end{itemize}
	\end{formula}

	\begin{tip}[Intersection de trois plans]
		L'intersection de trois plans de l'espace peut soit être \textbf{vide}, soit être \textbf{une droite}, soit être \textbf{un point}.
	\end{tip}

	\subsection{Orthogonalités}

	\begin{formula}[Définition]
		Soient $\mathcal{P}$ un plan de l'espace et $\mathcal{D}$ une droite de l'espace. On dit que $\mathcal{D}$ est \textbf{orthogonale} à $\mathcal{P}$ si $\mathcal{D}$ est orthogonale à toutes les droites de ce plan.
	\end{formula}

	\begin{tip}[Relation avec les vecteurs directeurs]
		Ainsi, $\mathcal{D}$ et $\mathcal{P}$ sont orthogonaux si un vecteur directeur de $\mathcal{D}$ est normal à $\mathcal{P}$.
	\end{tip}

	\begin{tip}[Propriétés]
		Plusieurs propriétés découlent de ceci :
		\begin{itemize}
			\item Si deux plans sont parallèles, alors toute droite orthogonale à l'un est orthogonale à l'autre.
			\item Si deux plans sont orthogonaux à une même droite, alors ils sont alors parallèles.
			\item Pour montrer qu'une droite est orthogonale à un plan, il suffit de montrer que cette droite est orthogonale à deux droites non orthogonales de ce plan.
			\item Si deux droites sont parallèles, tout plan orthogonal à l'une est orthogonal à l'autre.
			\item Si deux droites sont orthogonales à un même plan, alors elles sont parallèles entre elles.
		\end{itemize}
	\end{tip}

	\subsection{Plan médiateur}

	\begin{formula}[Définition]
		Soient $\mathcal{P}$ un plan de l'espace, $A$ et $B$ deux points de l'espace.
		\newpar
		$\mathcal{P}$ est un \textbf{plan médiateur} si $\mathcal{P}$ est orthogonal au segment $[AB]$ et passe par le milieu de ce segment.
	\end{formula}
	%</content>
\end{document}
