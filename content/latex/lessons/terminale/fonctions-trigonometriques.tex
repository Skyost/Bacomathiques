\documentclass[12pt, a4paper]{report}

% Packages :

\usepackage[french]{babel}
%\usepackage[utf8]{inputenc}
%\usepackage[T1]{fontenc}
\usepackage[pdfencoding=auto, pdfauthor={Bacomathiques}]{hyperref}
\usepackage{sectsty}
\usepackage[explicit]{titlesec}
\usepackage{xcolor}
%\usepackage{amsmath}
%\usepackage{amssymb}
\usepackage{amsthm}
\usepackage{fourier-otf}
\usepackage{titlesec}
\usepackage{fancyhdr}
\usepackage{catchfilebetweentags}
\usepackage[french, capitalise, noabbrev]{cleveref}
\usepackage[fit, breakall]{truncate}
\usepackage[margin=3cm]{geometry}
\usepackage{tocloft}
\usepackage{tikz}
\usepackage{tocloft}
\usepackage{microtype}
\usepackage{listings}
\usepackage{tabularx}
\usepackage{calc}
\usepackage[export]{adjustbox}
\usepackage[most]{tcolorbox}
\usepackage{standalone}
\usepackage{xlop}
\usepackage{etoolbox}
\usepackage{environ}

\usetikzlibrary{arrows.meta}
\usetikzlibrary{trees}

% Police :

\setmathfont{Erewhon Math}

% Paramètres :

\author{Bacomathiques}
\definecolor{graphe}{HTML}{93c9ff}
\setcounter{MaxMatrixCols}{12}
\setlength{\parindent}{0pt}
\setlength{\fboxsep}{0pt}
%\pdfsuppresswarningpagegroup=1

% Code :

\lstdefinestyle{style}{
	backgroundcolor=\color{white},
	commentstyle=\em\color[HTML]{999988},
	keywordstyle=\bfseries,
	identifierstyle=\normalfont,
	stringstyle=\color[rgb]{0.87, 0.07, 0.27},
	basicstyle=\ttfamily\color{black},
	breakatwhitespace=false,
	breaklines=true,
	captionpos=b,
	keepspaces=true,
	numbers=left,
	numbersep=5pt,
	showspaces=false,
	showstringspaces=false,
	showtabs=false,
	tabsize=2,
	numbers=none
}

\lstset{style=style}
\lstset{
	literate=
	{á}{{\'a}}1
	{à}{{\`a}}1
	{ã}{{\~a}}1
	{é}{{\'e}}1
	{ê}{{\^e}}1
	{í}{{\'i}}1
	{ó}{{\'o}}1
	{õ}{{\~o}}1
	{ú}{{\'u}}1
	{ü}{{\"u}}1
	{ç}{{\c{c}}}1
}

\lstset{
	framextopmargin=10pt,
	framexrightmargin=10pt,
	framexbottommargin=10pt,
	framexleftmargin=10pt,
	xleftmargin=10pt,
	xrightmargin=10pt,
}

% Couleurs :

\definecolor{title}{HTML}{912c21}
\definecolor{section}{HTML}{1c567d}
\definecolor{subsection}{HTML}{2980b9}

\definecolor{rule}{HTML}{c4c4c4}

\definecolor{formula}{HTML}{ebf3fb}
\definecolor{formula-left}{HTML}{3583d6}

\definecolor{tip}{HTML}{dcf3d8}
\definecolor{tip-left}{HTML}{26a65b}

\definecolor{demonstration}{HTML}{fff8de}
\definecolor{demonstration-left}{HTML}{f1c40f}

\definecolor{exercise}{HTML}{e0f2f1}
\definecolor{exercise-left}{HTML}{009688}

\definecolor{correction}{HTML}{e0f7fa}
\definecolor{correction-left}{HTML}{00bcd4}

\definecolor{toc}{HTML}{fceae9}
\definecolor{toc-left}{HTML}{e74c3c}
\definecolor{toc-dark}{HTML}{87281f}

% Titres :

\renewcommand{\thesection}{\Roman{section} - }
\renewcommand{\thesubsection}{\arabic{subsection}. }

\newcommand{\sectionstyle}{\normalfont\LARGE\bfseries\color{section}}
\titleformat{\section}{\sectionstyle}{\thesection #1}{0pt}{}
\titleformat{name=\section, numberless}{\sectionstyle}{#1}{0pt}{}

\newcommand{\subsectionstyle}{\normalfont\Large\bfseries\color{subsection}}
\titleformat{\subsection}{\subsectionstyle}{\thesubsection #1}{0pt}{}
\titleformat{name=\subsection, numberless}{\subsectionstyle}{#1}{0pt}{}

\titlelabel{\thetitle\ }

% Table des matières :

\addto\captionsfrench{\renewcommand\contentsname{}}
\renewcommand{\cftsecpagefont}{\color{toc-dark}}
\renewcommand{\cftsubsecpagefont}{\color{toc-dark}}
\renewcommand{\cftsecleader}{\cftdotfill{\cftdotsep}}
\renewcommand{\cftsecfont}{\bfseries}
\renewcommand{\cftsecpagefont}{\bfseries\color{toc-dark}}
\setlength{\cftbeforetoctitleskip}{0pt}
\setlength{\cftaftertoctitleskip}{0pt}
\setlength{\cftsecindent}{0pt}
\setlength{\cftsubsecindent}{20pt}
\setlength{\cftsubsecnumwidth}{20pt}

% Commandes :

\newcommand{\newpar}{\\[\medskipamount]}
\newcommand{\lesson}[3]{%
	\newcommand{\level}{#1}%
	\newcommand{\id}{#2}%
	\hypersetup{pdftitle={#3}}
	\begin{center}%
		\includegraphics[width=150px]{\imagespath/bacomathiques}%
		
		\vspace{30pt}%
		{\Huge\color{title} #3}%
		
		\vspace{10pt}%
		{Bacomathiques --- \href{https://bacomathiqu.es/cours/#1/#2}{\color{section} https://bacomathiqu.es}}%
		
		\vspace{20pt}%
	\end{center}%
	\begin{toc}
		\tableofcontents%
	\end{toc}
	\thispagestyle{empty}%
	\newpage%
	\setcounter{page}{1}%
}
\newcommand{\imagespath}{../../images}
\newcommand{\lessonimagespath}{\imagespath/lessons/\level/\id/}
\newcommand{\includelatexpicture}[2][\textwidth - 100pt]{%
	\begin{center}%
		\resizebox{#1}{!}{%
			\input{\lessonimagespath#2}%
		}%
	\end{center}%
	\medskip%
}
\newcommand{\includeimage}[1]{%
	\begin{center}%
		\includegraphics{\lessonimagespath#1}%
	\end{center}%
	\medskip%
}
\newcommand{\includerepresentation}[1]{%
	\begin{center}%
		\setlength{\fboxrule}{0.5pt}%
		\href{https://www.geogebra.org/m/#1}{\includegraphics[width=\textwidth-1pt,fbox]{\lessonimagespath#1}}%
	\end{center}%
}
\newcommand{\floor}[1]{\lfloor #1 \rfloor}

\makeatletter
\newcommand\inputcontent{\@ifstar{\inputcontent@star}{\inputcontent@nostar}}
\newcommand{\inputcontent@star}[1]{%
	\ExecuteMetaData[#1]{content}%
}
\newcommand{\inputcontent@nostar}[1]{%
	\newpage%
	\inputcontent@star{#1}%
}
\makeatother

\let\oldsection\section
\renewcommand\section{\clearpage\oldsection}
\newcommand{\contentwidth}[1][medium]{}

% En-têtes :

\pagestyle{fancy}

\renewcommand{\sectionmark}[1]{\markboth{\thesection \ #1}{}}

\fancyhead[R]{\truncate{0.23\textwidth}{\color{title}\thepage}}
\fancyhead[L]{\truncate{0.73\textwidth}{\color{title}\leftmark}}
\fancyfoot[C]{\scriptsize \href{https://bacomathiqu.es/cours/\level/\id}{\texttt{bacomathiqu.es}}}

\makeatletter
\patchcmd{\f@nch@head}{\rlap}{\color{rule}\rlap}{}{}
\patchcmd{\headrule}{\hrule}{\color{rule}\hrule}{}{}
\makeatother

% Environnements :

\newenvironment{nosummary}{}{}
\newcommand{\tcolorboxtitle}[2]{\setlength{\fboxsep}{2.5pt}\hspace{-10pt}\colorbox{#1-left}{\hspace{8pt}\MakeUppercase{#2} \hspace{2pt} \includegraphics[height=0.8em]{\imagespath/bubbles/#1}\hspace{5pt}}}
\newcommand{\tcolorboxsubtitle}[2]{\ifstrempty{#2}{}{\textcolor{#1-left}{\large#2}\\[\medskipamount]}}
\tcbset{
	frame hidden,
	boxrule=0pt,
	boxsep=0pt,
	enlarge bottom by=8.5pt,
	enhanced jigsaw,
	boxed title style={sharp corners,boxrule=0pt,coltitle={white},titlerule=0pt},
	fonttitle=\fontsize{6pt}{6pt}\bfseries\boldmath,
	top=10pt,
	right=10pt,
	bottom=10pt,
	left=10pt,
	arc=0pt,
	outer arc=0pt,
}
\newtcolorbox{toc}[1][]{
	colback=toc,
	borderline west={3pt}{0pt}{toc-left},
	title=\tcolorboxtitle{toc}{Table des matières},
	colbacktitle=toc,
	before upper={\tcolorboxsubtitle{toc}{#1}}
}
\newtcolorbox{formula}[1][]{
	colback=formula,
	borderline west={3pt}{0pt}{formula-left},
	title=\tcolorboxtitle{formula}{À retenir},
	colbacktitle=formula,
	before upper={\tcolorboxsubtitle{formula}{#1}}
}
\newtcolorbox{tip}[1][]{
	colback=tip,
	borderline west={3pt}{0pt}{tip-left},
	title=\tcolorboxtitle{tip}{À lire},
	colbacktitle=tip,
	before upper={\tcolorboxsubtitle{tip}{#1}}
}
\newtcolorbox{demonstration}[1][]{
	colback=demonstration,
	borderline west={3pt}{0pt}{demonstration-left},
	title=\tcolorboxtitle{demonstration}{Démonstration},
	colbacktitle=demonstration,
	before upper={\tcolorboxsubtitle{demonstration}{#1}}
}

\NewEnviron{whitetabularx}[1]{%
	\renewcommand{\arraystretch}{2.5}
	\colorbox{white}{%
		\begin{tabularx}{\textwidth}{#1}%
			\BODY%
		\end{tabularx}%
	}%
}

% Longueurs :

\newlength{\espacetitreliste}
\setlength{\espacetitreliste}{-16pt}
\newcommand{\entretitreetliste}{\vspace{\espacetitreliste}}

\begin{document}
	%<*content>
	\lesson{terminale}{4}{fonctions-trigonometriques}{Les fonctions trigonométriques}

	\header{caption}{On retrouve les fonctions trigonométriques dans le son et l'acoustique.}

	\header{description}{La trigonométrie est une branche des mathématiques qui traite des relations
		entre distances et angles dans les triangles et des fonctions trigonométriques telles
		que sinus, cosinus et tangente. Ce cours permettra d'étudier ces fonctions et d'en
		visualiser des propriétés (périodicité, parité, valeurs remarquables, et plus encore).}

	\header{difficulty}{3}

	\section{Sinus et cosinus}

	\subsection{Définition}

	Dans tout le cours, le plan sera muni d'un repère orthonormé $(O,\ \overrightarrow{i} ;\ \overrightarrow{j})$. Il sera également muni d'un cercle $\mathcal{C}$ appelé \textbf{cercle trigonométrique} de centre $O$ et de rayon $1$ orienté dans le sens inverse des aiguilles d'une montre (c'est le \textbf{sens direct}) :

	\includerepresentation{t52gsb2h}

	\begin{formula}[Cosinus et sinus]
		Soit $M$ un point quelconque situé sur le cercle $\mathcal{C}$ faisant un angle $x$ avec l'axe des abscisses. Les coordonnées de $M$ sont :
		\begin{itemize}
			\item L'abscisse de $M$ appelée \textbf{cosinus} est notée $\cos(x)$.
			\item L'ordonnée de $M$ appelée \textbf{sinus} est notée $\sin(x)$.
			\item Pour tout $x \in \mathbb{R}$, on a $-1 \leq \cos(x) \leq 1$ et $-1 \leq \sin(x) \leq 1$.
		\end{itemize}
	\end{formula}

	\subsection{Périodicité}

	Les fonctions sinus et cosinus sont périodiques de période $2\pi$.

	\begin{formula}[Périodicité]
		Ainsi pour tout $x$ réel et $k$ entier relatif :
		\begin{itemize}
			\item $\cos(x) = \cos(x + 2k\pi)$
			\item $\sin(x) = \sin(x + 2k\pi)$
		\end{itemize}
	\end{formula}

	\begin{tip}
		Concrètement, cela signifie que $\cos(x) = \cos(x + 2\pi) = \cos(x + 4\pi) = \dots = \cos(x + 2k\pi)$ et idem pour $\sin(x)$.
	\end{tip}

	\subsection{Formules de trigonométrie}

	\begin{formula}[Formules]
		On a les relations suivantes pour tout $x \in \mathbb{R}$ :
		\begin{itemize}
			\item $\cos(-x) = \cos(x)$ (la fonction cosinus est \textbf{paire})
			\item $\sin(-x) = -\sin(x)$ (la fonction sinus est \textbf{impaire})
			\item $\cos(\pi + x) = -\cos(x)$
			\item $\sin(\pi + x) = -\sin(x)$
			\item $\cos(\pi - x) = -\cos(x)$
			\item $\sin(\pi - x) = \sin(x)$
			\item $\cos \left(\frac{\pi}{2} + x \right) = -\sin(x)$
			\item $\sin \left(\frac{\pi}{2} + x \right) = \cos(x)$
			\item $\cos \left(\frac{\pi}{2} - x \right) = \sin(x)$
			\item $\sin \left(\frac{\pi}{2} - x \right) = \cos(x)$
			\item $\cos(x + y) = \cos(x) \times \cos(y) - \sin(x) \times \sin(y)$
			\item $\sin(x + y) = \sin(x) \times \cos(y) + \cos(x) \times \sin(y)$
			\item $\cos(x)^2 + \sin(x)^2 = 1$
		\end{itemize}
	\end{formula}

	\begin{tip}[Retrouver les formules]
		Il n'est aucunement demandé de mémoriser ces formules (sauf les trois dernières). Cependant, il doit être possible de les retrouver à l'aide du cercle trigonométrique. Ainsi, prenons l'exemple de $\cos(x + \pi)$ :

		\includerepresentation{xghkwjkf}

		On remarque que l'ordonnée reste la même (le sinus est le même). Cependant, on a bien une abscisse opposée. On a retrouvé la formule $\cos(x + \pi) = -\cos(x)$.
	\end{tip}

	\subsection{Résolution d'équations}

	Il est possible de résoudre des équations incluant des sinus et des cosinus.

	\begin{formula}[Résolution d'équations]
		Soient $x$ et $y$ deux réels. On a les relations suivantes :
		\begin{itemize}
			\item $\cos(x) = \cos(y) \iff \text{ il existe } k \in \mathbb{Z} \text{ tel que } \begin{cases} y = x + 2k\pi \\ \text{ou} \\ y = -x + 2k\pi\end{cases}$
			\item $\sin(x) = \sin(y) \iff \text{ il existe } k \in \mathbb{Z} \text{ tel que } \begin{cases} y = x + 2k\pi \\ \text{ou} \\ y = \pi - x + 2k\pi\end{cases}$
		\end{itemize}
	\end{formula}

	Comme précédemment, ces formules peuvent se retrouver à l'aide du cercle trigonométrique.

	\subsection{Fonctions réciproques}

	\begin{formula}[Définition]
		Soient $x$ et $y$ $\in \mathbb{R}$, on admettra qu'il existe une \textbf{fonction réciproque} à $\cos$ (notée $\arccos$) et une \textbf{fonction réciproque} à $\sin$ (notée $\arcsin$). On a les relations suivantes pour tout $x \in [0; 2\pi]$ et $y \in [-1; 1]$ :
		\begin{itemize}
			\item $\cos(x) = y \iff x = \arccos(y)$
			\item $\sin(x) = y \iff x = \sin(y)$
		\end{itemize}
	\end{formula}

	Cela signifie qu'à tout $x \in [0; 2\pi]$, la fonction $\arccos$ y associe son \textbf{antécédent} $y$ par rapport à $\cos$ (pareil pour $\arcsin$ avec $\sin$).

	\begin{tip}[Exemple]
		$\cos(0) = 1$, $\arccos(1) = 0$ et $\sin(\frac{\pi}{2}) = 1$, $\arcsin(1) = \frac{\pi}{2}$.
	\end{tip}

	Ces fonctions (accessibles depuis la calculatrice) peuvent également être utilisées pour résoudre certains types d'équations.

	\section{Étude des fonctions trigonométriques}

	\subsection{Dérivée}

	\begin{formula}[Dérivée d'une composée]
		Soit une fonction $u$ dérivable sur un intervalle $I$, on a pour tout $x$ appartenant à cet intervalle :
		\begin{itemize}
			\item $\cos'(u(x)) = -u'(x)\sin(u(x))$
			\item $\sin'(u(x)) = u'(x)\cos(u(x))$
		\end{itemize}
	\end{formula}

	\begin{formula}[Dérivée]
		Ainsi, si pour tout $x \in I$ on a $u(x) = x$, on trouve :
		\begin{itemize}
			\item $\cos'(x) = -\sin(x)$
			\item $\sin'(x) = \cos(x)$
		\end{itemize}
	\end{formula}

	\subsection{Signe et variations}

	L'étude du signe des dérivées des fonctions trigonométriques permet d'obtenir les variations de celles-ci. Nous allons donc voir le signe et les variations de ces fonctions.

	\begin{formula}[Signe et variation de la fonction cosinus]
		\contentwidth[big]
		\includelatexpicture{variations-cos}

		Veuillez noter que ce tableau est périodique de période $2\pi$.
	\end{formula}

	\begin{formula}[Signe et variation de la fonction sinus]
		\contentwidth[big]
		\includelatexpicture{variations-sin}

		Ce tableau est également périodique de période $2\pi$.
	\end{formula}

	\subsection{Limite}

	Les fonctions trigonométriques ont pour particularité de \textbf{ne pas admettre de limite} en $\pm\infty$. Ceci provenant du fait que ces fonctions sont périodiques et que leur valeur oscille entre $-1$ et $1$.

	\subsection{Valeurs remarquables}

	\begin{formula}[Valeurs remarquables]
		Voici un tableau regroupant quelques valeurs remarquables de sinus et de cosinus :
		\newpar
		\begin{whitetabularx}{|X|l|l|}
			\hline
			\textbf{Valeur de $x$ (à $2k\pi$ près, $k \in \mathbb{Z}$)} & \textbf{Valeur de $\cos(x)$} & \textbf{Valeur de $\sin(x)$} \\
			\hline
			$0$ & $1$ & $0$ \\
			\hline
			$\frac{\pi}{6}$ & $\frac{\sqrt{3}}{2}$ & $\frac{1}{2}$ \\
			\hline
			$\frac{\pi}{4}$ & $\frac{\sqrt{2}}{2}$ & $\frac{\sqrt{2}}{2}$ \\
			\hline
			$\frac{\pi}{3}$ & $\frac{1}{2}$ & $\frac{\sqrt{3}}{2}$ \\
			\hline
			$\frac{\pi}{2}$ & $0$ & $1$ \\
			\hline
			$\frac{2\pi}{3}$ & $-\frac{1}{2}$ & $\frac{\sqrt{3}}{2}$ \\
			\hline
			$\frac{3\pi}{4}$ & $-\frac{\sqrt{2}}{2}$ & $\frac{\sqrt{2}}{2}$ \\
			\hline
			$\frac{5\pi}{6}$ & $-\frac{\sqrt{3}}{2}$ & $\frac{1}{2}$ \\
			\hline
			$\pi$ & $-1$ & $0$ \\
			\hline
		\end{whitetabularx}
	\end{formula}

	\subsection{Représentation graphique}

	À l'aide de toutes les informations et valeurs données précédemment, il est possible d'établir une représentation graphique de la fonction cosinus :
	\includerepresentation{zhwqmkjd}

	De même pour la fonction sinus :
	\includerepresentation{gkpmaugu}
	On remarque sur ces graphiques plusieurs propriétés données : parité, signe, périodicité, etc.
	%</content>
\end{document}
